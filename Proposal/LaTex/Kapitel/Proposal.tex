\newpage
\restoregeometry
\pagenumbering{arabic}

\section{Motivation}
\label{s:Einleitung}
Im Rahmen des von der European Space Agency (ESA) gestarteten Erdbeobachtungsprogramms Copernicus werden unterschiedlichste Fernerkundungsdaten unter anderem von den 
Satelliten der Sentinel-Reihe aufgenommen. Diese zeichnen sich durch hohe räumliche und zeitliche Auflösung aus und eigenen sich daher für eine Vielzahl von Anwendungen 
in der Geoinformatik \cite{s1_product_definition}. Besonders das Krisen- und Risikomanagement kann von diesen Daten profitieren. Bevor die Rohdaten korrekt interpretiert 
und genutzt werden können, bedarf es häufig einer teilweise aufwendigen Vorverarbeitung und entsprechender Infrastrukturen. Um den Anwendern den Zugriff 
auf einsatzbereite Daten so einfach wie möglich zu machen und so eine vereinfachte Sicht auf die Daten zu erlauben, kann der OGC API - Processes - Part 1: Core Standard 
genutzt werden, um eine Datenschnittstelle zu entwerfen, welche reich an Interaktionsmöglichkeiten ist. Dieser Standard bedient sich des RESTful Paradigmas und ist von
Konzepten des OGC Web Processing Service 2.0 beeinflusst. Eine vollständige Implementierung des Letzteren ist jedoch nicht mehr erforderlich \cite{ogc_api_processes_core}. 
Eine API bietet sowohl für Entwickler als auch Anwender einige Vorteile. Als Entwickler kann auf viele Aspekte der Prozessierung sowie auf die Eigenschaften der Resultate Einfluss 
genommen werden. Für Anwender bietet eine API, in Abgrenzung zu simpleren Diensten, häufig die Möglichkeit Anfragen zu parametrisieren und so exakt auf ihre Fragestellung anzupassen.

\section{Zielsetzung}
\label{s:Zielsetzung}
Ziel der Arbeit ist das Implementieren eines leichtgewichtigen, OGC API - Processes - Part: 1 Core Standard konformen Application Programming Interface (API). Um einen 
praktischen Bezug zu schaffen, soll die API einen Prozess anbieten, welcher Überschwemmungsmonitoring auf Basis von Copernicus-Daten ermöglicht. Die API wird sämtliche 
Vorverarbeitungsschritte durchführen und als Resultat einsatzbereite Geodaten liefern, die sich für das Überschwemmungsmonitoring eignen. Neben der eigentlichen Implementierung 
der API soll untersucht werden, welche Möglichkeiten der Kopplung von Copernicus Daten mit der zu implementierenden API bestehen. Ein besonderes Augenmerk liegt dabei auf 
Aspekten wie Einfachheit, Wartbarkeit und Erweiterbarkeit der API und der Eignung des OGC API - Processes - Part 1: Core Standards für die Entwicklung von Datenschnittstellen 
zu Copernicus-Daten mit zahlreichen Interaktionsmöglichkeiten. 

\newpage
\section{Methoden}
\label{s:Methoden}
Damit eine möglichst leichtgewichtige, simple, aber erweiterbare API entworfen werden kann, wird die Programmiersprache Python und das Web Framework 
Flask zum Einsatz kommen. Für die eigentliche Prozessierung sollen möglichst nur bewährte Programme und Python Packages verwendet werden, um eine möglichst hohe 
Wartbarkeit zu gewährleisten. Die Versionierung erfolgt mit Git. Das Überschwemmungsmonitoring soll auf Basis von Radardaten der Sentinel-1 Mission erfolgen, da diese 
wetter- und tageszeitunabhängig Messungen durchführen können \cite{s1_product_definition}. Die notwendigen Kalibrierungen und Filterungen sollen ebenfalls Teil der 
bereitgestellten Prozessierung sein. Anschließend soll der Normalized Difference Sigma-Naught Index (NDSI) berechnet werden \cite{flood_proxy_mapping_ndsi}. 
Aus diesem können mithilfe eines automatischen Grenzwertverfahrens Überflutungsmasken abgeleitet werden.

