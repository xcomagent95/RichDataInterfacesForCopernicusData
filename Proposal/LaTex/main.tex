%Präambel
\documentclass[12pt, a4paper]{article} %Dokumentenklasse: Seitenformat und Schriftgröße
\usepackage{mathptmx}
\usepackage[T1]{fontenc} %Pakete für die Zeichenbelegung und Schriftart Times New Roman
\usepackage[utf8]{inputenc} %Paket für die Zeichenkodierung (WICHTIG: auf Kodierung des Editors achten!!!)
\usepackage[german, ngerman]{babel}	%Paket für die Silbentrennung (letztgenannte Sprache ist Standard)
\usepackage{csquotes} %Paket für Anführungszeichen
\usepackage{graphics, graphicx} %Paket für die Einbindung von Grafiken
\usepackage{amssymb, amsmath, amstext} %Pakete für Formeln
\usepackage[left=2.5cm, right=4cm, top=2.5cm, bottom=2.5cm]{geometry} %Einstellung der Seitenränder
\setlength{\parindent}{0em} %kein Einzug
\usepackage[onehalfspacing]{setspace}	%Paket für die Zeilenabstände

\usepackage{url} %Paket für Hyperlinks
\usepackage{acronym} % Paket für Abkürzungen
\usepackage[breaklinks=true, linktocpage=true]{hyperref} %Software-Paket für Verlinken von Querverweisen, URLs, DOIs etc.
\hypersetup{ %modifiziere Link-Farben im Dokument
    colorlinks,
    citecolor=black,
    filecolor=blue,
    linkcolor=blue,
    urlcolor=blue
}
\usepackage{cite}

%Dokument
\begin{document}
\newgeometry{left=2cm, right=2cm, top=2.5cm, bottom=2.5cm}
\thispagestyle{empty}
\begin{center}
\includegraphics[width=6cm]{Bilder/wwu_logo.png} \hspace{1.3cm} 
\includegraphics[width=6cm]{Bilder/ifgi_logo.png}

\vspace{3cm}

Westfälische Wilhelms-Universität Münster

Institut für Geoinformatik 

\vspace{3cm}

\textbf{\large Proposal für eine Bachelorarbeit} 

im Fach Geoinformatik
\vspace{1cm}

\textbf{\large Rich Data Interfaces for Copernicus Data}

\vspace{1cm}


Themensteller: Prof. Dr. Albert Remke\\
Betreuer: Dr. Christian Knoth, Dipl.-Geoinf. Matthes Rieke\\
Ausgabetermin: tbd.\\
Abgabetermin: tbd.\\
\vspace{0.5cm}
Vorgelegt von: Alexander Nicolas Pilz\\
Geboren: 06.12.1995\\
Telefonnummer: 0176 96982246\\
E-Mail-Adresse:	apilz@uni-muenster.de\\
Matrikelnummer: 512 269\\
Studiengang: Bachelor Geoinformatik\\
Fachsemester: 6. Semester\\

\end{center}
\newpage
\restoregeometry
\pagenumbering{arabic}

\section{Motivation}
\label{s:Einleitung}
Im Rahmen des von der European Space Agency (ESA) gestarteten Erdbeobachtungsprogramms Copernicus werden unterschiedlichste Fernerkundungsdaten unter anderem von den 
Satelliten der Sentinel-Reihe aufgenommen. Diese zeichnen sich durch hohe räumliche und zeitliche Auflösung aus und eigenen sich daher für eine Vielzahl von Anwendungen 
in der Geoinformatik \cite{s1_product_definition}. Besonders das Krisen- und Risikomanagement kann von diesen Daten profitieren. Bevor die Rohdaten korrekt interpretiert 
und genutzt werden können, bedarf es häufig einer teilweise aufwendigen Vorverarbeitung und entsprechender Infrastrukturen. Um den Anwendern den Zugriff 
auf einsatzbereite Daten so einfach wie möglich zu machen und so eine vereinfachte Sicht auf die Daten zu erlauben, kann der OGC API - Processes - Part 1: Core Standard 
genutzt werden, um eine Datenschnittstelle zu entwerfen, welche reich an Interaktionsmöglichkeiten ist. Dieser Standard bedient sich des RESTful Paradigmas und ist von
Konzepten des OGC Web Processing Service 2.0 beeinflusst. Eine vollständige Implementierung des Letzteren ist jedoch nicht mehr erforderlich \cite{ogc_api_processes_core}. 
Eine API bietet sowohl für Entwickler als auch Anwender einige Vorteile. Als Entwickler kann auf viele Aspekte der Prozessierung sowie auf die Eigenschaften der Resultate Einfluss 
genommen werden. Für Anwender bietet eine API, in Abgrenzung zu simpleren Diensten, häufig die Möglichkeit Anfragen zu parametrisieren und so exakt auf ihre Fragestellung anzupassen.

\section{Zielsetzung}
\label{s:Zielsetzung}
Ziel der Arbeit ist das Implementieren eines leichtgewichtigen, OGC API - Processes - Part: 1 Core Standard konformen Application Programming Interface (API). Um einen 
praktischen Bezug zu schaffen, soll die API einen Prozess anbieten, welcher Überschwemmungsmonitoring auf Basis von Copernicus-Daten ermöglicht. Die API wird sämtliche 
Vorverarbeitungsschritte durchführen und als Resultat einsatzbereite Geodaten liefern, die sich für das Überschwemmungsmonitoring eignen. Neben der eigentlichen Implementierung 
der API soll untersucht werden, welche Möglichkeiten der Kopplung von Copernicus Daten mit der zu implementierenden API bestehen. Ein besonderes Augenmerk liegt dabei auf 
Aspekten wie Einfachheit, Wartbarkeit und Erweiterbarkeit der API und der Eignung des OGC API - Processes - Part 1: Core Standards für die Entwicklung von Datenschnittstellen 
zu Copernicus-Daten mit zahlreichen Interaktionsmöglichkeiten. 

\newpage
\section{Methoden}
\label{s:Methoden}
Damit eine möglichst leichtgewichtige, simple, aber erweiterbare API entworfen werden kann, wird die Programmiersprache Python und das Web Framework 
Flask zum Einsatz kommen. Für die eigentliche Prozessierung sollen möglichst nur bewährte Programme und Python Packages verwendet werden, um eine möglichst hohe 
Wartbarkeit zu gewährleisten. Die Versionierung erfolgt mit Git. Das Überschwemmungsmonitoring soll auf Basis von Radardaten der Sentinel-1 Mission erfolgen, da diese 
wetter- und tageszeitunabhängig Messungen durchführen können \cite{s1_product_definition}. Die notwendigen Kalibrierungen und Filterungen sollen ebenfalls Teil der 
bereitgestellten Prozessierung sein. Anschließend soll der Normalized Difference Sigma-Naught Index (NDSI) berechnet werden \cite{flood_proxy_mapping_ndsi}. 
Aus diesem können mithilfe eines automatischen Grenzwertverfahrens Überflutungsmasken abgeleitet werden.


\newpage
\thispagestyle{empty}
\begin{thebibliography}{1}

\bibitem{tutorial_on_sar}
A. Moreira, M. Younis, P. Prats-Iraola, G. Krieger, I. Hajnsek und K. P. Papathanassiou (2013, April 17). A Tutorial on Synthetic Aperture Radar [Online]. Verfügbar unter: 
https://www.researchgate.net/publication/257008464\_A\_Tutorial\_on\_Synthetic\_Aperture\_Radar
(Zugriff am: 6. Juni 2022).

\bibitem{einfuehrung_in_fernerkundung}
J. Albertz, Einführung in die Fernerkundung, 4. Auflage Darmstadt: Wissenschaftliche Buchgesellschaft, 2009

\bibitem{history_of_copernicus}
Europäische Kommission (2018, Oktober 06). Copernicus: 20 years of History [Online]. Verfügbar unter: 
https://www.copernicus.eu/en/documentation/information-material/signature-esafrance-collaborative-ground-segment
(Zugriff am: 13. Juni 2022).

\bibitem{sentinel_overview}
European Space Agency (2018). Sentinels - Space for Copernicus [Online]. Verfügbar unter: 
https://www.d-copernicus.de/daten/satelliten/daten-sentinels/
(Zugriff am: 13. Juni 2022).

\bibitem{what_is_copernicus}
Europäische Kommission (2019). What is Copernicus [Online]. Verfügbar unter: 
https://www.copernicus.eu/en/documentation/information-material/brochuresbrochures
(Zugriff am: 13. Juni 2022).

\bibitem{sentinel_1_overview}
ESA Communications (2012, März). Sentinel-1 ESA's Radar Observatory Mission for GMES Operational Services [Online]. Verfügbar unter: 
https://sentinel.esa.int/web/sentinel/missions/sentinel-1/overview
(Zugriff am: 13. Juni 2022).

\end{thebibliography}
\newpage
\thispagestyle{empty}
\section*{Plagiatserklärung des Studierenden}	

Hiermit versichere ich, dass die vorliegende Arbeit über 

\begin{center}
Rich Data Interfaces for Copernicus Data
\end{center}

selbstständig verfasst worden ist, dass keine anderen 
Quellen und Hilfsmittel als die angegebenen benutzt worden sind und dass die Stellen der 
Arbeit, die anderen Werken – auch elektronischen Medien – dem Wortlaut oder Sinn nach 
entnommen wurden, auf jeden Fall unter Angabe der Quelle als Entlehnung kenntlich gemacht 
worden sind.\\

\vspace{0.75cm}
\parbox{17em}{\hrulefill} \\
Alexander Pilz, Münster den \today

\vspace{0.75cm}

Ich erkläre mich mit einem Abgleich der Arbeit mit anderen Texten zwecks Auffindung von 
Übereinstimmungen sowie mit einer zu diesem Zweck vorzunehmenden Speicherung der Arbeit 
in eine Datenbank einverstanden.\\

\vspace{0.75cm}
\parbox{17em}{\hrulefill} \\
Alexander Pilz, Münster den \today

\end{document}