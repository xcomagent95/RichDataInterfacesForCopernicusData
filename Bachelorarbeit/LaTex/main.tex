%Präambel
\documentclass[12pt, a4paper, twoside]{article} %Dokumentenklasse: Seitenformat und Schriftgröße
\usepackage{mathptmx}
\usepackage[T1]{fontenc} %Pakete für die Zeichenbelegung und Schriftart Times New Roman
\usepackage[utf8]{inputenc} %Paket für die Zeichenkodierung (WICHTIG: auf Kodierung des Editors achten!!!)
\usepackage[german, ngerman]{babel}	%Paket für die Silbentrennung (letztgenannte Sprache ist Standard)
\usepackage{csquotes} %Paket für Anführungszeichen
\usepackage{graphics, graphicx} %Paket für die Einbindung von Grafiken
\usepackage{amssymb, amsmath, amstext} %Pakete für Formeln
\usepackage[left=3cm, right=2cm, top=2.5cm, bottom=2.5cm]{geometry} %Einstellung der Seitenränder
\setlength{\parindent}{0em} %kein Einzug
\usepackage[onehalfspacing]{setspace}	%Paket für die Zeilenabstände
\usepackage{adjustbox}
\usepackage{float}
\usepackage{ragged2e}
\usepackage{url} %Paket für Hyperlinks
\usepackage{acronym} % Paket für Abkürzungen
\usepackage[breaklinks=true, linktocpage=true]{hyperref} %Software-Paket für Verlinken von Querverweisen, URLs, DOIs etc.
\hypersetup{ %modifiziere Link-Farben im Dokument
    colorlinks,
    citecolor=black,
    filecolor=black,
    linkcolor=black,
    urlcolor=blue
}
\usepackage{cite}

\usepackage{algorithm}
\usepackage[noend]{algorithmic}

\floatname{algorithm}{Programmablauf}

\usepackage[utf8]{inputenc}
\usepackage{listings}
\usepackage{fancyvrb}

\usepackage[T1]{fontenc}         % Not always necessary, but recommended.
% End of standard header.  What follows pertains to the problem at hand.

\usepackage{xcolor}
\usepackage{listings}
\colorlet{punct}{red!60!black}
\definecolor{background}{HTML}{EEEEEE}
\definecolor{delim}{RGB}{20,105,176}
\colorlet{numb}{magenta!60!black}
\lstdefinelanguage{json}{
    basicstyle=\normalfont\ttfamily,
    numbers=left,
    numberstyle=\tiny,
    stepnumber=1,
    numbersep=8pt,
    showstringspaces=false,
    breaklines=false,
    frame=single,
    literate=
     *{0}{{{\color{numb}0}}}{1}
      {1}{{{\color{numb}1}}}{1}
      {2}{{{\color{numb}2}}}{1}
      {3}{{{\color{numb}3}}}{1}
      {4}{{{\color{numb}4}}}{1}
      {5}{{{\color{numb}5}}}{1}
      {6}{{{\color{numb}6}}}{1}
      {7}{{{\color{numb}7}}}{1}
      {8}{{{\color{numb}8}}}{1}
      {9}{{{\color{numb}9}}}{1}
      {:}{{{\color{punct}{:}}}}{1}
      {,}{{{\color{punct}{,}}}}{1}
      {\{}{{{\color{delim}{\{}}}}{1}
      {\}}{{{\color{delim}{\}}}}}{1}
      {[}{{{\color{delim}{[}}}}{1}
      {]}{{{\color{delim}{]}}}}{1},
}

\lstdefinestyle{Python}{
    language        = Python,
    basicstyle      = \ttfamily\scriptsize,
    keywordstyle    = \color{blue},
    keywordstyle    = [2] \color{teal}, % just to check that it works
    stringstyle     = \color{orange},
    commentstyle    = \color{red}\ttfamily,
    tabsize         = 2,
    numberstyle     = \tiny
}

\lstdefinestyle{HTML}{
    language        = HTML,
    basicstyle      = \ttfamily\scriptsize,
    keywordstyle    = \color{blue},
    keywordstyle    = [2] \color{teal}, % just to check that it works
    stringstyle     = \color{orange},
    commentstyle    = \color{red}\ttfamily,
    tabsize         = 2,
    numberstyle     = \tiny
}

\lstdefinestyle{JSON}{
    language        = JSON,
    basicstyle      = \ttfamily\scriptsize,
    keywordstyle    = \color{blue},
    keywordstyle    = [2] \color{teal}, % just to check that it works
    stringstyle     = \color{orange},
    commentstyle    = \color{red}\ttfamily,
    tabsize         = 2,
    numberstyle     = \tiny
}

%Dokument
\begin{document}

\lstset{
    frame       = single,
    numbers     = left,
    showspaces  = false,
    showstringspaces    = false,
    captionpos  = t,
    caption     = \section,
    tabsize         = 2,
    numberstyle     = \tiny,
    firstnumber=1
}

\newgeometry{left=2cm, right=2cm, top=2.5cm, bottom=2.5cm}
\thispagestyle{empty}
\begin{center}
\includegraphics[width=6cm]{Bilder/wwu_logo.png} \hspace{1.3cm} 
\includegraphics[width=6cm]{Bilder/ifgi_logo.png}

\vspace{3cm}

Westfälische Wilhelms-Universität Münster

Institut für Geoinformatik 

\vspace{3cm}

\textbf{\large Bachelorarbeit} 

im Fach Geoinformatik
\vspace{1cm}

\textbf{\large Rich Data Interfaces for Copernicus Data}

\vspace{1cm}

Themensteller: Prof. Dr. Albert Remke\\
Betreuer: Dr. Christian Knoth, Dipl.-Geoinf. Matthes Rieke\\
Ausgabetermin: 20.05.2022\\
Abgabetermin: 22.08.2022\\
\vspace{0.5cm}
Vorgelegt von: Alexander Nicolas Pilz\\
Geboren: 06.12.1995\\
Telefonnummer: 0176 96982246\\
E-Mail-Adresse:	apilz@uni-muenster.de\\
Matrikelnummer: 512 269\\
Studiengang: Bachelor Geoinformatik\\
Fachsemester: 6. Semester\\

\end{center}
\pagenumbering{arabic}
\newpage
\tableofcontents 
\newpage
\listoffigures
\listoftables
\newpage
\newpage
\thispagestyle{empty}
\section*{Abkürzungsverzeichnis}
\begin{acronym}
    \acro{API}{Application Programming Interface}
    \acro{OGC}{Open Geospatial Consotium}
    \acro{SAR}{Synthetic Apeture Radar}
    \acro{ESA}{European Space Agency}
    \acro{GMES}{Global Monitoring for Environmental Security}
    \acro{CAMS}{Copernicus Atmosphere Monitoring Service}
    \acro{CMEMS}{Copernicus Marine Environment Monitoring Service}
    \acro{CLMS}{Copernicus Land Monitoring Service}
    \acro{EMS}{Emergency Management Service}
    \acro{C3S}{Climate Change Service}
    \acro{SM}{Stripmap Mode}
    \acro{SLC}{Single Look Complex}
    \acro{GRD}{Ground Rage Detected}
    \acro{OSW}{Ocean Swell Spectra}
    \acro{OWI}{Ocean Wind Field}
    \acro{RVL}{Radial Surface Velocity}
    \acro{NDSI}{Normalized Difference Sigma-Naught Index}
    \acro{DIAS}{Data and Information Access Services}
    \acro{JSON}{Java Script Object Notation}
    \acro{URL}{Uniform Resource Locator}
\end{acronym}
\newpage
\restoregeometry
\section{Einleitung}
\subsection{Test}

\newpage
\restoregeometry
\section{Grundlagen}
\subsection{Radarfernerkundung}
Bei der Radarfernerkundung werden vom Radarsystem in regelmäßigen Abständen elektromagnetische Signale ausgesandt. Nach dem Senden eines Signals 
(Chrip) folgt ein Zeitfenster, indem die Plattform auf Echos des ausgesandten Signals wartet.
Trifft das ausgesandte Signal auf eine Oberfläche, zum Beispiel 
die Erdoberfläche, wird ein Bruchteil in Richtung Empfänger reflektiert und als Echo vom Fernerkundungssystem empfangen \cite{tutorial_on_sar}.

Die Radarfernerkundung gehört zu den aktiven Fernerkundungsmethoden da hier im Gegensatz zur optischen Fernerkundung nicht nur 
von Oberflächen reflektierte Strahlung von anderen Strahlungsquellen wie der Sonne aufgenommen wird, sondern das Fernerkundungssystem 
selbst als Strahlungsquelle dient. Messungen können daher tageszeitunabhängig erfolgen. Bildgebende Radarsysteme werden auf mobilen Plattformen 
montiert und blicken seitlich auf die zu beobachtende Oberfläche. Die Flugrichtung wird Azimut und die Blickrichtung als Slant Range 
bezeichnet \cite{tutorial_on_sar} (Abbildung 1). 

Die Eigenschaften des reflektierten Signals hängen sowohl von Parametern des Aufnahmesystems als von Parametern der reflektierenden Oberfläche ab.
So werden in der Radarfernerkundung verschiedenen Frequenzbänder verwendet, welche sich in Frequenz und Wellenlänge unterscheiden. Da sich die Wechselwirkungen zwischen Signalen 
unterschiedlicher Frequenzbänder und den reflektierenden Oberflächen unterscheidet können so unterschiedliche Aspekte der beobachteten Oberflächen hervorgehoben werden. 
Dabei kommen in der Regel Wellenlängen von 0.75m bis 120m zum Einsatz (siehe Tabelle \ref{frequenzbaender}).
Mit einer größeren Wellenlänge kann ein Medium auch tiefer durchdrungen werden. Außerdem werden Wolken, Dunst und Rauch durchdrungen was den zusätzlich Vorteil bietet
wetterunabhängig Messungen durchführen zu können \cite{einfuehrung_in_fernerkundung}.

\begin{table}[H]
    \caption{Gängige Frequenz-Bänder in der Radarfernerkundung \cite{tutorial_on_sar}}
    \centering
    \begin{tabular}{c|c c c c c c c } 
        Frequenzband & Ka & Ku & X & C & S & L & P\\ 
        \hline
        Frequenz (GHz) & 40-25 & 17.6-12 & 12-7.5 & 7.5-3.75 & 3.75-2 & 2-1 & 0.5-0.25\\ 
        Wellenlänge (cm) & 0.75–1.2 & 1.7–2.5 & 2.5–4 & 4–8 & 8–15 & 15–30 & 60–120\\ 
    \end{tabular}
    \label{frequenzbaender}
\end{table}

Die Durchdringungstiefe hängt auch von der Dielektrizitätskonstante, also der Leitfähigkeit, ab. Ist diese groß, kommt es zu starken Reflektionen und die 
Durchdringungstiefe ist gering. Die Rauigkeit ist eine Eigenschaft der reflektierenden Oberfläche und hat großen Einfluss auf das reflektierte Signal. Ist diese im Verhältnis
zur verwandten Wellenlänge gering so kommt es zu spiegelnden Reflektionen und nur ein geringer Anteil des kehrt zum Empfänger zurück. Je diffuser
die Reflektion mit zunehmender Rauigkeit wird umso größer ist der Anteil des Signals welcher zum Empfänger zurückgeworfenen Signals. Doch auch die Form und Exposition der Oberfläche 
nimmt Einfluss auf das reflektierte Signal. So werden Flächen je nach Neigung unterschiedlich stark bestrahlt. Ist eine dem System abgewandte Fläche steiler geneigt als der Depressionswinkel 
liegen Sie sogar im Radarschatten und werden gar nicht bestrahlt \cite{einfuehrung_in_fernerkundung}. 
Zusätzlich ist die Polarisation der ausgesandten und empfangenen Signale bei der Messung ausschlaggebend. Sie können horizontal oder 
vertikal polarisiert sein. Dies führt zu vier möglichen Polarisationsmodi für das Senden und das Empfangen nämlich HH, VV, HV und VH. Auch die 
Polarisation sorgt für eine unterschiedliche Wiedergabe von beobachteten Objekten und kann somit verwendet werden, um bestimmte Aspekte hervorzuheben
 \cite{einfuehrung_in_fernerkundung}. Die Auflösung entlang des Azimut unterscheidet sich von der Auflösung in Blickrichtung. Die Auflösung in Azimutrichtung wird von 
der Antennenlänge bestimmt da diese festlegt wie lange die Reflektionen eines Objektes empfangen werden. Die Antennenlänge kann bauartbedingt nicht beliebig gesteigert werden.
Die Bauart der Antenne bestimmt auch den Abstrahlwinkel $\Theta_a$ und somit die Ausdehnung am Boden eines Impulses in Azimutrichtung. Diese nimmt mit zunehmender Entfernung
zu, während die Auflösung abnimmt.
Die Auflösung in Blickrichtung hängt von der Bandbreite ab welche sich aus der Sendefrequenz und der Signaldauer. Die Ausdehnung des beobachteten Gebietes 
in Blickrichtung hängt von der Laufzeit des ausgesandten Signales ab. Die Objekte werden abhängig von ihrer Entfernung zur Antenne verzerrt wiedergegeben da nahegelegene 
Objekte von der Wellenfront schneller durchlaufen werden. Dieser Unterschied zwischen Schrägdistanz und Bodendistanz lässt sich jedoch nahezu vollständig korrigieren 
\cite{einfuehrung_in_fernerkundung}. Die bisher beschriebenen Systeme werden auch als Systeme mit realer Apertur bezeichnet und eignen sich nur für geringe Flughöhen da hier 
der Abstand zwischen Antenne und Oberfläche gering ist. Bei Radarsystemen mit einer synthetischen Apertur wird durch die Bewegung des Sensors in Azimutrichtung die 
wirksame Antennenlänge rechnerisch verlängert indem die reflektierten Signale eines beobachteten Objektes von verschiedenen Standpunkten und unterschiedlichen Zeitpunkten 
miteinander korreliert werden. So können hohe Azimutauflösungen erzielt werden. Solche Systeme eigenen sich auch für den Einsatz auf Satelliten \cite{einfuehrung_in_fernerkundung}. 

\begin{figure}[H]
    \centering
    \includegraphics[width=0.6\textwidth]{Bilder/SAR_Prinzip.png}
    \caption{Prinzip eines SAR Fernerkundungssystems \cite{tutorial_on_sar}}
    \label{sar_prinzip}
\end{figure}

Solche Systeme können in unterschiedlichen Aufnahmeverfahren arbeiten. Das einfachste dieser Verfahren ist das Stripmap Verfahren bei dem nur ein Aufnahmestreifen
kontinuierlich aufgenommen wird. Breitere Aufnahmestreifen können mit dem ScanSAR Verfahren erzielt werden. Dabei werden unter verschiedenen Depressionswinkeln, 
in Blickrichtung und zeitversetzt mehrere Subaufnahmestreifen erzeugt. Im Vergleich zum Stripmap Verfahren ist Auflösung jedoch geringer. 
Wird eine höhere Auflösung benötigt kann das Spotlight Verfahren zum Einsatz kommen, bei dem eine fixe Region über einen längeren Zeitraum hinweg beobachtet wird. Dies führt zu 
einer sehr langen wirksamen Antenne. Angepasste Verfahren oder Mischformen können je Beobachtungsszenario zum Einsatz kommen (siehe Abbildung \ref{sar_scan_modi})\cite{tutorial_on_sar}. 

\begin{figure}[H]
    \centering
    \includegraphics[width=\textwidth]{Bilder/SAR_Modi.png}
    \caption{Aufnahmeverfahren SAR Systemen \cite{tutorial_on_sar}}
    \label{sar_scan_modi}
\end{figure}

Im Gegensatz zu optischen Aufnahmeverfahren liefern die Rohdaten 
einer Befliegung mit Radarsensoren noch keine Bilddaten. Um Bilder zu erzeugen, bedarf es zunächst einer komplexen Verarbeitung der aus Amplitude und Phase bestehenden 
reflektierten Signale. Dabei werden die Daten entlang des Azimuts und der Blickrichtung gefiltert. In der Regel repräsentieren die Pixelwerte eines aus Radardaten 
abgeleiteten Bildes die Reflektivität des korrespondierenden Bodenelements. Mittels Geocodierung kann das so entstandene Bild verortet werden. Zusätzlich können diverse 
Kalibrierungen vorgenommen werden. Dazu gehören Verfahren welche Rauscheffekte minimieren, die geometrischen Eigenschaften verbessern oder die Interpretation der Bilder 
erleichtern \cite{tutorial_on_sar}. 

\subsection{Copernicus Programm}
\subsubsection{Ziele}
Das Copernicus-Programm ging aus dem Global Monitoring for Environmental Security Programm (GMES) Programm hervor welches 1998 mit dem Ziel initiiert wurde um Europa 
zu ermöglichen eine führende Rolle bei der Lösung von weltweiten Problemen im Kontext Umwelt und Klima zu verschaffen. Teil dieser Bestrebungen ist der Aufbau eines 
leistungsfähigen Programms zur Erdbeobachtung. 2012 wurde das GMES-Programm zum Copernicus-Programm umbenannt \cite{history_of_copernicus}.
Erklärte Ziele des Copernicus-Programmes ist das Überwachen der Erde um den Schutz der Umwelt sowie Bemühungen von Katastrophen- und Zivilschutzbehörden zu 
unterstützen. Gleichzeitig soll die Wirtschaft im Bereich Raumfahrt und der damit verbundenen Dienstleistungen unterstützt und Chancen für neue Unternehmungen geschaffen
werden \cite{copernicus_regulation}.

\subsubsection{Aufbau}
Das Copernicus-Programm besteht aus Weltraum, In-Situ- und Service-Komponente. 
Zur Weltraum-Komponente gehören die verschiedenen Satellitenmissionen sowie Bodenstationen welche für den Betrieb sowie die Steuerung und Kalibrierung der 
Satelliten sowie der Verarbeitung und Validierung der Daten verantwortlich sind \cite{copernicus_regulation}. \\ 
Sentinel-1 Satelliten sind mit bildgebenden Radarsystemen ausgerüstet und beobachten wetter- und tageszeitunabhängig Land-, Wasser- und Eismassen, um unter andrem das 
Krisenmanagement zu unterstützen.
Satelliten der Sentinel-2 Mission führen hochauflösende, multispektrale Kameras mit und liefern weltweit optische Fernerkundungsdaten. \\
Altimetrische und radiometrische Daten von Land- und Wasserflächen werden von der Sentinel-3 Satellitenmission gesammelt während spektrometrische Daten zur 
Überwachung der Luftqualität von Sentinel-4 und 5 Satelliten erfasst werden.
Ozeanografische Daten sollen von den Sentinel-6 Satelliten geliefert werden \cite{sentinel_overview}.

Die In-Situ-Komponente sammelt Daten von See-, luft- und landbasierten Sensoren sowie geografische und geodätische Referenzdaten. Die harmonisierten Daten 
werden verwendet, um die Daten der Weltraum-Komponente zu verifizieren oder zu korrigieren. Gleichzeitig können räumliche oder thematische Lücken in der 
Datenabdeckung gefüllt werden \cite{copernicus_regulation}\cite{what_is_copernicus}. \\

Zur Service-Komponente gehören unterschiedliche Dienste, welche jeweils auf Themengebiet abgestimmt sind und Daten in hoher Qualität bereitstellen.
Der Copernicus Atmosphere Monitoring Service (CAMS) soll Informationen zur Luftqualität und der chemischen Zusammensetzung der Atmosphäre liefern. 
Daten bezüglich des Zustands und der Dynamik der Meere und deren Ökosysteme lassen sich über den Copernicus Marine Environment Monitoring Service (CMEMS) beziehen. 
Informationen zur Flächennutzung und Bodenbedeckung werden vom Copernicus Land Monitoring Service (CLMS) bereitgestellt. 
Um eine nachhaltige Klimapolitik planen und umsetzen zu können stellt der Copernicus Climate Change Service (C3S) aktuelle sowie historische Klimadaten bereit.  
Um den Zivilschutzbehörden schnelle Reaktionen auf Umweltkatastrophen zu ermöglichen, stellt der Emergency Management Service (EMS) entsprechende Fernerkundungsdaten 
breit. Ähnliche Daten können von europäischen Zoll- und Grenzschutzbehörden über den Copernicus Security Service bezogen werden
\cite{copernicus_regulation}\cite{what_is_copernicus}.

\subsubsection{Sentinel 1}
Die Sentinel-1 Satellitenmission liefert wetter- und tageszeitunabhängige Radardaten der Erdoberfläche. Die Mission besteht aus zwei Satelliten, Sentinel-1 A und B,
sowie einer Bodenkomponente welche für Steuerung und Kalibrierung und Datenverarbeitung verantwortlich ist. Die Satelliten tragen als Hauptinstrument ein 
bildgebendes Radar mit synthetischer Apertur welches im C-Frequenzband arbeitet. Es stehen zwei Polarisationsmodi, Single (HH, VV) oder Dual (HH+HV, VV+VH),
zur Verfügung \cite{sentinel_1_definition}. 
Die Erfassung von Daten kann in vier Aufnahmemodi erfolgen welche sich in Auflösung, Streifenbreite und Anwendungsszenario unterscheiden (siehe Tabelle \ref{aufnahmemodi_sentinel_1}). 
Der Standardmodus ist der Stripmap Modus (SM) bei dem Aufnahmestreifen mit einer kontinuierlichen Folge von Signalen abgetastet wird \cite{sentinel_1_definition}.
Die Aufnahmemodi Interferometric Wide Swath Mode (IW) und Extra-Wide Swath Mode (EW) arbeiten im TOPSAR Verfahren mit drei beziehungsweise
fünf Sub-Aufnahmestreifen um ein größeres Gebiet aber in geringerer Auflösung aufnehmen zu können. TOPSAR ist eine Abwandlung des ScanSAR Verfahrens bei 
dem die Antenne zusätzlich in Azimut-Richtung vor und zurück bewegt wird, um die radiometrische Qualität der resultierenden Bilder zu verbessern. 
Wenn der Wave Modus (WV) zu Einsatz kommt werden kleine, Vignetten genannte, Szenen im Stripmap Verfahren aufgenommen. Sie werden in regelmäßigen Abständen und
wechselnden Depressionwinkeln aufgenommen (siehe Abbildung \ref{sar_modi_sentinel_1})\cite{tutorial_on_sar}\cite{sentinel_1_definition}.   

\begin{figure}[H]
    \centering
    \includegraphics[width=\textwidth]{Bilder/Aquisition_Modes.png}
    \caption{Aufnahmemodi der Sentinel-1 Mission \cite{sentinel_1_overview}}
    \label{sar_modi_sentinel_1}
\end{figure}

\begin{center}
\begin{table}[H]
    \caption{Eigenschaften der Aufnahmemodi der Sentinel-1 Mission \cite{sentinel_1_overview}}
    \centering
    \begin{tabular}{c|c c c c } 
        Modus & IW & WV & SM & EW \\ 
        \hline
        Polarisation & Dual & Single & Dual & Dual \\ 
        Azimutauflösung (m) & 20 & 5 & 5 & 40 \\
        Rage-Auflösung (m) & 5 & 5 & 5 & 20 \\
        Streifenbreite (km) & 250 & 20x20 & 80 & 410\\
    \end{tabular}
    \label{aufnahmemodi_sentinel_1}
\end{table}
\end{center}

Beide Satelliten befinden sich auf einem polnahen, sonnensynchronen Orbit. Ein Zyklus dauert 12 Tage, in denen die Erde 175 umrundet wird. Da er sich um ein Satellitenpaar
handelt welches als Tandem die Erde umrundet wird ein Punkte alle sechs Tage von einem der Satelliten überflogen. Das System kann eine zuverlässige globale und systematische
Abdeckung liefern. Dabei können im IW Modus alle relevanten Land-, Wasser- und Eismassen alle zwölf Tage vollständig von einem Satelliten erfasst werden. 
In Krisensituationen können nach Bedarf innerhalb von zweieinhalb und fünf Tagen Daten erfasst werden \cite{sentinel_1_overview}. 

Nach dem Erfassen der Daten und Übersenden an eine Bodenstation werden diverse Vorverarbeitungsschritte vorgenommen in die sowohl interne also auch externe
Parameter einfließen. Daraus ergeben sich diverse Produkte welche sich durch Aufnahmemodus (IW, SM, EW und WV), Produkt-Typ sowie durch ihre 
Auflösung (Full-, High-, und Medium-Resolution) unterscheiden. Single Look Complex (SLC) Produkte sind im wesentlichen kalibrierte Rohdaten in denen Amplitude und Phase nicht
zur Reflektivität kombiniert wurden und die geometrische Auflösung sich in Azimut- und Blickrichtung unterscheidet. Ground Rage Detected (GRD) Produkte bilden hingegen die 
Reflektivität ab und haben eine annähern quadratische geometrische Auflösung. Die Reflektivität wird in der logarithmischen Maßeinheit Dezibel (dB) angegeben. Die Korrektur 
der Schrägdistanz in Blickrichtung erfolgt durch Projektion auf einen Ellipsoiden. \cite{sentinel_1_definition}. Aus den Level-1 Produkten, SLC und GRD, können die 
Level-2 Produkte OSW, OWI und RVL abgeleitet werden.

\subsubsection{Datenzugang}
Die Daten des Copernicus-Programmes sollen einer möglichst breiten Nutzergruppe möglichst einfach zugänglich gemacht werden. Sie sollen frei zugänglich und kostenlos angeboten 
werden \cite{copernicus_regulation}. Daten der Sentinel-1, 2, 3 und 5 können über das von der ESA betriebene Copernicus Open Access Hub bezogen werden. Datensätze können sowohl
auf der Webseite als auch mithilfe einer API gesucht und heruntergeladen werden. Der Zugang zu Daten der Sentinel-3, 6 und 4 sowie weiterer Satelliten können über das 
dem Copernicus Open Access Hub ähnlichen EUMETCast bezogen werden.
In Ergänzung zu diesen Quellen werden Daten von fünf privaten, in Kooperation mit dem Copernicus-Programm stehenden Unternehmen in unterschiedlichen Formen bereitgestellt. 
Diese als Data and Information Access Services (DIAS) bezeichneten Zugänge stellen unverarbeitete und abgeleitete Daten sowie Werkzeuge zur Analyse zur Verfügung \cite{dias_factsheet}.
Da die DIAS kommerziell betrieben werden müssen einige Dienste und Werkzeuge bezahlt werden während Nutzer sich lediglich am Copernicus Open Access Hub oder EUMETCast 
registrieren müssen. Zu erwähnen ist das die DIAS Zugriff auf die gesamten Daten gestatten. Aus dem Copernicus Open Access Hub lassen sich nur Teile der Daten synchron beziehen.
In der Regel müssen Daten welche älter als einen Monat sind aus dem Archiv wiederhergestellt werden. Dieser Vorgang kann einige Zeit in Anspruch nehmen. 


\subsection{Überschwemmungsmonitoring}
Um Wasserflächen und damit auch überflutete Areale auf Radarbildern zu erkennen können die Reflektionseigenschaften von Wasserflächen genutzt werden. Das Wasser eine 
sehr niedrige Rauigkeit besitzt kommt beim Aufprall eines Radarsignals zu einer spiegelnden Reflektion und nur ein sehr geringer Teil des Signals wird zum Empfänger 
zurückgeworfen. In den resultierenden Bildern äußert sich dieser Umstand in niedrigen Refelektivitätswerten. 
Um die Areale mit niedrigen Reflektionswerten zu detektieren können Verfahren genutzt werden, welche aus den Histogrammen der Bilder einen Schwellwert ermitteln.
Um die Ergebnisse einer solchen Schwellwertbestimmung zu verbessern, sollten die Radardaten, zum Beispiel Sentinel-1 IW GRD, zusätzlich Kalibriert werden. 
So können die genaue Kenntnis über die tatsächliche Flugbahn des Satelliten dazu betragen die geografische Genauigkeit zu verbessern. Diese kann zusätzlich durch Verfahren 
wie die Diffentialentzerrung gesteigert werden die die durch das Relief enstandenen Lagefehler ausgleicht \cite{einfuehrung_in_fernerkundung}.
Die radiometrische Genauigkeit kann gesteigert werden indem zum Beispiel thermisches Rauschen aus den Daten entfernt wird und die Reflektivitätswerte zum 
sogenannten $\sigma_0$-Wert umgerechnet werden. Dieser repräsentiert den Querschnitt der Reflektivität für eine normierte Fläche am Boden \cite{radiometric_calibration_of_S1_level1_products}.
Dieses Maß erlaubt zudem das Vergleichen unterschiedlicher Radaraufnahmen.  
Auch sollte ein Speckle-Filter zum Einsatz kommen um. Dieser reduziert körnige Bildstrukturen welche auf homogenen Flächen in Radarbildern auftreten und die 
rechnerische Bildauswertung erschweren können. \cite{einfuehrung_in_fernerkundung}\cite{sentinel_1_flood_mapping_tutorial}.  
Auf Basis des Schwellwertes kann ein Binärisierung des Bilder durchgeführt werden. Die entstehenden Werte würden überflutete beziehungsweise trocken liegende 
Areale repräsentieren \cite{sentinel_1_flood_mapping_tutorial}.
Die Binärisierung kann direkt auf Basis der Radaraufnahme der Überflutung, oder auf abgeleiteten Daten erfolgen. So können zum Beispiel das Radaraufnahme der überflutung mit 
einer überflutungsfreien Referenzaufnahme kombiniert zum Normalized Difference Sigma-Naught Index (NDSI) \cite{flood_proxy_mapping_ndsi}. Dabei die Reflektivitätswerte von 
zwei unterschiedlichen Zeitpunkten zu einem Maß verrechnet welches als stärke der Veränderung interpretiert werden kann. 

\begin{equation}
    NDSI = \frac{\sigma_0^f-\sigma_0^r}{\sigma_0^f+\sigma_0^r}
\end{equation} 

Dieses Maß bewegt sich zwischen $-1$ und $1$ wobei 
Werte um $0$ für identische Reflektionswerte an beiden Zeitpunkten und daher für geringe Veränderung stehen. 
Aufgrund der Reflektionseigenschaften von Wasserflächen deuten Werte nahe $-1$ auf überflutete Areale hin \cite{flood_proxy_mapping_ndsi}. 

\subsection{Schnittstellen}
\subsection{OGC und OGC Standards}
Das Open Geospatial Consotium (OGC) widmet sich der Aufgabe die Entwicklung von internationalen Standards und unterstützender Dienste welche die Interoperabilität im 
Bereich der Geoinformatik verbessern voranzutreiben. Das OGC soll dabei offene Systeme und Techniken verbreiten welche es erlauben Dienste und Prozesse mit geobezug
in Kreisen der Informatik verbreiten und die Nutzung von interoperabler und kommerzieller Software fördern.


\subsection{OGC API - Processes - Part 1: Core}
Der OGC API - Processes - Part 1: Core Standrad soll das Bereitstellen von aufwendingen Prozessierungsaufgaben und ausführbaren Prozessen welche über eine webbasierte 
Programmierschittstelle von anderen Programmen aufgerufen und gestartet werden können unterstützen \cite{ogc_api_processes_core}. Der Standard ist dabei von Konzepten des 
OGC Web Processing Service 2.0 Interface Standards beeinflusst und bedient sich des RESTful Paradigmas sowie der Java Script Object Notation (JSON)
\subsection{Evaluationskriterien}



\newpage
\restoregeometry
\section{Implementierung}
\subsection{Softwarestack}
Die prototypische Entwicklung eines leichtgewichtigen Rich Data Interface für Copernicus-Daten erfolgte im Rahmen dieser Arbeit mit der Programmiersprache Python.
Diese ist nicht nur aufgrund ihrer Einfachheit vorteilhaft sondern erlaubt auch den Zugriff auf eine große Zahl von Packages für die unterschiedlichsten 
Anwendungsfälle. 
Weite Teile des Rich Data Interface wurden mithilfe des Flask-Frameworks umgesetzt. Dieses erlaubt das schnelle Entwickeln einer leichtgewichtigen API.   

\subsection{Struktur}
Die Anwendung ist in vier Python-Scripte aufgeteilt. Im api.py Script ist die API der Anwendung definiert. 
Die geordnete Abarbeitung der angelegten Jobs werden vom Script processing.py gesteuert. Die eigentlichen Prozesse sowie 
Hilfsfunktionen befinden sich im utils.py Script. Das test.py Script kann dazu genuzt werden um die Stabilität und Standardkonformität der API zu testen.

Diese Scripte verwalten Dateien in einem einfachen Verzeichnissystem. Templates für statische Ressourcen befinden sich im Verzeichnis \textit{templates/}. HTML-Dateien 
befinden sich im Verzeichnis \textit{templates/html/} und JSON-Dateien im Verzeichnis \textit{templates/json/}. Das Unterverzeichnis \textit{templates/json/processes/} enthält die 
Beschreibungen der von der Anwendung angebotenen Prozesse. 
Die Anwendung erlaubt das persistente hinterlegen von Sentinel-1 Datensätzen um zeitaufwendiges Herunterladen zu vermeiden. Diese Datensätze können im Verzeichnis \textit{data/} abgelegt werden. Jeder Sentinel-1 Datensatz enthält eine .kml-Datei welche 
Metadaten zum Datensatz enthält. Diese werden im Unterverzeichnis \textit{data/coverage/} abgelegt. 
Jeder angelegte Job, also jede auszuführende Instanz eines Prozesses erhält ein einzigartiges Verzeichnis innerhalb des Verzeichnisses \textit{jobs/}. In diesem 
Verzeichnis befinden sich eine Status- und Job-Datei sowie ein Footprint. Neben diesen Dateien enthält jedes Job-Verzeichnis ein Unterverzeichnis \textit{results/} in dem
die Ergebnisse des jeweiligen Jobs abgelegt werden.

Insgesamt handelt es sich bei der protypischen Implemntierung eines Rich Data Interfaces for Copernicus-Daten also um eine Anwendung welche Überschwemmungsmasken und Daten zur Hochwasseranalyse als Ressourcen bereitstellt.
Zugrill auf diese Ressourcen erhalten Nutzer über eine OGC API - Processes - Part 1: Core standardkonforme API. 

\subsection{Ressourcen}
Ressourcen sind die üer die Endpoints der API bereitgestellten Informationen und Dateien. Sie können in unterschiedlichen Repräsentationen vorliegen.
Die meisten angebotenen Ressourcen können um Media-Type \textit{text/html}, also als HTML-Dateien oder im Media-Type \textit{application/json} also als 
JSON-Dateien angefragt werden. Um aus diesen Dateien einen Response zu generieren werden .html-Dateien zuvor mit der Methode \textit{render\_template()}, welcher auch 
dynamische Inhalte übergeben werden können gerendert während .json-Dateien zunächst mit der Methode \textit{jsonify()} bearbeitet werden. Beide genannten Methoden
geben ein Response-Objekt zurück welches versandt werden kann.  

Die Struktur dieser Ressourcen ist in Schemata beschrieben. Diese definieren neben den Bezeichnungen für Elemente auch ihre Datentypen und 
ob sie optional sind oder nicht. Jede Ressource enthält eine Verknüpfungen zu sich selbst mit der Relation \textit{self} und eine Verknüpfung 
zur Ressource im jeweils anderen Media-Type mit der Relation \textit{alternate}. ??Übersicht über Ressourcen??

\subsection{Requirements Classes für Encodings}
In der Requirements Class JSON wird definiert welche Ressourcen im Media-Type \textit{application/json} angefragt werden können. Dazu gehören alle Responses der 
Endpunkte API Landing Page, API Definition, Conformance Deklaration, Prozess Liste, Prozess Beschreibung, Prozess Ausführung und Job Status welche mit dem 
HTTP-Statuscode 200 versandt werden. Da die prototypische Implementierung auch die Endpunkte Job Liste und Coverage bereitstellt können die korrespondierenden
Ressourcen auch im Media-Type \textit{application/json} angefragt werden.\\

In der Requirements-Class HTML werden analog zu Requirements Class JSON jene Ressourcen definiert welche im Media-Type \textit{text/html} angefragt werden können. Jedoch
entfällt in dieser Requirements-Class die Einschränkung auf bestimmte Endpunkte und alle Responses welche mit dem HTTP-Statuscode 200 versandt werden müssen den 
Media-Type \textit{text/html} unterstützen.\\
Stellen Endpoints ihre Ressourcen sowohl den Media-Type \textit{application/json} als auch \textit{text/html} zur Verfügung so können Nutzer diesen über den optional Parameter
\textit{f} oder \textit{content\_type} spezifizieren. Wird kein Media-Type über diese Parameter spezifiziert so wird standardmäßig der Media-Type \textit{text/html} verwendet. \\


\subsection{Requirements Class Core}
\subsubsection{HTTP 1.1}
Die Umsetzung der Requirements-Class HTTP 1.1 (RFC 2616) verlangt das die API exklusiv das HTTP 1.1 unterstützt. 
Falls die API ebenfalls HTTPS unterstützt muss ebenfalls HTTP over TLS (RFC 2818) eingehalten werden. 
Das Flask-Framework nutzt standardmäßig das HTTP 1.0. Teil des Flask-Frameworks ist die WSGI Bibliothek Werkzeug welche
das Implementieren von Webanwendungen erlaubt. Um die verwendete HTTP-Version von 1.0 auf 1.1 umzustellen müssen Variablen 
in Werkzeug angepasst werden. Nach dem Import der Module WSGIRequestHandler und BaseWSGIServer kann in beiden die 
Version des HTTP Protokolls angepasst werden (siehe Anhang \ref{appendixconfWerkzeug}). 

In dieser Requirements-Class werden zudem alle HTTP-Statuscodes gelistet die Nutzer von einer standardkonformen Implementierung mindestens erwarten können. 
\begin{table}[H]
    \caption{Vorgesehene HTTP-Statuscodes \cite{ogc_api_processes_core}}
    \centering
    \begin{tabular}{c c} 
        HTTP-Statuscode & Bedeutung\\ 
        \hline
        200 & OK\\
        201 & Created\\
        204 & No Content\\
        400 & Bad Request\\
        401 & Unauthorized\\
        403 & Forbidden\\
        404 & Not Found\\
        405 & Method Not Allowed\\
        406 & Not Acceptable\\
        410 & Gone\\
        429 & Too Many Requests\\
        500 & Internal Server Error\\
        501 & Not Implemented\\
    \end{tabular}\label{httpcodes}
\end{table}
Alle erfolgreichen Anfragen welche eine Resource liefern mit dem HTTP-Statuscode 200 beantwortet. Die Verwendung nicht zulässiger HTTP-Methoden resultieren 
in Antworten mit dem Status-Code 405 während Anfragen für nicht unterstütze Media-Types mit dem Status-Code 406 beantwortet werden. Kommt es zu Fehlern bei der Ausführung 
des Programmcodes antwortet die Anwendung mit dem HTTP-Statuscode 500. Werden durch eine Anfrage Ressourcen neu erzeugt oder nicht gefunden antwortet die Anwendung mit 
den HTTP-Statuscodes 201 beziehungsweise 404. Der Standard erlaubt die Nutzung weiterer HTTP-Statuscodes \cite{ogc_api_processes_core}.

\subsubsection{Limit Paramter}
Der \textit{limit}-Parameter wird von den Endpoints Process List und Job List unterstützt. Mit ihm kann gesteuert werden wie viele Elemente im Response gelistet werden. 
Der \textit{limit}-Parameter ist optional. Ist er nicht Teil eines Requests werden standardmäßig 10 Elemente zurückgegeben. Es können maximal 1000 Elemente angefragt werden.
Ergibt die Überprüfung des \textit{limit}-Parameters das Werte außerhalb des gültigen Wertebereichs von 0 bis 1000 angefragt werden wird der Parameter auf 10 zurückgesetzt. 
Ein Response kann weniger, aber nie mehr Elemente als durch den \textit{limit}-Parameter spezifiziert werden enthalten \ref{appendixlimityaml}. 

\subsubsection{API Landig Page}
Der API Landing Page kann über den URL \textit{http://HOST:PORT/?f=<MEDIA-TYPE>} angefragt werden und liefert als Resource die 
API Landing Page (siehe Anhang \ref{appendixlandngPageyaml}). 
Die einzig zulässige HTTP-Methode für diesen Endpoint ist die HTTP-Get Methode.\\ 
Die API Landing Page kann Eintrittspunkt zu allen anderen Funktionalitäten der Anwendung bezeichnet werden. Sie enthält Verknüpfungen zu den Endpoints, API Landing Page, API Definition, 
Conformance, Process List, Process Description, Job List und Coverage.\\
Die API Landing Page kann in den Media-Types \textit{text/html} (siehe Anhang \ref{appendixlandingPageHTML}) und \textit{application/json} 
(siehe Anhang \ref{appendixlandingPageJSON}) abgerufen werden.\\
 
Wird ein Request für gültig befunden so wird, je nach gewähltem Media-Type, ein passender Response generiert. 
Dieser wird zusammen mit dem \textit{link}- und \textit{resource}-Header versandt (siehe Anhang \ref{appendixLandingPage}). 

!Pseudode!

\subsubsection{API Definition}
Der API Definition Endpoint kann über den URL \textit{http://HOST:PORT/api?f=<MEDIA-TYPE>} angefragt werden und liefert als Ressource die API Definition (siehe Anhang !ref einfügen!).
Auch für diesen Endpoint ist die einzig zulässige HTTP-Methode Get. 
Die API Definition enhält detailierte Informationen zur API. In ihr sind alle verfügbaren Endpoints mit ihren Parametern und Responses aufgeführt. Auch die API Definition kann in den Media-Types
textit{text/html} (siehe Anhang !ref einfügen!) und \textit{application/json} (siehe Anhang !ref einfügen!) abgerufen werden.\\
Nach einem gültigen Request wird ein zum angefrgten Media-Type passen Response generiert. Dieser wird zusammen mit dem \textit{link}- und \textit{resource}-Header versandt (siehe Anhang \ref{appendixAPIDefinition}). 

!Pseudode!
\subsubsection{Conformance Decalration}
Der Endpoint Conformance Declaration kann über den URL \textit{http://HOST:PORT/conformance?f=<MEDIA-TYPE>} angefragt werden und liefert als Ressource die Conformance Declaration (siehe Anhang \ref{appendixconfClassesyaml}).
Get ist ebenfalls die einzige zulässige HTTP-Methode für Requests an diesen Endpoint.

Die Conformance Declaration enthält Verknüpfungen zu allen von der Anwendung implementierten Requiremnets Classes und steht in den Media-Types textit{text/html} (siehe Anhang \ref{appendixconfClassesHTML}) und 
\textit{application/json} (siehe Anhang \ref{appendixconfClassesJSON}) zu Verfügung.

Gültige Requests Lösen die Generierung eines Responses im angefragten Media-Type aus. Der Response wird zusammen mit \textit{link}- und \textit{resource}-Header versandt (siehe Anhang \ref{appendixConformance}).
!Pseudode!

\subsubsection{Process List Endpoint}
Der Process List Endpoint ist über den URL \textit{http://HOST:PORT/processes?f=<MEDIA-TYPE>&limit=<INTEGRER>} angefragt werden. 
Request sind nur mit der HTTP-Methode Get zulässig. Als Ressource erhalten nutzer eine detaillierte Liste der durch die Anwendung agebotenen Prozesse (siehe Anhang \ref{appendixprocessListyaml}). In dieser Listee finden sich die Bezeichnungen, 
Steueroptionen sowie die 
Ein- und Ausgaben der Prozesse. Die Process Liste kann in den Media-Types \textit{text/html} (siehe Anhang \ref{appendixprocessListHTML}) und \textit{application/json} angefragt werden. 

Die Generierung eines Responses im angefrgten Media-Type wird nach einem gültigen Request gestartet. Dazu werden zunächst im \textit{templates/json/processes} Verzeichnis hinterlegten Prozess Beschreibungen geladen und in ein Array geschrieben. 
Dieses kann nun falls der Media-Type \textit{application/json} zu einem Objekt hinzugefügt werden welches zusärtzlich die Verknüfungen zur Ressource enthält. Wurde der Media-Type \textit{text/html} angefragt wird das Array zusammen mit dem HTML-Template dem 
gerendert. 
Der Response wird zusammen mit \textit{link}- und \textit{resource}-Header versandt (siehe Anhang \ref{appendixProcessList}).
!Pseudode!

\subsubsection{Process Description Endpoint}
Unter dem URL \textit{http://HOST:PORT/processes/<processID>?f=<MEDIA-TYPE>} kann der Process Description Endpoint angefragt werden. Welche Prozessdetails zurückgegeben werden hängt vom \textit{processID}-Paramter ab. Alle Prozesse werden über eine 
eindeutige Bezeichnung, die \textit{processID} gekennzeichnet. Sie kann der Prozess Liste entnommen werden. Um eine Prozess Beschreibung anzufragen darf nur die HTTP-Get Methode verwendet werden.   
Als Ressource liefert der Endpoint eine detaillierte Beschreibung des im Request spezifizierten Prozesses (siehe Anhang \ref{appendixoutputDescriptionyaml}). Diese Beschreibung enthält Informationen zu den Steueroptionen sowie den Ein- und Ausgaben des Prozesses.
Wie die Prozess Liste kann die auch die Process Beschreibung im Media-Type \textit{text/html} oder \textit{application/json} angefragt werden. 
Die Generierung eines Responses verläuft ähnlich der der Prozess Liste. Zunächst wird die Prozess Beschreibung des angefragten Prozesses geladen. Soll ein Response mit dem Media-Type \textit{application/json} generiert werden wird diese versandt. 
Für Request mit dem Media-Type \textit{text/html} wird die Prozessbeschreibung zusammen mit einem Template an den Renderer übergegen. 
!Pseudode!
\subsubsection{Prozess Ausführung}
?Wie erreiche ich den Endpoint?
?Welche Ressource liefert der Endpoint?
?Welche HTTP-Methoden sind erlaubt?
?Was ist der Inhalt der Resource?
?Welche Media Types gibt es?
?Wie läuft die generierung des Response ab und was enthält dieser?
!Pseudode!
\subsubsection{Job Status}?Wie erreiche ich den Endpoint?
?Welche Ressource liefert der Endpoint?
?Welche HTTP-Methoden sind erlaubt?
?Was ist der Inhalt der Resource?
?Welche Media Types gibt es?
?Wie läuft die generierung des Response ab und was enthält dieser?
!Pseudode!
\subsubsection{Job Resultate}
?Wie erreiche ich den Endpoint?
?Welche Ressource liefert der Endpoint?
?Welche HTTP-Methoden sind erlaubt?
?Was ist der Inhalt der Resource?
?Welche Media Types gibt es?
?Wie läuft die generierung des Response ab und was enthält dieser?
!Pseudode!
\subsection{Requirements Class OGC Process Description}
\subsection{Requirements Class Job List}
?Wie erreiche ich den Endpoint?
?Welche Ressource liefert der Endpoint?
?Welche HTTP-Methoden sind erlaubt?
?Was ist der Inhalt der Resource?
?Welche Media Types gibt es?
?Wie läuft die generierung des Response ab und was enthält dieser?
!Pseudode!
\subsection{Requirements Class Dismiss}
?Wie erreiche ich den Endpoint?
?Welche Ressource liefert der Endpoint?
?Welche HTTP-Methoden sind erlaubt?
?Was ist der Inhalt der Resource?
?Welche Media Types gibt es?
?Wie läuft die generierung des Response ab und was enthält dieser?
!Pseudode!
\subsection{Requirements Class OpenAPI 3.0}
Der OGC API - Processes - Part 1: Core Standard macht über die eigentlichen Funktionen der API auch Vorgaben zur Art der Dokumentation der API. Hierfür soll der 
OpenAPI 3.0 Standard genutzt werden. 
\subsection{Prozesse}
\subsubsection{Echo}
Da eine standardkonforme API mindestens einen testbaren Prozess anbieten muss ist der Echo-Prozess ebenfalls Teil der prototypischen Implementierung. 
Dieser nimmt einen beliebigen Wert entgegen. Nach einer kurzen, simulierten Bearbeitungszeit kann dieser wieder als Resultat abgefragt werden. 

Nach dem Start eines Echo Prozesses wird zunächst überprüft ob der Job nicht den Status \textit{dismissed} aufweist. Wäre dies der Fall wird die Ausführung abgebrochen. 
Andernfalls wird der \textit{started}-Eintrag in der status.json mit dem aktuellen Zeitstempel versehen und der zurückzugebende Wert aus den Eingaben des Jobs gelesen.
Schlägt dies fehl wird der \textit{status}-Eintrag in der status.json auf \textit{failed} gesetzt und der Ausführung abgebrochen. 
Anschließend wartet das Programm fünf Sekunden. Nach einer erneuten Prüfung des Job-Status wird das Ergebnis als .json in das \textit{results/}-Verzeichnis des Jobs geschrieben.
Diese results.json enthält den Eingabewert und die Nachricht das es sich um ein Echo handelt. 
Als letzter Schritt wird der Job-Status, der Fortschritt, der Infotext sowie der Beendigungszeitpunkt in der Status-Datei des Jobs aktualisiert \ref{appendixEchoProcess}. 
\subsubsection{Überflutungsmonitoring}
\subsection{Zusätzliche Funktionalitäten}
\subsubsection{Coverage} 
Der nicht im Standard definierte Coverage Endpoint kann über den URL \textit{http://HOST:PORT/coverage?f=<MEDIA-TYPE>} erreicht werden. Als Ressource liefert er eine Liste aller Sentinel-1 Datensätze welche persistent gespiechert sind. 
Jobs welche auf diese Datensätze zugreifen können schneller abgearbeitet werden da ein zeitaufwendiges herunterladen der Datensätze entfällt. Die Coverage Ressource kann nur mit der HTTP-Get Methode angefragt werden. 

?Was ist der Inhalt der Resource?
?Welche Media Types gibt es?
?Wie läuft die generierung des Response ab und was enthält dieser?
\newpage
\restoregeometry
\section{Evaluation}
\subsection{Wartbarkeit}



\newpage
\restoregeometry
\section{Diskussion}
%Frage
%Im Rahmen dieser Arbeit sollte untersucht werden, ob sich der OGC API - Processes - Part 1: Core Standard dazu eignet 
%unter Verwendung der Programmiersprache Python APIs für Rich Data Interfaces zu entwickeln, welche das Verarbeiten von Daten des Copernicus Programmes erlauben. 
%Ein weiterer Teil der Untersuchung ist die Betrachtung der Nutzerfreundlichkeit einer solchen Anwendungen. \\
%Dazu wurde eine prototypische Anwendung in der Programmiersprache Python entwickelt. Die prototypische Anwendung ermöglicht es Nutzer*innen 
%über eine OGC API - Processes - Part 1: Core standardkonforme API Überschwemmungsmonitoring auf Basis von Daten der Sentinel-1 Mission 
%zu betreiben. \\

%Umsetzung
%Die API ist dabei weitestgehend mit dem \verb|flask|-Framework realisiert worden. Die API setzt dabei einen Großteil der im 
%OGC API - Processes - Part 1: Core Standard definierten Requirements um.
%Die Kopplung zu den Daten der Sentinel-1 Mission erfolgt mit dem \verb|sentinelsat|-Package. Dabei werden zu den Anfragen passende Datensätze aus
%dem Copernicus Open Access Hub gesucht und heruntergeladen. Alternativ erlaubt die Anwendung auch das persistente Hinterlegen von 
%Sentinel-1 Datensätzen. 
%Das geometrische und radiometrische Kalibrieren der Sentinel-1 Datensätze erfolgt mit dem Python-Wrapper \verb|snappy|. Dieser erlaubt die Nutzung der
%Funktionen der Sentinel-1 Toolbox der SNAP. 
%Das auf Basis der kalibrierten Sentinel-1 Datensätze durchgeführte Überschwemmungsmonitoring erzeugt binäre Überschwemmungsmasken aus dem NDSI.
%Die dazu nötigen bandmathematischen und statistischen Berechnungen werden mit den Python-Packages \verb|skimage| und \verb|osgeo| durchgeführt. 
%Die Untersuchung der Nutzerfreundlichkeit der Anwendung erfolgte auf Basis der von Nielsen vorgeschlagenen Heuristiken. \\

%Bewertung
Die prototypisch implementierte OGC API - Processes - Part 1: Core standardkonforme API konnte zeigen, dass die im Standard formulierten 
Ziele erreicht werden. Die API erlaubt Nutzer*innen mit wenigen, einfach zu bedienenden Endpoints Prozesse, welche Geodaten erzeugen oder 
verarbeiten, zu starten, zu überwachen und deren Ergebnisse abzurufen. Dabei werden die Architekturstile REST und HATEOAS berücksichtigt. 
Die angebotenen Ressourcen stehen in menschen- und maschinenlesbaren Formaten zur Verfügung und enthalten Verknüpfungen zu anderen 
Ressourcen.
Die prototypische Anwendung konnte zeigen, dass sich das \verb|flask|-Framework gut dazu eignet, auf wenige Funktionen beschränkte APIs zu entwickeln. 
Endpoints und die mit ihnen verknüpften Funktionen lassen sich mit geringem Aufwand implementieren. Dabei können die bereitgestellten Ressourcen 
auch dynamische Inhalte enthalten.
Das Copernicus Open Access Hub ist als Datenquelle zu Daten des Copernicus-Programmes geeignet. Das \verb|sentinelsat|-Package ermöglicht einfaches 
Auffinden von geeigneten Datensätzen. Sofern diese sich nicht im Langzeitarchiv befinden, können diese direkt heruntergeladen werden. Die 
Limitierungen des Langzeitarchivs können durch das persistente Speichern von Datensätzen umgangen werden. Die räumliche und 
zeitliche Abdeckung wird dann jedoch durch den zur Verfügung stehenden Speicherplatz beschränkt. 
Die SNAP Plattform und der Python-Wrapper \verb|snappy| ermöglichen eine vollständige Kalibrierung von Sentinel-1 Datensätzen. 
Die nötige Installation und Konfiguration der Plattform wirken sich allerdings 
auch nachteilig auf die Eigenschaften der prototypischen Anwendung aus. So wird zum Beispiel die Python Version eingeschränkt. Auch gestaltet sich 
etwaige Containerisierung aufwendig. Zu bemerken ist jedoch, dass für das Kalibrieren von Sentinel-1 Datensätzen kaum alternative Lösungen oder 
Python-Packages zur Verfügung stehen. Allerdings stellen manche DIAS Plattformen bereits kalibrierte Datensätze zur Verfügung. Die Verwendung  
dieser spart zwar die Installation und Konfiguration von SNAP, nimmt Entwicklern allerdings die Möglichkeit, Einfluss auf die Kalibrierungen zu nehmen.
Das implementierte Verfahren zum Überschwemmungsmonitoring auf Basis der Daten der Sentinel-1 Mission erlaubt ein Detektieren von Überschwemmungen. 
Die Berechnung des NDSI und dessen Binärisierung anhand eines Schwellwertes kann leicht mit den Python-Packages \verb|skimage| und \verb|osgeo| 
implementiert werden. Allerdings schwankt die Qualität der Ergebnisse stark. So liefert das gewählte Schwellwertverfahren die verlässlichsten Schwellwerte, 
wenn der NDSI eine stark bimodale Verteilung aufweist. Ist diese nur schwach oder gar nicht vorhanden, kann es zu einer wenig brauchbaren Binärisierung kommen. 
Die Ausprägung der Bimodalität hängt dabei vom gewählten Raumausschnitt sowie dem Verhältnis von überfluteten zu trockenliegenden Flächen in diesem ab. 
Da die Anwendung die Ausgabe von NDSI und binärer Überschwemmungsmaske erlaubt, können Nutzer*innen sowohl analysebereite als auch interpretationsfähige Daten 
beziehen.  \\

Die Evaluation der Nutzbarkeit der prototypischen Anwendung offenbarte, dass diese für Anwendungen erstrebenswerte Eigenschaften aufweist. Zum einen ist die 
Anwendung detailliert dokumentiert. Zum anderen werden Nutzer*innen bereits in vielfältiger Weise über den Stand seiner gestarteten Prozesse informiert. 
Ein wichtiger Aspekt computergestützter Verfahren ist ihre Reproduzierbarkeit. Das Committee on Reproducibility and Replicability in Science definiert 
ein vorgestelltes computergestütztes Verfahren als reproduzierbar, wenn mit identischen Eingaben und unter gleichen Systembedingungen identische Resultate erzielt werden können.  
Dies bedeutet zunächst, dass die verwendete Software quelloffen und kostenlos zur Verfügung steht, um Interessierte in die Lage zu versetzen, das vorgestellte Verfahren
selbst durchzuführen. Um dies so einfach wie möglich zu machen, sollte die verwendete Software detailliert dokumentiert sein. Ein besonderes Augenmerk sollte dabei auf den 
zugrundeliegenden Daten, den verwendeten Methoden und der ursprünglichen Systemumgebung liegen \cite{reproducibility}.
Die API der prototypischen Implementierung ist durch die Dokumente der OGC und die API-Definition gut und ausführlich dokumentiert. Diese gewährleisten zusätzlich das Nutzer*innen 
die Bedienung der API schnell erlernen können. 
Die zum Überschwemmungsmonitoring verwendeten Methoden in dieser Arbeit und den im Rahmen dieser Arbeit zurate gezogenen Arbeiten ausführlich beschrieben. Sie stehen jedoch nicht 
gesammelter und aufbereiteter Form  zur Verfügung. Die verwendete Systemumgebung kann aus der 
der Anwendung beigefügten \verb|environment.yaml| nachvollzogen werden. \\

%Beschränkungen
Da die prototypisch implementierte API den OGC API - Processes - Part 1: Core Standard nicht vollständig umsetzt, konnten nicht alle beschriebenen 
Funktionen getestet werden. Dazu zählen die Callback Funktionalität, die Ausgabe von Ergebnissen im Response-Typ \verb|reference| und ein 
vollständiger Test-Suit.
Im Rahmen dieser Arbeit wurden nur Daten aus dem Copernicus Open Access Hub bezogen. Zur Struktur und Qualität der von den DIAS Plattformen 
bereitgestellten Daten kann keine Aussage gemacht werden. 
Da die Evaluation der Anwendung auf heuristischer Basis erfolgte, sind die vorgestellten Ergebnisse subjektiv und vom Evaluierenden abhängig. Außerdem 
wurden nur Nutzbarkeitsaspekte betrachtet. Eine Evaluierung von technischen Aspekten ist nicht erfolgt. 
Ein Vergleich mit anderen Implementierungen welche zum Beispiel andere Frameworks verwenden oder in anderen Programmiersprachen verfasst sind ist aus 
Gründen des Umfangs ebenfalls nicht Teil dieser Arbeit. 

%Mit Hinblick auf die Beschränkungen der prototypischen Anwendung scheint es sinnvoll andere Programmiersprachen wie R und Java zu erproben da auch für diese 
%eine Vielzahl von Packages und Bibliotheken zur Verfügung stehen um Webanwendungen und Prozesse zu entwickeln. 
%Darüber hinaus scheint die Erprobung der DIAS Plattformen als Datenquelle für Copernicus Daten sinnvoll.
%Um die Verlässlichkeit der Ergebnisse des Überschwemmungsmonitorings zu verbessern, sollten andere, stabilere Verfahren zur Extraktion von 
%Überschwemmungsmasken gefunden und erprobt werden.  
%Um die Evaluation der Anwendung zu verbessern, sollten in Ergänzung zu Heuristiken auch vergleichbare Metriken gefunden und angewandt werden. 
%Neben der Evaluation der Nutzbarkeit sollten auch technische Aspekte wie Skalierbarkeit, Wartbarkeit und Erweiterbarkeit untersucht werden. 



\restoregeometry
\section{Fazit und Ausblick}
Im Rahmen dieser Arbeit konnte ein funktionsfähiges Rich Data Interface für die Daten des Copernicus-Programmes 
implementiert werden. Teil der Anwendung ist eine 
API, welche diese wesentlichen im OGC API - Processes - Part 1: Core Standard beschriebenen Funktionen anbietet. 
Mithilfe der API kann Überschwemmungsmonitoring auf Basis der Radardaten der Sentinel-1 Mission betrieben werden. 
Die heuristische Untersuchung dieser prototypischen Implementierung konnte bestätigen das sich der OGC API - Processes - Part 1: Core
Standard gut dazu eignet nutzerfreundliche APIs für Abwendungen zu entwickeln, welche Nutzer*innen in die 
Lage versetzten komplexe Verarbeitungen von Daten des Copernicus-Programmes durchzuführen. 
Nutze*innen können dabei sowohl analysebereite als auch interpretationsfähige Daten zur Verfügung gestellt werden.
Ebenso konnte bestätigt werden das sich solche Anwendungen mir der Programmiersprache Python sowie die zur Verfügung stehenden Packages
implementieren lassen. Die Nutzung des \verb|flask|-Frameworks erlaubt schnell und leicht APIs zu implementieren während Packages wie 
\verb|osgeo| und \verb|skimage| auch komplexe Verarbeitungen von Rasterdaten erlauben. Die Daten des Copernicus-Programmes 
können ebenfalls in sinnvoller Weise mit einer Anwendung gekoppelt werden. Die Beschaffung von Datensätzen ist mit dem 
\verb|sentinelsat|-Package gut umsetzbar während der Python-Wrapper \verb|snappy| die vollständige Kalibrierung der Datensätze ermöglicht. 
Die limitierenden Eigenschaften des Copernicus Open Access Hub können zumindest für eine begrenzte Anzahl von Datensätzen umgangen werden.
Das vorgestellte Verfahren zum Detektieren von Überschwemmungen funktioniert unter optimalen Bedingungen hinreichend gut.
Gleichzeitig konnte die vorgestellte Anwendung deutlich Potentiale zur Weiterentwicklung aufzeigen. \\


Um die vorgestellte prototypische Anwendung zu einem Softwareprodukt zu machen, welches produktiv eingesetzt werden 
kann, sollten einige Aspekte ausgebaut und erweitert werden. 
Zunächst sollte die Installation der Anwendung deutlich vereinfacht werden, da das Erzeugen eines passenden Python-Environments und 
die Konfiguration von SNAP sich als umständlich und zeitaufwendig erweisen können. 
Eine mögliche Lösung wäre die Containerisierung der Anwendung mithilfe von Docker \cite{testbed_16}. 

Auch werden bisher nicht alle Aspekte des OGC API - Processes - Part 1: Core Standards umgesetzt. Die Anwendung sollte im nächsten Schritt also 
vervollständigt werden um zum Beispiel auch den Transmission-Mode \verb|reference| zu unterstützen und einen Callback-Mechanismus bereitzustellen. 
Auch Recommendations wie die Unterstützung von HTTPS und CORS sollten erwogen werden.  
Zudem sollte die Anwendung durch das OGC Validierungsverfahren auf ihre Standardkonformität hin überprüft und gegebenenfalls durch die OGC zertifiziert werden. 

%Ein wichtiger Aspekt computergestützter Verfahren ist ihre Reproduzierbarkeit. Das Committee on Reproducibility and Replicability in Science definiert 
%ein vorgestelltes computergestütztes Verfahren als reproduzierbar, wenn mit identischen Eingaben und unter gleichen Systembedingungen identische Resultate erzielt werden können.  
%Dies bedeutet zunächst, dass die verwendete Software quelloffen und kostenlos zur Verfügung steht, um Interessierte in die Lage zu versetzen, das vorgestellte Verfahren
%selbst durchzuführen. Um dies so einfach wie möglich zu machen, sollte die verwendete Software detailliert dokumentiert sein. Ein besonderes Augenmerk sollte dabei auf den 
%zugrundeliegenden Daten, den verwendeten Methoden und der ursprünglichen Systemumgebung liegen \cite{reproducibility}. 
%Die API der prototypischen Implementierung ist durch die Dokumente der OGC und die API-Definition gut und ausführlich dokumentiert. Die zum Überschwemmungsmonitoring 
%verwendeten Methoden in dieser Arbeit und den im Rahmen dieser Arbeit zurate gezogenen Arbeiten ausführlich beschrieben. 
%Diese sollten jedoch gesammelt und gegebenenfalls 
%im Git-Repository verknüpft werden. Zusätzlich sollte ein ausführbares Prozessierunsgbeispiel ergänzt werden mit dem Nutzer*innen die Anwendung erproben können. Die während dieser 
%Arbeit verwendete Systemumgebung ist lediglich über die \verb|environment.yaml| beschrieben. \\
Darüber hinaus sollte die Anwendung sollte aus Gründen der Reproduzierbarkeit um eine zusammengefasste und aufbereitete Methodenbeschreibung sowie ein 
ausführbares Beispiel ergänzt werden.

%Werden keine Daten für einen bestimmten Zeitraum und einen bestimmten Raum in der Anwendung hinterlegt, müssen sie aus dem Copernicus Open Access Hub heruntergeladen werden. 
%Aus diesem können nicht alle Daten der Sentinel-1 Mission direkt heruntergeladen werden, sondern nur jene, welche sich nicht im 
%Langzeitarchiv befinden. Diese Limitierungen können teilweise umgangen werden, wenn die DIAS Plattformen zur Datenbeschaffung genutzt werden. \\
Da die Limitierungen der vorgestellten Kopplung zu den Daten des Copernicus-Programmes möglicherweise durch die Nutzung von DIAS Plattformen umgangen werden können, sollten
diese als alternative Datenquellen erschlossen werden. 

%Das momentan für die Schwellwertermittlung verwendete Verfahren liefert für bimodale Verteilungen belastbare Schwellwerte. 
Um die Schwellwertbildung unabhängiger von den Eigenschaften der Verteilung der Reflexionswerte zu machen, sollten 
andere Verfahren untersucht und gegebenenfalls implementiert werden. Dies können komplexere statistische Verfahren wie der SNDSI, aber auch maschinelle Lernverfahren und 
Modelle sein \cite{flood_proxy_mapping_ndsi,deep_learning_approach}. 
Des Weiteren sollte untersucht werden ob sich die vorgestellte Anwendung in sinnvoller Wiese 
um die Konzepte der CARD4L Initiative des Committee on Earth Observation Satellites (CEOS) erweitert werden kann. 
Dazu zählen unter anderem die Formulierung von Produkt Spezifikationen sowie die Einführung von Qualitätsmerkmalen für die bereitgestellten Daten \cite{testbed_16}. 

Im Rahmen dieser Arbeit wurden nur die Nutzbarkeitsaspekte evaluiert. Zusätzlich sollten jedoch auch Aspekte wie die Wartbarkeit, Skalierbarkeit und Performanz näher 
mit geeigneten Heuristiken und Verfahren untersucht werden. Falls Aspekte mit Metriken bewertet werden können, sollten die heuristischen Evaluationen um diese ergänzt werden. 
Mit Hinblick auf eine mögliche technische Evaluationen scheint es sinnvoll Frameworks wie \verb|django| und andere Programmiersprachen wie R und Java zu erproben da auch für diese 
eine Vielzahl von Packages und Bibliotheken zur Verfügung stehen um Webanwendungen und Prozesse zu entwickeln. 





%\input{Kapitel/Fazit.tex}
\newpage
\thispagestyle{empty}
\begin{thebibliography}{1}

\bibitem{tutorial_on_sar}
A. Moreira, M. Younis, P. Prats-Iraola, G. Krieger, I. Hajnsek und K. P. Papathanassiou (2013, April 17). A Tutorial on Synthetic Aperture Radar [Online]. Verfügbar unter: 
https://www.researchgate.net/publication/257008464\_A\_Tutorial\_on\_Synthetic\_Aperture\_Radar
(Zugriff am: 6. Juni 2022).

\bibitem{einfuehrung_in_fernerkundung}
J. Albertz, Einführung in die Fernerkundung, 4. Auflage Darmstadt: Wissenschaftliche Buchgesellschaft, 2009

\bibitem{history_of_copernicus}
Europäische Kommission (2018, Oktober 06). Copernicus: 20 years of History [Online]. Verfügbar unter: 
https://www.copernicus.eu/en/documentation/information-material/signature-esafrance-collaborative-ground-segment
(Zugriff am: 13. Juni 2022).

\bibitem{sentinel_overview}
European Space Agency (2018). Sentinels - Space for Copernicus [Online]. Verfügbar unter: 
https://www.d-copernicus.de/daten/satelliten/daten-sentinels/
(Zugriff am: 13. Juni 2022).

\bibitem{what_is_copernicus}
Europäische Kommission (2019). What is Copernicus [Online]. Verfügbar unter: 
https://www.copernicus.eu/en/documentation/information-material/brochuresbrochures
(Zugriff am: 13. Juni 2022).

\bibitem{sentinel_1_overview}
ESA Communications (2012, März). Sentinel-1 ESA's Radar Observatory Mission for GMES Operational Services [Online]. Verfügbar unter: 
https://sentinel.esa.int/web/sentinel/missions/sentinel-1/overview
(Zugriff am: 13. Juni 2022).

\end{thebibliography}
\appendix
\newpage
\restoregeometry
\counterwithin{lstlisting}{section}

\section{Schemata}
\renewcommand{\lstlistingname}{Pseudocode}  

\subsection{Ablauf eines Requests an den API Landing Page Endpoint}
\begin{algorithm}[H]
\caption{Ablauf eines Requests an den API Landing Page Endpoint}\label{appendixpsedoLandingPage}
\scriptsize
\begin{algorithmic}     
    \STATE HTTP-Methode $\gets$ Anfrage.HTTP-Methode
    \STATE Encoding $\gets$ Anfrage.f
    \IF{HTTP-Methode != GET}
        \RETURN{HTTP-Statuscode 405}
    \ELSE{}
        \STATE \hskip0.5em \textbf{try}
        \begin{ALC@g}
        \IF{Encoding == text/html \OR Encoding == None}
            \STATE Response $\gets$ render\_template(templates/html/landingPage.html)
            \STATE Link-Header $\gets$ http://HOST:PORT/?f=text/html
            \STATE Resource-Header $\gets$ landingPage
            \RETURN{HTTP-Statuscode 200 \AND Antwort mit Link- und Resource-Header}
        \ELSIF{Encoding == application/json}
            \STATE JSON $\gets$ open(templates/json/landingPage.json)
            \STATE Response $\gets$ jsonify(Datei)
            \STATE Link-Header $\gets$ http://HOST:PORT/?f=application/json
            \STATE Resource-Header $\gets$ landingPage
            \RETURN{HTTP-Statuscode 200 \AND Antwort mit Link- und Resource-Header}
        \ELSE{}
            \RETURN{HTTP-Statuscode 406}
        \ENDIF{}
        \end{ALC@g}
        \STATE \hskip0.5em \textbf{exept}
        \begin{ALC@g}
            \RETURN{HTTP-Statuscode 500}
        \end{ALC@g}
    \ENDIF{}
\end{algorithmic}
\end{algorithm}

\subsection{Ablauf eines Requests an den API Definition Endpoint}
\begin{algorithm}[H]
\caption{Ablauf eines Requests an den API Definition Endpoint}\label{appendixpsedoAPIDefinition}
\scriptsize
\begin{algorithmic}     
    \STATE HTTP-Methode $\gets$ Anfrage.HTTP-Methode
    \STATE Encoding $\gets$ Anfrage.f
    \IF{HTTP-Methode != GET}
        \RETURN{HTTP-Statuscode 405}
    \ELSE{}
        \STATE \hskip0.5em \textbf{try}
        \begin{ALC@g}
        \IF{Encoding == text/html \OR Encoding == None}
            \STATE Response $\gets$ render\_template(templates/html/apiDefinition.html)
            \STATE Link-Header $\gets$ http://HOST:PORT/api?f=text/html
            \STATE Resource-Header $\gets$ apiDefinition
            \RETURN{HTTP-Statuscode 200 \AND Antwort mit Link- und Resource-Header}
        \ELSIF{Encoding == application/json}
            \STATE JSON $\gets$ open(templates/json/apiDefinition.json)
            \STATE Response $\gets$ jsonify(Datei)
            \STATE Link-Header $\gets$ http://HOST:PORT/api?f=application/json
            \STATE Resource-Header $\gets$ apiDefinition
            \RETURN{HTTP-Statuscode 200 \AND Antwort mit Link- und Resource-Header}
        \ELSE{}
            \RETURN{HTTP-Statuscode 406}
        \ENDIF{}
        \end{ALC@g}
        \STATE \hskip0.5em \textbf{exept}
        \begin{ALC@g}
            \RETURN{HTTP-Statuscode 500}
        \end{ALC@g}
    \ENDIF{}
\end{algorithmic}
\end{algorithm}
    
\subsection{Ablauf eines Requests an den Conformance Endpoint}
\begin{algorithm}[H]
\caption{Ablauf eines Requests an den Conformance Endpoint}\label{appendixpsedoConformance}
\scriptsize
\begin{algorithmic}     
    \STATE HTTP-Methode $\gets$ Anfrage.HTTP-Methode
    \STATE Encoding $\gets$ Anfrage.f
    \IF{HTTP-Methode != GET}
        \RETURN{HTTP-Statuscode 405}
    \ELSE{}
        \STATE \hskip0.5em \textbf{try}
        \begin{ALC@g}
        \IF{Encoding == text/html \OR Encoding == None}
            \STATE Response $\gets$ render\_template(templates/html/confClasses.html)
            \STATE Link-Header $\gets$ http://HOST:PORT/conformance?f=text/html
            \STATE Resource-Header $\gets$ conformance
            \RETURN{HTTP-Statuscode 200 \AND Antwort mit Link- und Resource-Header}
        \ELSIF{Encoding == application/json}
            \STATE JSON $\gets$ open(templates/json/confClasses.json)
            \STATE Response $\gets$ jsonify(Datei)
            \STATE Link-Header $\gets$ http://HOST:PORT/conformance?f=application/json
            \STATE Resource-Header $\gets$ conformance
            \RETURN{HTTP-Statuscode 200 \AND Antwort mit Link- und Resource-Header}
        \ELSE{}
            \RETURN{HTTP-Statuscode 406}
        \ENDIF{}
        \end{ALC@g}
        \STATE \hskip0.5em \textbf{exept}
        \begin{ALC@g}
            \RETURN{HTTP-Statuscode 500}
        \end{ALC@g}
    \ENDIF{}
\end{algorithmic}
\end{algorithm}

\subsection{Ablauf eines Requests an den Process List Endpoint}
\begin{algorithm}[H]
\caption{Ablauf eines Requests an den Process List Endpoint}\label{appendixpsedoProcessList}
\scriptsize
\begin{algorithmic}     
    \STATE HTTP-Methode $\gets$ Anfrage.HTTP-Methode
    \STATE Encoding $\gets$ Anfrage.f
    \STATE Limit $\gets$ Anfrage.limit
    \IF{HTTP-Methode != GET}
        \RETURN{HTTP-Statuscode 405}
    \ELSE{}
        \STATE \hskip0.5em \textbf{try}
        \begin{ALC@g}
            \IF{Limit == None \OR Limit <= 0 \OR Limit > 10000}
                \STATE Limit $\gets$ 10
            \ELSE{}
                \STATE Limit $\gets$ Anfrage.limit
            \ENDIF
        \IF{Encoding == text/html \OR Encoding == None}
            \STATE Process List $\gets$ []
            \STATE Processes $\gets$ List of Process descriptions in templates/json/processes
            \FOR{Process \textbf{in} Processes}
                \STATE JSON $\gets$ open(jobs/Job-ID/status.json)
                \STATE Process List \textbf{append} JSON
            \ENDFOR{}
            \STATE Response $\gets$ render\_template(templates/html/processList.html, Process List[0:Limit])
            \STATE Link-Header $\gets$ http://HOST:PORT/processList?f=application/json
            \STATE Resource-Header $\gets$ processList
            \RETURN{HTTP-Statuscode 200 \AND Antwort mit Link- und Resource-Header}
        \ELSIF{Encoding == application/json}
            \STATE Process List $\gets$ []
            \STATE Processes $\gets$ List of Process descriptions in templates/json/processes
            \FOR{Process \textbf{in} Processes}
                \STATE JSON $\gets$ open(jobs/Job-ID/status.json)
                \STATE Process Liste \textbf{append} JSON
            \ENDFOR{}
            \STATE \textbf{add} Links \textbf{to} Process Liste
            \STATE Response $\gets$ jsonify(Process Liste[0:Limit])
            \STATE Link-Header $\gets$ http://HOST:PORT/processList?f=application/json
            \STATE Resource-Header $\gets$ processList
            \RETURN{HTTP-Statuscode 200 \AND Antwort mit Link- und Resource-Header}
        \ELSE{}
            \RETURN{HTTP-Statuscode 406}
        \ENDIF{}
        \end{ALC@g}
        \STATE \hskip0.5em \textbf{exept}
        \begin{ALC@g}
            \RETURN{HTTP-Statuscode 500}
        \end{ALC@g}
    \ENDIF{}
\end{algorithmic}
\end{algorithm}

\subsection{Ablauf eines Requests an den Process Description Endpoint}
\begin{algorithm}[H]
\caption{Ablauf eines Requests an den Process Description Endpoint}\label{appendixpsedoProcessDescription}
\scriptsize
\begin{algorithmic}     
    \STATE HTTP-Methode $\gets$ Anfrage.HTTP-Methode
    \STATE Encoding $\gets$ Anfrage.f
    \STATE Process-ID $\gets$ Anfrage.processID
    \IF{HTTP-Methode != GET}
        \RETURN{HTTP-Statuscode 405}
    \ELSE{}
        \STATE \hskip0.5em \textbf{try}
        \begin{ALC@g}
            \IF{Encoding == text/html \OR Encoding == None}
                \IF{templates/json/processes/Process-ID.json exists}
                    \STATE JSON $\gets$ open(json/processes/Process-ID.json)
                    \STATE Response $\gets$ render\_template(templates/html/process.html, JSON)
                    \STATE Link-Header $\gets$ http://HOST:PORT/processes/<processID>?f=text/html
                    \STATE Resource-Header $\gets$ Process-ID
                    \RETURN{HTTP-Statuscode 200 \AND Antwort mit Link- und Resource-Header}
                \ELSE{}
                    \STATE Exception $\gets$ No such process exception
                    \STATE Resource-Header $\gets$ no-such-process
                    \RETURN{HTTP-Statuscode 404 \AND Exception mit Resource-Header}
                \ENDIF{}
            \ELSIF{Encoding == application/json}
                \IF{templates/json/processes/Process-ID.json exists}
                    \STATE JSON $\gets$ open(json/processes/Process-ID.json)
                    \STATE Response $\gets$ jsonify(JSON)
                    \STATE Link-Header $\gets$ http://HOST:PORT/processes/<processID>?f=application/json
                    \STATE Resource-Header $\gets$ Process-ID
                    \RETURN{HTTP-Statuscode 200 \AND Antwort mit Link- und Resource-Header}
                \ELSE{}
                    \STATE Exception $\gets$ No such process exception
                    \STATE Resource-Header $\gets$ no-such-process
                    \RETURN{HTTP-Statuscode 404 \AND Exception mit Resource-Header}
                \ENDIF{}
        \ELSE{}
            \RETURN{HTTP-Statuscode 406}
        \ENDIF{}
        \end{ALC@g}
        \STATE \hskip0.5em \textbf{exept}
        \begin{ALC@g}
            \RETURN{HTTP-Statuscode 500}
        \end{ALC@g}
    \ENDIF{}
\end{algorithmic}
\end{algorithm}

\subsection{Ablauf eines Requests an den Job Status Endpoint}
\begin{algorithm}[H]
\scriptsize
\caption{Ablauf eines Requests an den Job Status Endpoint}\label{appendixpsedoJobStatus}
\begin{algorithmic}     
    \STATE HTTP-Methode $\gets$ Anfrage.HTTP-Methode
    \STATE Encoding $\gets$ Anfrage.f
    \STATE Job-ID $\gets$ Anfrage.jobID
    \IF{HTTP-Methode == GET}
        \STATE Test
    \ELSIF{HTTP-Methode == DELETE}
        \STATE Test
    \ELSE{}
        \RETURN{HTTP-Statuscode 405}
    \ENDIF{}
\end{algorithmic}
\end{algorithm}

\renewcommand{\lstlistingname}{Quellcode}
\section{Quellcodeverzeichnis}
\subsection{Konfiguration von Werkzeug auf HTTP 1.1}
\begin{lstlisting}[caption={Konfiguration von Werkzeug auf HTTP 1.1}, style = Python]
    from flask import Flask
    from werkzeug.serving import WSGIRequestHandler
    from werkzeug.serving import BaseWSGIServer
    WSGIRequestHandler.protocol_version = "HTTP/1.1"
    BaseWSGIServer.protocol_version = "HTTP/1.1"
\end{lstlisting}\label{appendixconfWerkzeug}

\subsection{Quellcode Landing Page Endpoint}
\begin{lstlisting}[caption={Landing Page Endpoint}, style = Python]
#landingpage endpoint
@app.route('/',  methods = ['GET'])
def getLandingPage():
    app.logger.info('/') 
    try:
        if(request.content_type == "text/html" or
        request.args.get('f')=="text/html" or 
        request.args.get('f') == None):
                response = render_template('html/landingPage.html') 
                return response, 200, {
                "link": "localhost:5000/?f=text/html", 
                "resource": "landingPage"
                } 
        elif(request.content_type == "application/json" or
        request.args.get('f')=="application/json"): 
            file = open('templates/json/landingPage.json',) 
            payload = json.load(file) 
            file.close() 
            response = jsonify(payload) 
            return response, 200, {
                "link": "localhost:5000/?f=application/json", 
                "resource": "landingPage"} 
        else:
            return "HTTP status code 406: not acceptable", 406 
        except:
            return "HTTP status code 500: internal server error", 500 
\end{lstlisting}\label{appendixLandingPage}

\newpage
\subsection{Quellcode API Definition Endpoint}
\begin{lstlisting}[caption={API Definition Endpoint}, style = Python]
#api endpoint
@app.route('/api',  methods = ['GET']) 
def getAPIDefinition():
    app.logger.info('/api') 
    try:
        if(request.content_type == "text/html" or
        request.args.get('f')=="text/html" or 
        request.args.get('f') == None): 
            response = render_template('html/apiDefinition.html') 
            return response, 200, {
                "link": "localhost:5000/apiDefinition?f=text/html", 
                "resource": "apiDefinition"} 
        elif(request.content_type == "application/json" or 
        request.args.get('f')=="application/json"): 
            file = open('templates/json/apiDefinition.json',) 
            payload = json.load(file) 
            file.close() #close apiDefinition.json
            response = jsonify(payload) 
            return response, 200, {
                "link": "localhost:5000/api?f=application/json", 
                "resource": "apiDefinition"} 
        else:
            return "HTTP status code 406: not acceptable", 406 
    except:
        return "HTTP status code 500: internal server error", 500
\end{lstlisting}\label{appendixAPIDefinition}

\newpage
\subsection{Quellcode Conformance Endpoint}
\begin{lstlisting}[caption={Conformance Endpoint}, style = Python]
#conformance endpoint
@app.route('/conformance',  methods = ['GET'])
def getConformance():
    app.logger.info('/conformance') 
    try:
        if(request.content_type == "text/html" or 
        request.args.get('f')=="text/html" or 
        request.args.get('f') == None): 
            response = render_template('html/confClasses.html') 
            return response, 200, {
                "link": "localhost:5000/conformance?f=text/html", 
                "resource": "conformance"}
        elif(request.content_type == "application/json" or 
        request.args.get('f')=="application/json"): 
            file = open('templates/json/confClasses.json',)
            payload = json.load(file) 
            file.close() 
            response = jsonify(payload) 
            return response, 200, {
                "link": "localhost:5000/conformance?f=application/json",
                "resource": "conformance"} 
        else:
            return "HTTP status code 406: not acceptable", 406 
    except:
        return "HTTP status code 500: internal server error", 500
\end{lstlisting}\label{appendixConformance}   

\newpage
\subsection{Quellcode Process List Endpoint}
\begin{lstlisting}[caption={Process List Endpoint}, style = Python]
#processes endpoint
@app.route('/processes', methods = ['GET']) 
def getProcesses():
    app.logger.info('/processes') 
    if(request.args.get('limit') == None or 
        int(request.args.get('limit')) <= 0 or 
        int(request.args.get('limit')) > 1000): 
        limit = 10 #set limit to default value
    else:
        limit = int(request.args.get('limit'))
    try:
        if(request.content_type == "text/html" or
            request.args.get('f')=="text/html" or 
            request.args.get('f') == None): 
                processList = [] #initialize list of processes
                processDescriptions = os.listdir("templates/json/processes") 
                counter = 0
                for i in processDescriptions:
                    file = open('templates/json/processes/' + i,) 
                    process = json.load(file) 
                    file.close() 
                    processList.append(process) 
                    counter += 1 
                    if(counter == limit): 
                        break 
                response = render_template('html/processes.html', 
                    processes=processList) 
                return response, 200, {
                    "link": "localhost:5000/processes?f=text/html", 
                    "resource": "processes"} 
        elif(request.content_type == "application/json" or 
                request.args.get('f')=="application/json"): 
            processList = [] #initialize list of processes
            processDescriptions = os.listdir("templates/json/processes")     
            for i in processDescriptions: 
                file = open('templates/json/processes/' + i,) 
                process = json.load(file) 
                file.close() 
                processList.append(process) 
            processes = {"processes": processList[0:limit],
                        "links": [ #add links to self and alternate
                            {
                            "href": "localhost:5000/processes?f=applicattion/json",
                            "rel": "self",
                            "type": "application/json"
                            },
                            {
                            "href": "localhost:5000/processes?f=text/html",
                            "rel": "alternate",
                            "type": "text/html"
                            }
                        ]}
            response = jsonify(processes) 
            return response, 200, {
                "link": "localhost:5000/processes?f=application/json", 
                "resource": "processes"} 
        else:
            return "HTTP status code 406: not acceptable", 406 
    except:
        return "HTTP status code 500: internal server error", 500 
\end{lstlisting}\label{appendixProcessList}   
\newpage
\subsection{Quellcode Process Description Endpoint}
\begin{lstlisting}[caption={Process Description Endpoint}, style = Python]
#process endpoint
@app.route('/processes/<processID>', methods = ['GET']) 
def getProcess(processID):
    app.logger.info('/processes/' + processID) 
    try:
        if(request.content_type == "text/html" or 
            request.args.get('f')=="text/html" or 
            request.args.get('f') == None): 
            if(os.path.exists('templates/json/processes/' 
            + str(processID) + 'ProcessDescription.json')): 
                file = open('templates/json/processes/' 
                + str(processID) 
                + 'ProcessDescription.json',) 
                process = json.load(file) 
                file.close() 
                response = render_template("html/Process.html", process=process) 
                return response, 200, {"link": "localhost:5000/processes/" 
                + str(processID) 
                + "?f=text/html", 
                "resource": str(processID)} 
            else:
                exception = render_template('html/exception.html', 
                title="No such process exception", 
                description="Requested process could not be found", 
                type="no-such-process")
                return exception, 404, {"resource": "no-such-process"}
        elif(request.content_type == "application/json" or
                request.args.get('f')=="application/json"): 
            if(os.path.exists('templates/json/processes/' 
            + str(processID) 
            + 'ProcessDescription.json')): 
                file = open('templates/json/processes/' 
                + str(processID) 
                + 'ProcessDescription.json',)
                payload = json.load(file) 
                file.close() 
                response = jsonify(payload) 
                return response, 200, {"link": "localhost:5000/processes/" 
                + str(processID) 
                + "?f=application/json", 
                "resource": str(processID)} 
            else:
                exception = {"title": "No such process exception", 
                "description": "Requested process could not be found", 
                "type": "no-such-process"}
                return exception, 404, {"resource": "no-such-process"} 
        else:
            return "HTTP status code 406: not acceptable", 406
    except:
        return "HTTP status code 500: internal server error", 500 
\end{lstlisting}\label{appendixProcessDescription}      

\newpage
\subsection{Quellcode Process Execution Endpoint}
\begin{lstlisting}[caption={Process Execution}, style = Python]
@app.route('/processes/<processID>/execution', methods = ['POST']) 
def executeProcess(processID):
    app.logger.info('/processes/' + processID + '/execution') 
    try:
        if(os.path.exists('templates/json/processes/' 
        + str(processID) 
        + 'ProcessDescription.json')): 
            data = json.loads(request.data.decode('utf8').replace("'", '"')) 
            inputParameters = utils.parseInput(processID, data)
            if(inputParameters == False):
                return "HTTP status code 400: bad request", 400 
            jobID = str(uuid.uuid4()) 
            created = str(datetime.datetime.now().strftime("%Y-%m-%d %H:%M:%S"))
            os.mkdir("jobs/" + jobID) 
            os.mkdir("jobs/" + jobID + "/results/") 

            job_file = {"jobID": str(jobID), 
                        "processID": str(processID), 
                        "input": inputParameters[0], 
                        "output": inputParameters[2],
                        "responseType": inputParameters[1], 
                        "path": "jobs/" + jobID, 
                        "results": "jobs/" + jobID + "/results/", 
                        "downloads": "jobs/" + jobID + "/downloads/"} 
            json.dumps(job_file, indent=4) 
            with open("jobs/" + jobID + "/job.json", 'w') as f: 
                json.dump(job_file, f) 
            f.close()

            status_file = {"jobID": str(jobID), 
                            "processID": str(processID), 
                            "status": "accepted", 
                            "message": "Step 0/1", 
                            "type": "process", 
                            "progress": 0, 
                            "created": created, 
                            "started": "none", 
                            "finished": "none", 
                            "links": [ 
                                {
                                "href": "localhost:5000/jobs/" 
                                + jobID + "?f=application/json",
                                "rel": "self",
                                "type": "application/json",
                                "title": "this document as JSON"},
                                {
                                "href": "localhost:5000/jobs/" 
                                + jobID + "?f=text/html",
                                "rel": "alternate",
                                "type": "text/html",
                                "title": "this document as HTML"
                                }
                            ]}
            json.dumps(status_file, indent=4) #dump content
            with open("jobs/" + jobID + "/status.json", 'w') as f: #create file
                json.dump(status_file, f) #write content
            f.close() #close file

            response = jsonify(status_file) #create response
            return response, 201, {"location": "localhost:5000/jobs/" 
            + jobID + "?f=application/json", 
            "resource": "job"} 
            exception = {"title": "No such process exception", 
            "description": "Requested process could not be found", 
            "type": "no-such-process"}
            return exception, 404 
    except:
        return "HTTP status code 500: internal server error", 500
\end{lstlisting}\label{appendixProcessExecution}   

\newpage
\subsection{Quellcode Job Endpoint}
\begin{lstlisting}[caption={Job Endpoint}, style = Python]
#job endpoint for status and dismiss
@app.route('/jobs/<jobID>', methods = ['GET', 'DELETE']) 
def getJob(jobID):
    if(request.method == 'GET'):
        app.logger.info('[GET] /jobs/' + jobID)
        try:
            if(request.content_type == "application/json" or 
                request.args.get('f')=="application/json"): 
                if(os.path.exists('jobs/' + str(jobID) + '/status.json')):
                    file = open('jobs/' + str(jobID) + '/status.json')
                    data = json.load(file) 
                    file.close() 
                    response = jsonify(data) 
                    return  response, 200, {"link": "localhost:5000/jobs/" 
                    + str(jobID) 
                    + "?f=application/json",
                     "resource": "job - " 
                     + str(jobID)} 
                else:
                    exception = {"title": "No such job exception", 
                    "description": "No job with the requested jobID could be found", 
                    "type": "no-such-job"}
                    return exception, 404, {"resource": "no-such-job"} 
            elif(request.content_type == "text/html" or
                    request.args.get('f')=="text/html" or 
                    request.args.get('f') == None): 
                if(os.path.exists('jobs/' + str(jobID) + '/status.json')):
                    file = open('jobs/' + str(jobID) + '/status.json') 
                    job = json.load(file) #create response   
                    file.close() #close status.json
                    response = render_template("html/Job.html", job=job)
                    return response, 200, {"link": "localhost:5000/jobs/" + str(jobID) 
                    + "?f=text/html", 
                    "resource": "job - " 
                    + str(jobID)} 
                else:
                    exception = render_template('html/exception.html', title="No such job exception", description="No job with the requested jobID could be found", type="no-such-job")
                    return exception, 404, {"resource": "no-such-job"} #return not found if requested job is not found
            else:
                return "HTTP status code 406: not acceptable", 406 #return not acceptable if requested content-type is not supported
        except:
            return "HTTP status code 500: internal server error", 500 #return internal server error if something went wrong
        
    if(request.method == 'DELETE'):
        app.logger.info('[DELETE] /jobs/' + jobID) #add log entry when endpoint is called
        try:       
            if(os.path.exists('jobs/' + str(jobID) + '/status.json')): #check if jobID exists
                with open('jobs/' + str(jobID) + '/status.json', "r") as f: #open status.json
                    file = json.load(f) #load data from status.json
                    if(file["status"] != "dismissed"): #if job is not dismissed
                        file["status"] = "dismissed" #set status to dismissed
                        file["message"] = "job dismissed" #set emssage to dismissed
                        f.close() #close status.json
                        with open('jobs/' + str(jobID) + '/status.json', "w") as f: #write status.json
                            json.dump(file, f) #dump content
                            f.close() #close status.json                    
                        file = open('jobs/' + str(jobID) + '/status.json') #open status.json
                        data = json.load(file) #load data from status.json 
                        file.close() #close status.json
                        
                        if(request.content_type == "text/html" or #check requested content-type from request body
                            request.args.get('f')=="text/html" or 
                            request.args.get('f') == None):
                            file = open('jobs/' + str(jobID) + '/status.json') #open status.json
                            job = json.load(file) #create response   
                            file.close() #close status.json
                            response = render_template("html/Job.html", job=job) #render dynamic job
                            return response, 200, {"link": "localhost:5000/jobs/" + str(jobID) + "?f=text/html", "resource": "job-dismissed"} #return response and ok
                        else:
                            response = jsonify(data) #create response
                            return response, 200, {"link": "localhost:5000/jobs/" + str(jobID) + "?f=application/json", "resource": "job-dismissed"} #return response and ok with link und resource header
                    elif(request.content_type == "application/json" or #check requested content-type from request body
                            request.args.get('f')=="application/json"): #check requested content-type from inline request):
                        data = json.load(file) #load data from status.json 
                        file.close() #close status.json
                        response = jsonify(data) #create response
                        return response, 410, {"link": "localhost:5000/jobs/" + str(jobID) + "?f=application/json", "resource": "job-dismissed"} #return gone when job was already dismissed
            else:
                exception = {"title": "No such job exception", "description": "No job with the requested processID could be found", "type": "no-such-job"}
                return exception, 404 #return not found if requested job is not found 
        except:            
            return "HTTP status code 500: internal server error", 500 #return internal server error if something went wrong
\end{lstlisting}\label{appendixJob}  

\newpage
\subsection{Quellcode Echo Process}
\begin{lstlisting}[caption={Echo Process}, style = Python]
def echoProcess(job):
    if(checkForDismissal(job.path + '/status.json') == True):
        return
    
    setStarted(job.path + '/status.json')
    
    try:
        input = job.input[0]
        time.sleep(5)
    except:
        updateStatus(job.path + '/status.json', "failed", "The job has failed", "-")
        return
    
    if(checkForDismissal(job.path + '/status.json') == True):
        return
    
    result ={"result": input,
             "message": "This is an echo"}
    json.dumps(result, indent=4)
    with open(job.results + "result.json", 'w') as f: #create file
        json.dump(result, f) #write content
        f.close() #close file
    updateStatus(job.path + '/status.json', "successful", "Step 1 of 1 completed", "100")
    setFinished(job.path + '/status.json')
\end{lstlisting}\label{appendixEchoProcess}  

\section{Schemata}
\renewcommand{\lstlistingname}{Schema}  
\subsection{landingPage.yaml}
\begin{lstlisting}[caption={landingPage.yaml}, style = JSON]
type: object
required:
  - links
properties:
  title:
    type: string
    example: Example processing server
  description:
    type: string
    example: Example server 
  links:
    type: array
    items:
      $ref: "link.yaml"
\end{lstlisting}\label{appendixlandngPageyaml}  

\subsection{confClasses.yaml}
\begin{lstlisting}[caption={confClasses.yaml}, style = JSON]
type: object
required:
    - conformsTo
properties:
    conformsTo:
    type: array
    items:
        type: string
        example: "http://www.opengis.net/spec/ogcapi-processes-1/1.0/conf/core"
\end{lstlisting}\label{appendixconfClassesyaml}  

\subsection{processList.yaml}
\begin{lstlisting}[caption={processList.yaml}, style = JSON]
type: object
required:
  - processes
  - links
properties:
  processes:
    type: array
    items:
      $ref: "processSummary.yaml"
  links:
    type: array
    items:
      $ref: "link.yaml"
\end{lstlisting}\label{appendixprocessListyaml}  

\subsection{limit.yaml}
\begin{lstlisting}[caption={limit.yaml}, style = JSON]
name: limit
in: query
required: false
schema:
  type: integer
  minimum: 1
  maximum: 10000
  default: 10
style: form
explode: false
\end{lstlisting}\label{appendixlimityaml}  

\subsection{link.yaml}
\begin{lstlisting}[caption={link.yaml}, style = JSON]
type: object
required:
  - href
properties:
  href:
    type: string
  rel:
    type: string
    example: service
  type:
    type: string
    example: application/json
  hreflang:
    type: string
    example: en
  title:
    type: string
\end{lstlisting}\label{appendixlinkyaml}  

\subsection{processList.yaml}
\begin{lstlisting}[caption={processList.yaml}, style = JSON]
type: object
required:
    - processes
    - links
properties:
    processes:
    type: array
    items:
        $ref: "processSummary.yaml"
    links:
    type: array
    items:
        $ref: "link.yaml"
\end{lstlisting}\label{appendixprocessListyaml}  

\subsection{processSummary.yaml}
\begin{lstlisting}[caption={processSummary.yaml}, style = JSON]
allOf:
    - $ref: "descriptionType.yaml"
    - type: object
      required:
        - id
        - version
      properties:
        id:
          type: string
        version:
          type: string
        jobControlOptions:
          type: array
          items:
            $ref: "jobControlOptions.yaml"
        outputTransmission:
          type: array
          items:
            $ref: "transmissionMode.yaml"
        links:
          type: array
          items:
            $ref: "link.yaml"
\end{lstlisting}\label{appendixprocessSummaryyaml} 

\subsection{jobControlOptions.yaml}
\begin{lstlisting}[caption={jobControlOptions.yaml}, style = JSON]
type: string
enum:
    - sync-execute
    - async-execute
    - dismiss
\end{lstlisting}\label{appendixjobControlOptionsyaml} 

\subsection{transmissionMode.yaml}
\begin{lstlisting}[caption={transmissionMode.yaml}, style = JSON]
type: string
enum:
    - value
    - reference
default:
    - value
\end{lstlisting}\label{appendixtransmissionModeyaml} 

\subsection{process.yaml}
\begin{lstlisting}[caption={process.yaml}, style = JSON]
allOf:
  - $ref: "processSummary.yaml"
  - type: object
    properties:
      inputs:
        additionalProperties:
          $ref: "inputDescription.yaml"
      outputs:
        additionalProperties:
          $ref: "outputDescription.yaml"
\end{lstlisting}\label{appendixprocessyaml} 

\subsection{inputDescription.yaml}
\begin{lstlisting}[caption={inputDescription.yaml}, style = JSON]
allOf:
- $ref: "descriptionType.yaml"
- type: object
    required:
    - schema
    properties:
    minOccurs:
        type: integer
        default: 1
    maxOccurs:
        oneOf:
        - type: integer
            default: 1
        - type: string
            enum:
            - "unbounded"
    schema:
        $ref: "schema.yaml"
\end{lstlisting}\label{appendixinputDescriptionyaml} 

\subsection{outputDescription.yaml}
\begin{lstlisting}[caption={outputDescription.yaml}, style = JSON]
allOf:
- $ref: "descriptionType.yaml"
- type: object
    required:
    - schema
    properties:
    schema:
        $ref: "schema.yaml"
\end{lstlisting}\label{appendixoutputDescriptionyaml} 

\subsection{description.yaml}
\begin{lstlisting}[caption={description.yaml}, style = JSON]
type: object
properties:
    title:
    type: string
    description:
    type: string
    keywords:
    type: array
    items:
        type: string
    metadata:
    type: array
    items:
        $ref: "metadata.yaml"
    additionalParameters:
    allOf:
        - $ref: "metadata.yaml"
        - type: object
        properties:
            parameters:
            type: array
            items:
                $ref: "additionalParameter.yaml"
\end{lstlisting}\label{appendixdescriptionyaml} 

\subsection{jobList.yaml}
\begin{lstlisting}[caption={jobList.yaml}, style = JSON]
type: object
required:
    - jobs
    - links
properties:
    jobs:
    type: array
    items:
        $ref: "statusInfo.yaml"
    links:
    type: array
    items:
        $ref: "link.yaml"
\end{lstlisting}\label{appendixjobListyaml} 

\subsection{statusInfo.yaml}
\begin{lstlisting}[caption={statusInfo.yaml}, style = JSON]
type: object
required:
    - jobID
    - status
    - type
properties:
    processID:
        type: string
    type:
        type: string
        enum:
        - process
    jobID:
        type: string
    status:
        $ref: "statusCode.yaml"
    message:
        type: string
    created:
        type: string
        format: date-time
    started:
        type: string
        format: date-time
    finished:
        type: string
        format: date-time
    updated:
        type: string
        format: date-time
    progress:
        type: integer
        minimum: 0
        maximum: 100
    links:
        type: array
        items:
            $ref: "link.yaml"
\end{lstlisting}\label{appendixstatusInfoyaml}

\subsection{statusCode.yaml}
\begin{lstlisting}[caption={statusCode.yaml}, style = JSON]
type: string
nullable: false
enum:
    - accepted
    - running
    - successful
    - failed
    - dismissed
\end{lstlisting}\label{appendixstatusCodeyaml}

\section{Ressourcen}
\renewcommand{\lstlistingname}{Ressource}
\subsection{landingPage.html}
\begin{lstlisting}[caption={landingPage.html}, style = HTML]
<!DOCTYPE html>
<html>
	<body>
	<h1>links:</h1>
		<p>
			href:<a href="localhost:5000/?f=text/html">
            localhost:5000/?f=text/html</a><br>
			rel: self<br>
			type: text/html<br>
			title: This document as HTML
		</p>
		<p>
			href:<a href="localhost:5000/?f=application/json">
            localhost:5000/?f=application/json</a><br>
			rel: alternate<br>
			type: application/json<br>
			title: This document as JSON 
		</p>
		<p>
			href:<a href="localhost:5000/api?f=application/json">
            localhost:5000/apiDefinition?f=application/json</a><br>
			rel: service-desc<br>
			type: application/json<br>
			title: API definition for this endpoint as JSON
		</p>
		<p>
			href:<a href="localhost:5000/api?f=text/html">
            localhost:5000/apiDefinition?f=text/html</a><br>
			rel: service-desc<br>
			type: text/html<br>
			title: API definition for this endpoint as HTML
		</p>
		<p>
			href:<a href="localhost:5000/conformance?f=application/json">
            localhost:5000/conformance?f=application/json</a><br>
			rel: conformance<br>
			type: application/json<br>
			title: OGC API - Processes conformance classes implemented by this server as JSON
		</p>
		<p>
			href:<a href="localhost:5000/conformance?f=text/html">
            localhost:5000/conformance?f=text/html</a><br>
			rel: conformance<br>
			type: text/html<br>
			title: OGC API - Processes conformance classes implemented by this server as HTML
		</p>
		<p>
			href:<a href="localhost:5000/processes?f=application/json">
            localhost:5000/processes?f=application/json</a><br>
			rel: processes<br>
			type: application/json<br>
			title: Metadata about the processes as JSON
		</p>
		<p>
			href:<a href="localhost:5000/processes?f=text/html">
            localhost:5000/processes?f=text/html</a><br>
			rel: processes<br>
			type: text/html,<br>
			title: Metadata about the processes as HTML
		</p>
		<p>
			href:<a href="localhost:5000/jobs?f=application/json">
            localhost:5000/jobs?f=application/json</a><br>
			rel: jobs<br>
			type: application/json<br>
			title: The endpoint for job monitoring as JSON
		</p>
		<p>
			href:<a href="localhost:5000/jobs?f=text/html">
            localhost:5000/jobs?f=text/html</a><br>
			rel: jobs<br>
			type: text/html<br>
			title: The endpoint for job monitoring as HTML
		</p>
		<p>
			href:<a href="localhost:5000/coverage?f=application/json">
            localhost:5000/coverage?f=application/json</a><br>
			rel: coverage<br>
			type: application/json<br>
			title: The endpoint for coverage as JSON
		</p>
		<p>
			href:<a href="localhost:5000/coverage?f=text/html">
            localhost:5000/coverage?f=text/html</a><br>
			rel: coverage<br>
			type: text/html<br>
			title: The endpoint for coverage as HTML
		</p>
	</body>
</html>
\end{lstlisting}\label{appendixlandingPageHTML}  

\subsection{landingPage.json}
\begin{lstlisting}[caption={landingPage.json}, style = JSON]
{
    "links": [
    {
        "href": "localhost:5000/?f=application/json",
        "rel": "self",
        "type": "application/json",
        "title": "This document"
    },{
        "href": "localhost:5000/?f=text/html",
        "rel": "alternate",
        "type": "text/html",
        "title": "This document as HTML"
    },
    {
        "href": "localhost:5000/api?f=application/json",
        "rel": "service-desc",
        "type": "application/json",
        "title": "API definition for this endpoint as JSON"
    },
    {
        "href": "localhost:5000/api?f=text/html",
        "rel": "service-desc",
        "type": "text/html",
        "title": "API definition for this endpoint as HTML"
    },
    {
        "href": "localhost:5000/conformance?f=application/json",
        "rel": "conformance",
        "type": "application/json",
        "title": "OGC API - Processes conformance classes implemented by this server as JSON"
    },
    {
        "href": "localhost:5000/conformance?f=text/html",
        "rel": "conformance",
        "type": "text/html",
        "title": "OGC API - Processes conformance classes implemented by this server as HTML"
    },
    {
        "href": "localhost:5000/processes?f=application/json",
        "rel": "processes",
        "type": "application/json",
        "title": "Metadata about the processes as JSON"
    },
        {
        "href": "localhost:5000/processes?f=text/html",
        "rel": "processes",
        "type": "text/html",
        "title": "Metadata about the processes as HTML"
    },
    {
        "href": "localhost:5000/jobs?f=application/json",
        "rel": "jobs",
        "type": "application/json",
        "title": "The endpoint for job monitoring as JSON"
    },
        {
        "href": "localhost:5000/jobs?f=text/html",
        "rel": "jobs",
        "type": "text/html",
        "title": "The endpoint for job monitoring as HTML"
    },
    {
        "href": "localhost:5000/coverage?f=application/json",
        "rel": "coverage",
        "type": "application/json",
        "title": "The endpoint for coverage as JSON"
    },
        {
        "href": "localhost:5000/coverage?f=text/html",
        "rel": "coverage",
        "type": "text/html",
        "title": "The endpoint for coverage as HTML"
    }
    ]
}
\end{lstlisting}\label{appendixlandingPageJSON}  

\subsection{confClasses.html}
\begin{lstlisting}[caption={confClasses.html}, style = HTML]
<!DOCTYPE html>
<html>
    <body>
        <h1>conforms to:</h1>
            <p><a href="https://docs.ogc.org/is/18-062r2/18-062r2.html#toc21">
            https://docs.ogc.org/is/18-062r2/18-062r2.html#toc21</a></p>
            <p><a href="https://docs.ogc.org/is/18-062r2/18-062r2.html#toc40">
            https://docs.ogc.org/is/18-062r2/18-062r2.html#toc40</a></p>
            <p><a href="https://docs.ogc.org/is/18-062r2/18-062r2.html#toc41">
            https://docs.ogc.org/is/18-062r2/18-062r2.html#toc41</a></p>
            <p><a href="https://docs.ogc.org/is/18-062r2/18-062r2.html#toc42">
            https://docs.ogc.org/is/18-062r2/18-062r2.html#toc42</a></p>
			<p><a href="https://docs.ogc.org/is/18-062r2/18-062r2.html#toc47">
            https://docs.ogc.org/is/18-062r2/18-062r2.html#toc47</a></p>
			<p><a href="https://docs.ogc.org/is/18-062r2/18-062r2.html#toc53">
            https://docs.ogc.org/is/18-062r2/18-062r2.html#toc53</a></p>
            <p>
                <b>links:</b><br><br>
                <b>href:</b> <a href="localhost:5000/conformance?f=text/html">
                localhost:5000/conformance?f=text/html</a><br>
                <b>rel:</b> self<br>
                <b>type:</b> text/html<br>
                <b>title:</b> This Document as HTML<br>
                <br>
                <b>href:</b> <a href="localhost:5000/conformance?f=application/json">
                localhost:5000/conformance?f=application/json</a><br>
                <b>rel:</b> alternate<br>
                <b>type:</b> application/json<br>
                <b>title:</b> This document as JSON<br>
            </p>
    </body>
</html>
\end{lstlisting}\label{appendixconfClassesHTML}  

\subsection{confClasses.json}
\begin{lstlisting}[caption={confClasses.json}, style = JSON]
{
    "conformsTo": [
        "https://docs.ogc.org/is/18-062r2/18-062r2.html#toc21",
        "https://docs.ogc.org/is/18-062r2/18-062r2.html#toc40",
        "https://docs.ogc.org/is/18-062r2/18-062r2.html#toc41",
        "https://docs.ogc.org/is/18-062r2/18-062r2.html#toc42",
        "https://docs.ogc.org/is/18-062r2/18-062r2.html#toc47",
        "https://docs.ogc.org/is/18-062r2/18-062r2.html#toc53"
    ],
    "links": [
        {
            "href": "localhost:5000/conformance?f=application/json",
            "rel": "self",
            "type": "application/json",
            "title": "This Document as JSON"
        },
            {
            "href": "localhost:5000/conformance?f=text/html",
            "rel": "alternate",
            "type": "text/html",
            "title": "This Document as HTML"
        }
    ]
}
\end{lstlisting}\label{appendixconfClassesJSON}  

\subsection{processList.html}
\begin{lstlisting}[caption={processList.html}, style = HTML]
<!DOCTYPE html>
<html>
    <body>
        
        <p>
            
            <b>Title:</b> {{process.title}}<br>
            <b>processID: </b>{{process.id}}<br>
            Description: {{process.description}}<br>
            Version: {{process.version}}<br>
            Job control options: {{process.jobControlOptions}}<br>
            Output transmission: {{process.outputTransmission}}<br>
            <p><b>inputs:</b></p>
            <p id='{{process.id}}inputs'><p>
            <p><b>outputs:</b></p>
            <p id='{{process.id}}outputs'><p>
            <script>
                var inputs = '{{process.inputs}}'
                var inputsString = inputs.replaceAll('&#39;', '"')
                var inputsJSON = JSON.parse(inputsString)
                var outputs = '{{process.outputs}}'
                var outputsString = outputs.replaceAll('&#39;', '"')
                var outputsJSON = JSON.parse(outputsString)
                console.log(outputsJSON)
                for (var key in inputsJSON) {
                    var inputs = document.createElement("p")
                    inputs.innerHTML = "<p><b>" + key 
                    + "</b><br>Description: " 
                    + inputsJSON[key].description 
                    + "<br><b>Schema:</b><br>Type: " 
                    + inputsJSON[key].schema.type + "</p>"
                    document.getElementById('{{process.id}}inputs').appendChild(inputs);
                }
                for (var key in outputsJSON) {
                    var outputs = document.createElement("p")
                    outputs.innerHTML = "<p><b>" + key 
                    + "</b><br><b>Description: </b>" 
                    + outputsJSON[key].description 
                    + "<br><b>Schema:</b><br>Type: " 
                    + outputsJSON[key].schema.type + "</p>"
                    document.getElementById('{{process.id}}outputs').appendChild(outputs);
                }
            </script>
            <p>
                <b>links: </b><br>
                href: <a href=localhost:5000/processes/{{process.id}}?f=application/json>
                localhost:5000/processes/{{process["id"]}}?f=application/json</a><br>
                rel: process<br>
                type: application/json<br>
                title: Process description<br><br>
                href: <a href=localhost:5000/processes/{{process.id}}?f=text/html>
                localhost:5000/processes/{{process["id"]}}?f=text/html</a><br>
                rel: process<br>
                type: text/html<br>
                title: Process description<br>
            </p>
            <p>======================================================================</p>
            
        </p>
        
        <p><b>links:</b><br>
            href:<a href="localhost:5000/processes?f=text/html">
            localhost:5000/processes?f=text/html</a><br>
            rel: self<br>
            type: text/html<br>
            title: This document<br>
            <br>
            href:<a href="localhost:5000/processes?f=application/json">
            localhost:5000/processes?f=application/json</a><br>
            rel: alternate<br>
            type: application/json<br>
            title: This document as JSON<br>
        </p>
    </body>
</html>
\end{lstlisting}\label{appendixprocessListHTML}  

\subsection{processDescription.html}
\begin{lstlisting}[caption={processDescription.html}, style = HTML]
<!DOCTYPE html>
<html>
	<body>
        <p>
            <b>processID:</b> {{process.id}}<br>
            <b>Title:</b> {{process.title}}<br>
            <b>Description:</b> {{process.description}}<br>
            <b>Version:</b> {{process.version}}<br>
            <b>Job control options:</b> {{process.jobControlOptions}}<br>
            <b>Output transmission mode:</b> {{process.outputTransmission}}
        </p>
        <p><b>inputs:</b></p>
        <p id='{{process.id}}'><p>
        <script>
            var inputs = '{{process.inputs}}'
            var inputsString = inputs.replaceAll('&#39;', '"')
            var inputsJSON = JSON.parse(inputsString)
            console.log(inputsJSON)
            for (var key in inputsJSON) {
                var newElement = document.createElement("p")
                newElement.innerHTML = "<p><b>" + key 
                + "</b><br>Description: " 
                + inputsJSON[key].description 
                + "<br><b>Schema:</b><br>Type: " 
                + inputsJSON[key].schema.type + "</p>"
                document.getElementById('{{process.id}}').appendChild(newElement);
            }
        </script>
        <p>
        <b>links:</b><br>
        href:<a href="localhost:5000/processes/{{process.id}}?f=text/html">
        localhost:5000/processes/{{process.id}}?f=text/html</a><br>
        rel: self<br>
        type: text/html<br>
        title: This document as HTML<br>
        <br>
        href:<a href="localhost:5000/processes/{{process.id}}?f=application/json">
        localhost:5000/processes/{{process.id}}?f=application/json</a><br>
        rel: alternate<br>
        type: application/json<br>
        title: This document as JSON
		</p>
	</body>
<html>
\end{lstlisting}\label{appendixprocessDescriptionHTML}  

\subsection{FloodMonitoringProcessDescription.json}
\begin{lstlisting}[caption={FloodMonitoringProcessDescription.json}, style = JSON]
{
  "id": "FloodMonitoring",
  "title": "Flood Monitoring",
  "description": "This process accepts a Test input and returns an echo",
  "version": "1.0.0",
  "jobControlOptions": [
    "async-execute", "dismiss"
  ],
  "outputTransmission": [
    "value"
  ],
  "inputs": {
    "preDate": {
      "title": "preDate",
      "description": "The input value",
      "schema": {
        "type": "string"
      }
	},
	"postDate": {
      "title": "postDate",
      "description": "The input value",
      "schema": {
        "type": "string"
      }
	},
	"username": {
      "title": "username",
      "description": "The input value",
      "schema": {
        "type": "string"
      }
	},
	"password": {
      "title": "password",
      "description": "The input value",
      "schema": {
        "type": "string"
      }
	},
	"bbox": {
		"title": "Bounding Box Input Example",
		"description": "This is an example of a BBOX literal input.",
			"schema": {
				"type": "object",
				"required": [
					"bbox"
				],
				"properties": {
					"bbox": {
						"type": "array",
						"oneOf": [
							{
								"minItems": 4,
								"maxItems": 4
							},
							{
								"minItems": 6,
								"maxItems": 6
							}
						],
						"items": {
							"type": "number"
						}
					},
					"crs": {
						"type": "string",
						"format": "uri",
						"default": "http://www.opengis.net/def/crs/OGC/1.3/CRS84",
						"enum": [
							"http://www.opengis.net/def/crs/OGC/1.3/CRS84",
							"http://www.opengis.net/def/crs/OGC/0/CRS84h"
						]
					}
				}
			}
		}
  },
  "outputs": {
   "bin": {
	  "title": "outputDocument",
	  "description": "The output document",
      "schema": {
			"type": "string",
			"contentEncoding": "binary",
			"contentMediaType": "application/tiff"
      }
    },
	"ndsi": {
	  "title": "outputDocument",
	  "description": "The output document",
      "schema": {
			"type": "string",
			"contentEncoding": "binary",
			"contentMediaType": "application/tiff"
      }
    }
  },
  "links": [
	{
		"href": "localhost:5000/processes/Echo?f=application/json",
		"rel": "self",
		"type": "application/json",
		"title": "This document"
	},
	{
		"href": "localhost:5000/processes/Echo?f=text/html",
		"rel": "alternate",
		"type": "text/html",
		"title": "This document as HTML"
	}
  ]
}
\end{lstlisting}\label{appendixFloodMonitoringProcessDescriptionJSON}  

\subsection{EchoProcessDescription.json}
\begin{lstlisting}[caption={EchoProcessDescription.json}, style = JSON]
{
    "id": "Echo",
    "title": "Echo",
    "description": "This process accepts a Test input and returns an echo",
    "version": "1.0.0",
    "jobControlOptions": [
        "async-execute", "dismiss"
    ],
    "outputTransmission": [
        "value"
    ],
    "inputs": {
        "echo": {
        "title": "echo",
        "description": "Value to be echoed",
        "schema": {
            "type": "string"
        }
        }
    },
    "outputs": {
        "outgoingEcho": {
        "title": "outgoingEcho",
        "description": "The output document containing the echoed value",
        "schema": {
            "type": "object",
            "contentMediaType": "application/json",
            "required": [
            "result",
            "message"
            ],
            "properties": {
            "result": {
                "type": "string"
            },
            "message": {
                "type": "string"
            }
            }
        }
        }
    },
    "links": [
        {
            "href": "localhost:5000/processes/Echo?f=application/json",
            "rel": "self",
            "type": "application/json",
            "title": "This document"
        },
        {
            "href": "localhost:5000/processes/Echo?f=text/html",
            "rel": "alternate",
            "type": "text/html",
            "title": "This document as HTML"
        }
    ]
    }     
\end{lstlisting}\label{appendixEchoProcessDescriptionJSON}  
\newpage
\thispagestyle{empty}
\section*{Plagiatserklärung des Studierenden}	

Hiermit versichere ich, dass die vorliegende Arbeit über 

\begin{center}
Rich Data Interfaces for Copernicus Data
\end{center}

selbstständig verfasst worden ist, dass keine anderen 
Quellen und Hilfsmittel als die angegebenen benutzt worden sind und dass die Stellen der 
Arbeit, die anderen Werken – auch elektronischen Medien – dem Wortlaut oder Sinn nach 
entnommen wurden, auf jeden Fall unter Angabe der Quelle als Entlehnung kenntlich gemacht 
worden sind.\\

\vspace{0.75cm}
\parbox{17em}{\hrulefill} \\
Alexander Pilz, Münster den \today

\vspace{0.75cm}

Ich erkläre mich mit einem Abgleich der Arbeit mit anderen Texten zwecks Auffindung von 
Übereinstimmungen sowie mit einer zu diesem Zweck vorzunehmenden Speicherung der Arbeit 
in eine Datenbank einverstanden.\\

\vspace{0.75cm}
\parbox{17em}{\hrulefill} \\
Alexander Pilz, Münster den \today
\end{document}