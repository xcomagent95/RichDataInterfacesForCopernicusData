\newpage
\thispagestyle{empty}
\begin{thebibliography}{1}

\bibitem{tutorial_on_sar}
A. Moreira, M. Younis, P. Prats-Iraola, G. Krieger, I. Hajnsek und K. P. Papathanassiou (2013, April 17). A Tutorial on Synthetic Aperture Radar [Online]. Verfügbar unter: 
https://www.researchgate.net/publication/257008464\_A\_Tutorial\_on\_Synthetic\_Aperture\_Radar
(Zugriff am: 6. Juni 2022).

\bibitem{einfuehrung_in_fernerkundung}
J. Albertz, Einführung in die Fernerkundung, 4. Auflage Darmstadt: Wissenschaftliche Buchgesellschaft, 2009

\bibitem{history_of_copernicus}
Europäische Kommission (2018, Oktober 06). Copernicus: 20 years of History [Online]. Verfügbar unter: 
https://www.copernicus.eu/en/documentation/information-material/signature-esafrance-collaborative-ground-segment
(Zugriff am: 13. Juni 2022).

\bibitem{sentinel_overview}
European Space Agency (2018). Sentinels - Space for Copernicus [Online]. Verfügbar unter: 
https://www.d-copernicus.de/daten/satelliten/daten-sentinels/
(Zugriff am: 13. Juni 2022).

\bibitem{what_is_copernicus}
Europäische Kommission (2019). What is Copernicus [Online]. Verfügbar unter: 
https://www.copernicus.eu/en/documentation/information-material/brochuresbrochures
(Zugriff am: 13. Juni 2022).

\bibitem{sentinel_1_overview}
ESA Communications (2012, März). Sentinel-1 ESA's Radar Observatory Mission for GMES Operational Services [Online]. Verfügbar unter: 
https://sentinel.esa.int/web/sentinel/missions/sentinel-1/overview
(Zugriff am: 13. Juni 2022).

\end{thebibliography}