\newpage
\thispagestyle{empty}
\begin{thebibliography}{1}
\bibitem{einfuehrung_in_fernerkundung}
J. Albertz, Einführung in die Fernerkundung, 4. Auflage Darmstadt: Wissenschaftliche Buchgesellschaft, 2009
    
\bibitem{sentinel_1_definition}
M. Bourbigot, H. Johnson, R. Piantanida (2016, März 03). Sentinel-1 Product Definition [Online]. Verfügbar unter: 
https://sentinel.esa.int/web/sentinel/user-guides/sentinel-1-sar/document-library/-/asset\_publisher/1dO7RF5fJMbd/content/sentinel-1-product-definition
(Zugriff am: 13. Juni 2022).

\bibitem{reproducibility}
Committee on Reproducibility and Replicability in Science (2019, Mai 07). Reproducibility and Replicability in Science [Online]. Verfügbar unter:
https://nap.nationalacademies.org/catalog/25303/reproducibility-and-replicability-in-science
(Zugriff am: 27. Juli 2022).

\bibitem{ogc_specification_model}
S. Cox, D. Danko, J. Greenwood, J.R. Herring, A. Matheus, R. Pearsall, C. Portele, B. Reff, P. Scarponcini, A. Whiteside (2009, Oktober 19). The Specification Model - A Standard for Modular specifications [Online]. Verfügbar unter: 
https://www.ogc.org/standards/modularspec 
(Zugriff am: 27. Juni 2022).

\bibitem{deep_learning_approach}
S. El Amrani Abouelassad (2019, November 30). Flood Detection with a Deep Learning Approach Using Optical and SAR Satellite Data [Online]. Verfügbar unter: 
https://www.ipi.uni-hannover.de/de/lehre/abschlussarbeiten/abgeschlossene-masterarbeiten/
(Zugriff am: 25. Juli 2022).

\bibitem{history_of_copernicus}
Europäische Kommission (2018, Oktober 06). Copernicus: 20 years of History [Online]. Verfügbar unter: 
https://www.copernicus.eu/en/documentation/information-material/signature-esafrance-collaborative-ground-segment
(Zugriff am: 13. Juni 2022).

\bibitem{dias_factsheet}
Europäische Kommission (2018, Juni). The DIAS: User-friendly Access to Copernicus Data and Information [Online]. Verfügbar unter:
https://www.copernicus.eu/en/access-data/dias
(Zugriff am: 24. Juni 2022)

\bibitem{what_is_copernicus}
Europäische Kommission (2019). What is Copernicus [Online]. Verfügbar unter: 
https://www.copernicus.eu/en/documentation/information-material/brochuresbrochures
(Zugriff am: 13. Juni 2022).

\bibitem{copernicus_regulation}
Europäisches Parlament und Rat der Europäischen Union (2014, April 24).  Regulation (EU) No 377/2014 Establishing the Copernicus Programme and repealing Regulation (EU) No 911/2010 [Online]. Verfügbar unter: 
https://www.kowi.de/Portaldata/2/Resources/horizon2020/coop/Copernicus-regulation.pdf
(Zugriff am: 13. Juni 2022).

\bibitem{open_access_hub_overview}
European Space Agency (2022). Copernicus Open Access Hub - Overview [Online]. Verfügbar unter:
https://scihub.copernicus.eu/userguide/WebHome
(Zugriff am: 10. Juni 2022).

\bibitem{sentinel_1_overview}
European Space Agency (2012, März). Sentinel-1 ESA's Radar Observatory Mission for GMES Operational Services [Online]. Verfügbar unter: 
https://sentinel.esa.int/web/sentinel/missions/sentinel-1/overview
(Zugriff am: 13. Juni 2022).

\bibitem{sentinel_overview}
European Space Agency (2018). Sentinels - Space for Copernicus [Online]. Verfügbar unter: 
https://www.d-copernicus.de/daten/satelliten/daten-sentinels/
(Zugriff am: 13. Juni 2022).

\bibitem{geospatial_apis}
C. Holmes, D. WWesloh, C. Heazel, G. Gale, A. Christl, J. Lieberman, C. Reed, J. Herring, M. Desruisseaux, D. Blodgett, S. Simmons, B. de Lathower und G. Percivall (2017, Februar 23). OGC Open Geospatial APIs - White Paper [Online]. Verfügbar unter: 
https://docs.ogc.org/wp/16-019r4/16-019r4.html
(Zugriff am: 03. Juli 2022).

\bibitem{testbed_11}
F. Houbie, S. Sankaran, J. Lieberman, P. Vretanos, J. Masó (2016, Januar 16). OGC Testbed 11 REST Interface Engineering Report [Online]. Verfügbar unter: 
https://www.ogc.org/docs/er
(Zugriff am: 05. Juli 2022).

\bibitem{speckle_filtering}
J.S. Lee, I. Juekevich, P. Dewaele, P. Wambacq, A. Oosterlinck (1994, Februar). Speckle filtering of synthetic aperture radar images: A Review [Online]. Verfügbar unter: 
https://www.researchgate.net/publication/239659062\_Speckle\_filtering\_of\_synthetic\_aperture
\_radar\_images\_A\_Review
(Zugriff am: 27. Juli 2022).

\bibitem{api_design}
M. Biehl, API Design, 1. Auflage Suurstoffi: API-University Press, 2016

\bibitem{testbed_16}
J. Maso, A. Zabala and A. Brodia (2021, Januar 01). OGC Testbed-16: Analysis Ready Data Engineering Report [Online]. Verfügbar unter:
https://scihub.copernicus.eu/userguide/WebHome
(Zugriff am: 11. Juni 2022).

\bibitem{sentinel_1_flood_mapping_tutorial}
A. McVittie (2019, Februar). Sentinel-1 Flood mapping tutorial [Online]. Verfügbar unter: 
https://step.esa.int/main/doc/tutorials/
(Zugriff am: 15. Juni 2022).

\bibitem{open_api}
D. Miller, J. Whitlock, M. Gardiner, M. Ralphson, R. Ratovsky, U. Sarid (2021, February 15). OpenAPI Specification v3.1.0 [Online]. Verfügbar unter: 
https://spec.openapis.org/oas/v3.0.1
(Zugriff am: 07. Juli 2022).

\bibitem{radiometric_calibration_of_S1_level1_products}
N. Miranda und P.J. Meadows (2015, Mai 21). Radiometric Calibration of S-1 Level-1 Products Generated by the S-1 IPF [Online]. Verfügbar unter: 
https://sentinel.esa.int/web/sentinel/user-guides/document-library/-/asset\_publisher/xlslt4309D5h/content/sentinel-1-radiometric-calibration-of-products-generated-by-the-s1-ipf
(Zugriff am: 15. Juni 2022).

\bibitem{tutorial_on_sar}
A. Moreira, M. Younis, P. Prats-Iraola, G. Krieger, I. Hajnsek und K. P. Papathanassiou (2013, April 17). A Tutorial on Synthetic Aperture Radar [Online]. Verfügbar unter: 
https://www.researchgate.net/publication/257008464\_A\_Tutorial\_on\_Synthetic\_Aperture\_Radar
(Zugriff am: 6. Juni 2022).

\bibitem{nielsen_poster}
J. Nielsen (2020, November 15). 10 Usability Heuristics for User Interface Design [Online]. Verfügbar unter: 
https://www.nngroup.com/articles/ten-usability-heuristics/
(Zugriff am: 22. Juli 2022).

\bibitem{heuristics_website}
J. Nielsen (2020, November 15). 10 Usability Heuristics for User Interface Design [Online]. Verfügbar unter: 
https://www.nngroup.com/articles/ten-usability-heuristics/
(Zugriff am: 27. Juli 2022).

\bibitem{usability_engineering}
J. Nielsen, Usability Engineering, Mountain View: Academic Press Inc., 1993

\bibitem{ogc_bylaws}
Open Geospatial Consortium (2021, Dezember 16). Bylaws of Open Geospatial Consortium [Online]. Verfügbar unter: 
https://www.ogc.org/ogc/policies
(Zugriff am: 27. Juni 2022).

\bibitem{standards_architecture_diagram}
Open Geospatial Consortium (2022). OGC Standards [Online]. Verfügbar unter: 
https://www.ogc.org/docs/is
(Zugriff am: 27. Juli 2022).

\bibitem{thresholds_selection}
N. Otsu (1976, Januar). A Threshold Selection Method from Gray-Level Histograms [Online]. Verfügbar unter: 
https://ieeexplore.ieee.org/document/4310076/citations\#citations
(Zugriff am: 14. Juni 2022).

\bibitem{code}
A. Pilz (2022, August 08). Prototypical Rich Data Interface for Copernicus Data [Online]. Verfügbar unter: 
https://zenodo.org/record/6979963
(Zugriff am: 10. August 2022).

\bibitem{ogc_api_processes_core}
B. Pross und P. A. Vretanos. (2021, Dezember 20). OGC API – Processes – Part 1: Core [Online]. Verfügbar unter: 
https://docs.opengeospatial.org/is/18-062r2/18-062r2.html 
(Zugriff am: 24. Juni 2022).

\bibitem{flood_proxy_mapping_ndsi}
N. I. Ulloa, S.-H. Chiang und S.-H. Yun (2020, April 27). Flood Proxy Mapping with Normalized Difference Sigma-Naught Index and Shannon’s Entropy [Online]. Verfügbar unter: 
https://doi.org/10.3390/rs12091384 
(Zugriff am: 21. Juni 2022).



























\end{thebibliography}