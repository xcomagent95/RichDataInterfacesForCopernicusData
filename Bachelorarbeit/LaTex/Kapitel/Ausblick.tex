\restoregeometry
\section{Fazit und Ausblick}
Im Rahmen dieser Arbeit konnte ein funktionsfähiges Rich Data Interface für die Daten des Copernicus-Programmes 
implementiert werden. Teil der Anwendung ist eine 
API, welche diese wesentlichen im OGC API - Processes - Part 1: Core Standard beschriebenen Funktionen anbietet. 
Mithilfe der API kann Überschwemmungsmonitoring auf Basis der Radardaten der Sentinel-1 Mission betrieben werden. 
Die heuristische Untersuchung dieser prototypischen Implementierung konnte bestätigen das sich der OGC API - Processes - Part 1: Core
Standard gut dazu eignet nutzerfreundliche APIs für Abwendungen zu entwickeln, welche Nutzer*innen in die 
Lage versetzten komplexe Verarbeitungen von Daten des Copernicus-Programmes durchzuführen. 
Nutze*innen können dabei sowohl analysebereite als auch interpretationsfähige Daten zur Verfügung gestellt werden.
Ebenso konnte bestätigt werden das sich solche Anwendungen mir der Programmiersprache Python sowie die zur Verfügung stehenden Packages
implementieren lassen. Die Nutzung des \verb|flask|-Frameworks erlaubt schnell und leicht APIs zu implementieren während Packages wie 
\verb|osgeo| und \verb|skimage| auch komplexe Verarbeitungen von Rasterdaten erlauben. Die Daten des Copernicus-Programmes 
können ebenfalls in sinnvoller Weise mit einer Anwendung gekoppelt werden. Die Beschaffung von Datensätzen ist mit dem 
\verb|sentinelsat|-Package gut umsetzbar während der Python-Wrapper \verb|snappy| die vollständige Kalibrierung der Datensätze ermöglicht. 
Die limitierenden Eigenschaften des Copernicus Open Access Hub können zumindest für eine begrenzte Anzahl von Datensätzen umgangen werden.
Das vorgestellte Verfahren zum Detektieren von Überschwemmungen funktioniert unter optimalen Bedingungen hinreichend gut.
Gleichzeitig konnte die vorgestellte Anwendung deutlich Potentiale zur Weiterentwicklung aufzeigen. \\


Um die vorgestellte prototypische Anwendung zu einem Softwareprodukt zu machen, welches produktiv eingesetzt werden 
kann, sollten einige Aspekte ausgebaut und erweitert werden. 
Zunächst sollte die Installation der Anwendung deutlich vereinfacht werden, da das Erzeugen eines passenden Python-Environments und 
die Konfiguration von SNAP sich als umständlich und zeitaufwendig erweisen können. 
Eine mögliche Lösung wäre die Containerisierung der Anwendung mithilfe von Docker \cite{testbed_16}. 

Auch werden bisher nicht alle Aspekte des OGC API - Processes - Part 1: Core Standards umgesetzt. Die Anwendung sollte im nächsten Schritt also 
vervollständigt werden um zum Beispiel auch den Transmission-Mode \verb|reference| zu unterstützen und einen Callback-Mechanismus bereitzustellen. 
Auch Recommendations wie die Unterstützung von HTTPS und CORS sollten erwogen werden.  
Zudem sollte die Anwendung durch das OGC Validierungsverfahren auf ihre Standardkonformität hin überprüft und gegebenenfalls durch die OGC zertifiziert werden. 

%Ein wichtiger Aspekt computergestützter Verfahren ist ihre Reproduzierbarkeit. Das Committee on Reproducibility and Replicability in Science definiert 
%ein vorgestelltes computergestütztes Verfahren als reproduzierbar, wenn mit identischen Eingaben und unter gleichen Systembedingungen identische Resultate erzielt werden können.  
%Dies bedeutet zunächst, dass die verwendete Software quelloffen und kostenlos zur Verfügung steht, um Interessierte in die Lage zu versetzen, das vorgestellte Verfahren
%selbst durchzuführen. Um dies so einfach wie möglich zu machen, sollte die verwendete Software detailliert dokumentiert sein. Ein besonderes Augenmerk sollte dabei auf den 
%zugrundeliegenden Daten, den verwendeten Methoden und der ursprünglichen Systemumgebung liegen \cite{reproducibility}. 
%Die API der prototypischen Implementierung ist durch die Dokumente der OGC und die API-Definition gut und ausführlich dokumentiert. Die zum Überschwemmungsmonitoring 
%verwendeten Methoden in dieser Arbeit und den im Rahmen dieser Arbeit zurate gezogenen Arbeiten ausführlich beschrieben. 
%Diese sollten jedoch gesammelt und gegebenenfalls 
%im Git-Repository verknüpft werden. Zusätzlich sollte ein ausführbares Prozessierunsgbeispiel ergänzt werden mit dem Nutzer*innen die Anwendung erproben können. Die während dieser 
%Arbeit verwendete Systemumgebung ist lediglich über die \verb|environment.yaml| beschrieben. \\
Darüber hinaus sollte die Anwendung sollte aus Gründen der Reproduzierbarkeit um eine zusammengefasste und aufbereitete Methodenbeschreibung sowie ein 
ausführbares Beispiel ergänzt werden.

%Werden keine Daten für einen bestimmten Zeitraum und einen bestimmten Raum in der Anwendung hinterlegt, müssen sie aus dem Copernicus Open Access Hub heruntergeladen werden. 
%Aus diesem können nicht alle Daten der Sentinel-1 Mission direkt heruntergeladen werden, sondern nur jene, welche sich nicht im 
%Langzeitarchiv befinden. Diese Limitierungen können teilweise umgangen werden, wenn die DIAS Plattformen zur Datenbeschaffung genutzt werden. \\
Da die Limitierungen der vorgestellten Kopplung zu den Daten des Copernicus-Programmes möglicherweise durch die Nutzung von DIAS Plattformen umgangen werden können, sollten
diese als alternative Datenquellen erschlossen werden. 

%Das momentan für die Schwellwertermittlung verwendete Verfahren liefert für bimodale Verteilungen belastbare Schwellwerte. 
Um die Schwellwertbildung unabhängiger von den Eigenschaften der Verteilung der Reflexionswerte zu machen, sollten 
andere Verfahren untersucht und gegebenenfalls implementiert werden. Dies können komplexere statistische Verfahren wie der SNDSI, aber auch maschinelle Lernverfahren und 
Modelle sein \cite{flood_proxy_mapping_ndsi,deep_learning_approach}. 
Des Weiteren sollte untersucht werden ob sich die vorgestellte Anwendung in sinnvoller Wiese 
um die Konzepte der CARD4L Initiative des Committee on Earth Observation Satellites (CEOS) erweitert werden kann. 
Dazu zählen unter anderem die Formulierung von Produkt Spezifikationen sowie die Einführung von Qualitätsmerkmalen für die bereitgestellten Daten \cite{testbed_16}. 

Im Rahmen dieser Arbeit wurden nur die Nutzbarkeitsaspekte evaluiert. Zusätzlich sollten jedoch auch Aspekte wie die Wartbarkeit, Skalierbarkeit und Performanz näher 
mit geeigneten Heuristiken und Verfahren untersucht werden. Falls Aspekte mit Metriken bewertet werden können, sollten die heuristischen Evaluationen um diese ergänzt werden. 
Mit Hinblick auf eine mögliche technische Evaluationen scheint es sinnvoll Frameworks wie \verb|django| und andere Programmiersprachen wie R und Java zu erproben da auch für diese 
eine Vielzahl von Packages und Bibliotheken zur Verfügung stehen um Webanwendungen und Prozesse zu entwickeln. 




