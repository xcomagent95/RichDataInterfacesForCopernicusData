\newpage
\restoregeometry

\makeatletter
\newenvironment{breakablealgorithm}
  {% \begin{breakablealgorithm}
   \begin{center}
     \refstepcounter{algorithm}% New algorithm
     \hrule height.8pt depth0pt \kern2pt% \@fs@pre for \@fs@ruled
     \renewcommand{\caption}[2][\relax]{% Make a new \caption
       {\raggedright\textbf{\fname@algorithm~\thealgorithm} ##2\par}%
       \ifx\relax##1\relax % #1 is \relax
         \addcontentsline{loa}{algorithm}{\protect\numberline{\thealgorithm}##2}%
       \else % #1 is not \relax
         \addcontentsline{loa}{algorithm}{\protect\numberline{\thealgorithm}##1}%
       \fi
       \kern2pt\hrule\kern2pt
     }
  }{% \end{breakablealgorithm}
     \kern2pt\hrule\relax% \@fs@post for \@fs@ruled
   \end{center}
  }
\makeatother

\counterwithin{lstlisting}{section}
\renewcommand{\lstlistingname}{Pseudocode} 
\section{Pseudocode} 
\subsection{Ablauf eines Requests an den API Landing Page Endpoint}
\begin{breakablealgorithm}
\caption{Ablauf eines Requests an den API Landing Page Endpoint}\label{PseudocodeLandingPage}
\scriptsize
\begin{algorithmic} 
    \STATE \textbf{Input:} Request    
    \IF{Request.HTTP-Method != GET}
        \RETURN{HTTP-Statuscode 405}
    \ELSE{}
        \STATE \hskip0.5em \textbf{try}
        \begin{ALC@g}
        \IF{Request.f == text/html \OR Request.f == None}
            \STATE Response $\gets$ render\_template(templates/html/landingPage.html)
            \STATE Link-Header $\gets$ http://HOST:PORT/?f=text/html
            \STATE Resource-Header $\gets$ landingPage
            \RETURN{Response with Link- and Resource-Header \AND HTTP-Statuscode 200}
        \ELSIF{Request.f == application/json}
            \STATE JSON $\gets$ open(templates/json/landingPage.json)
            \STATE Response $\gets$ jsonify(JSON)
            \STATE Link-Header $\gets$ http://HOST:PORT/?f=application/json
            \STATE Resource-Header $\gets$ landingPage
            \RETURN{Response with Link- and Resource-Header \AND HTTP-Statuscode 200}
        \ELSE{}
            \RETURN{HTTP-Statuscode 406}
        \ENDIF{}
        \end{ALC@g}
        \STATE \hskip0.5em \textbf{except}
        \begin{ALC@g}
            \RETURN{HTTP-Statuscode 500}
        \end{ALC@g}
    \ENDIF{}
\end{algorithmic}
\end{breakablealgorithm}

\subsection{Ablauf eines Requests an den API Definition Endpoint}
\begin{breakablealgorithm}
\caption{Ablauf eines Requests an den API Definition Endpoint}\label{PseudocodeAPIDefinition}
\scriptsize
\begin{algorithmic}     
    \STATE \textbf{Input:} Request   
    \IF{Request.HTTP-Method != GET}
        \RETURN{HTTP-Statuscode 405}
    \ELSE{}
        \STATE \hskip0.5em \textbf{try}
        \begin{ALC@g}
        \IF{Request.f == text/html \OR Request.f == None}
            \STATE Response $\gets$ render\_template(templates/html/apiDefinition.html)
            \STATE Link-Header $\gets$ http://HOST:PORT/api?f=text/html
            \STATE Resource-Header $\gets$ apiDefinition
            \RETURN{Response with Link- and Resource-Header \AND HTTP-Statuscode 200}
        \ELSIF{Request.f == application/json}
            \STATE JSON $\gets$ open(templates/json/apiDefinition.json)
            \STATE Response $\gets$ jsonify(JSON)
            \STATE Link-Header $\gets$ http://HOST:PORT/api?f=application/json
            \STATE Resource-Header $\gets$ apiDefinition
            \RETURN{Response with Link- and Resource-Header \AND HTTP-Statuscode 200}
        \ELSE{}
            \RETURN{HTTP-Statuscode 406}
        \ENDIF{}
        \end{ALC@g}
        \STATE \hskip0.5em \textbf{except}
        \begin{ALC@g}
            \RETURN{HTTP-Statuscode 500}
        \end{ALC@g}
    \ENDIF{}
\end{algorithmic}
\end{breakablealgorithm}

\newpage
\subsection{Ablauf eines Requests an den Conformance Endpoint}
\begin{breakablealgorithm}
\caption{Ablauf eines Requests an den Conformance Endpoint}\label{PseudocodeConformance}
\scriptsize
\begin{algorithmic}    
    \STATE \textbf{Input:} Request    
    \IF{Request.HTTP-Method != GET}
        \RETURN{HTTP-Statuscode 405}
    \ELSE{}
        \STATE \hskip0.5em \textbf{try}
        \begin{ALC@g}
        \IF{Request.f == text/html \OR Request.f == None}
            \STATE Response $\gets$ render\_template(templates/html/confClasses.html)
            \STATE Link-Header $\gets$ http://HOST:PORT/conformance?f=text/html
            \STATE Resource-Header $\gets$ conformance
            \RETURN{Response with Link- and Resource-Header \AND HTTP-Statuscode 200}
        \ELSIF{Request.f == application/json}
            \STATE JSON $\gets$ open(templates/json/confClasses.json)
            \STATE Response $\gets$ jsonify(JSON)
            \STATE Link-Header $\gets$ http://HOST:PORT/conformance?f=application/json
            \STATE Resource-Header $\gets$ conformance
            \RETURN{Response with Link- and Resource-Header \AND HTTP-Statuscode 200}
        \ELSE{}
            \RETURN{HTTP-Statuscode 406}
        \ENDIF{}
        \end{ALC@g}
        \STATE \hskip0.5em \textbf{except}
        \begin{ALC@g}
            \RETURN{HTTP-Statuscode 500}
        \end{ALC@g}
    \ENDIF{}
\end{algorithmic}
\end{breakablealgorithm}

\newpage
\subsection{Ablauf eines Requests an den Process List Endpoint}
\begin{breakablealgorithm}
\caption{Ablauf eines Requests an den Process List Endpoint}\label{PseudocodeProcessList}
\scriptsize
\begin{algorithmic}    
    \STATE \textbf{Input:} Request    
    \IF{Request.HTTP-Method != GET}
        \RETURN{HTTP-Statuscode 405}
    \ELSE{}
        \STATE \hskip0.5em \textbf{try}
        \begin{ALC@g}
            \IF{Request.limit == None \OR Request.limit <= 0 \OR Request.limit > 100000}
                \STATE Limit $\gets$ 10
            \ELSE{}
                \STATE Limit $\gets$ Anfrage.limit
            \ENDIF
        \IF{Request.f == text/html \OR Request.f == None}
            \STATE Process List $\gets$ []
            \STATE Processes $\gets$ Process Descriptions in templates/json/processes
            \FOR{Process \textbf{in} Processes}
                \STATE JSON $\gets$ open(templates/json/processes/<Process>ProcessDescription.json)
                \STATE Process List \textbf{append} JSON
            \ENDFOR{}
            \STATE Response $\gets$ render\_template(templates/html/processList.html, Process List[0:Limit])
            \STATE Link-Header $\gets$ http://HOST:PORT/processList?f=text/html
            \STATE Resource-Header $\gets$ processList
            \RETURN{Response with Link- and Resource-Header \AND HTTP-Statuscode 200}
        \ELSIF{Request.f == application/json}
            \STATE Process List $\gets$ []
            \STATE Processes $\gets$ List of Process descriptions in templates/json/processes
            \FOR{Process \textbf{in} Processes}
                \STATE JSON $\gets$ open(templates/json/processes/<Process>ProcessDescription.json)
                \STATE Process List \textbf{append} JSON
            \ENDFOR{}
            \STATE \textbf{add} Links to self and alternate \textbf{to} Process List
            \STATE Response $\gets$ jsonify(Process List[0:Limit])
            \STATE Link-Header $\gets$ http://HOST:PORT/processList?f=application/json
            \STATE Resource-Header $\gets$ processList
            \RETURN{Response with Link- and Resource-Header \AND HTTP-Statuscode 200}
        \ELSE{}
            \RETURN{HTTP-Statuscode 406}
        \ENDIF{}
        \end{ALC@g}
        \STATE \hskip0.5em \textbf{except}
        \begin{ALC@g}
            \RETURN{HTTP-Statuscode 500}
        \end{ALC@g}
    \ENDIF{}
\end{algorithmic}
\end{breakablealgorithm}

\newpage
\subsection{Ablauf eines Requests an den Process Description Endpoint}
\begin{breakablealgorithm}
\caption{Ablauf eines Requests an den Process Description Endpoint}\label{PseudocodeProcessDescription}
\scriptsize
\begin{algorithmic}     
    \STATE \textbf{Input:} Request, Process-ID   
    \IF{Request.HTTP-Method != GET}
        \RETURN{HTTP-Statuscode 405}
    \ELSE{}
        \STATE \hskip0.5em \textbf{try}
        \begin{ALC@g}
            \IF{Request.f == text/html \OR Request.f == None}
                \IF{templates/json/processes/<Process-ID>ProcessDescription.json \textbf{exists}}
                    \STATE JSON $\gets$ open(templates/json/processes/<Process-ID>ProcessDescription.json)
                    \STATE Response $\gets$ render\_template(templates/html/process.html, JSON)
                    \STATE Link-Header $\gets$ http://HOST:PORT/processes/<Process-ID>?f=text/html
                    \STATE Resource-Header $\gets$ process - Request.<Process-ID>
                    \RETURN{Response with Link- and Resource-Header \AND HTTP-Statuscode 200}
                \ELSE{}
                    \STATE Exception $\gets$ No such process exception
                    \STATE Resource-Header $\gets$ no-such-process
                    \RETURN{Exception with Resource-Header \AND HTTP-Statuscode 404}
                \ENDIF{}
            \ELSIF{Request.f == application/json}
                \IF{templates/json/processes/<Process-ID>ProcessDescription.json \textbf{exists}}
                    \STATE JSON $\gets$ open(templates/json/processes/<Process-ID>ProcessDescription.json)
                    \STATE Response $\gets$ jsonify(JSON)
                    \STATE Link-Header $\gets$ http://HOST:PORT/processes/<Process-ID>?f=application/json
                    \STATE Resource-Header $\gets$ process - Request.<Process-ID>
                    \RETURN{Response with Link- and Resource-Header \AND HTTP-Statuscode 200}
                \ELSE{}
                    \STATE Exception $\gets$ No such process exception
                    \STATE Resource-Header $\gets$ no-such-process
                    \RETURN{Exception with Resource-Header \AND HTTP-Statuscode 404}
                \ENDIF{}
        \ELSE{}
            \RETURN{HTTP-Statuscode 406}
        \ENDIF{}
        \end{ALC@g}
        \STATE \hskip0.5em \textbf{exept}
        \begin{ALC@g}
            \RETURN{HTTP-Statuscode 500}
        \end{ALC@g}
    \ENDIF{}
\end{algorithmic}
\end{breakablealgorithm}

\newpage
\subsection{Ablauf eines Requests an den Process Execution Endpoint}
\begin{breakablealgorithm}
\caption{Ablauf eines Requests an den Process Execution Endpoint}\label{PseudocodeProcessExecution}
\scriptsize
\begin{algorithmic}  
    \STATE \textbf{Input:} Request, Process-ID    
    \IF{Request.HTTP-Methode != POST}
        \RETURN{HTTP-Statuscode 405}
    \ELSE{}
        \STATE \hskip0.5em \textbf{try}
        \begin{ALC@g}
            \IF{templates/json/processes/<Process-ID>ProcessDescription.json \textbf{exists}}   
                \STATE Input Parameters $\gets$ parse(Request.Parameters)
                \IF{Input Parameters == False}
                    \RETURN{HTTP-Statuscode 400}
                \ELSE{}
                    \STATE Job-ID $\gets$ generateUUID()
                    \STATE \textbf{create} Job Directory
                    \STATE \textbf{create} Results Directory
                    \STATE \textbf{create} status.json
                    \STATE \textbf{create} job.json
                    \STATE Response $\gets$ jsonify(status.json)
                    \STATE Location-Header $\gets$ http://HOST:PORT/jobs/<Job-ID>?f=application/json
                    \STATE Resource-Header $\gets$ job - Job-ID
                    \RETURN{Response with Location- and Resource-Header \AND HTTP-Statuscode 201}
                \ENDIF{}
            \ELSE{}
                \STATE Exception $\gets$ No such process exception
                \STATE Resource-Header $\gets$ no-such-process
                \RETURN{Exception with Resource-Header \AND HTTP-Statuscode 404}
            \ENDIF{}
        \end{ALC@g}
        \begin{ALC@g}
            \RETURN{HTTP-Statuscode 500}
        \end{ALC@g}   
    \ENDIF{}     
\end{algorithmic}
\end{breakablealgorithm}

\newpage
\subsection{Ablauf eines Requests an den Job List Endpoint}
\begin{breakablealgorithm}
\scriptsize
\caption{Ablauf eines Requests an den Job List Endpoint}\label{PseudocodeJobList}
\begin{algorithmic}     
    \STATE \textbf{Input:} Request
    \IF{Request.HTTP-Method != GET}
        \RETURN{HTTP-Statuscode 405}
    \ELSE{}
        \STATE \hskip0.5em \textbf{try}
        \begin{ALC@g}
            \IF{Request.limit == None \OR Request.limit <= 0 \OR Request.limit > 100000}
                \STATE Limit $\gets$ 10
            \ELSE{}
                \STATE Limit $\gets$ Request.limit
            \ENDIF
            \IF{Request.type == None}
                \STATE Type = [process]
            \ELSE{}
                \STATE Type = Request.type
            \ENDIF{}

            \IF{Request.processID == None}
                \STATE Process-ID = [Echo, FloodMonitoring]
            \ELSE{}
                \STATE Process-ID = request.processID
            \ENDIF{}

            \IF{Request.status == None}
                \STATE Status = [accepted, running, successful, failed, dismissed]
            \ELSE{}
                \STATE Status = request.status
            \ENDIF{}
            \STATE Jobs $\gets$ Jobs in jobs/
            \STATE Job List $\gets$ []
            \FOR{Job \textbf{in} Jobs}
                \STATE JSON $\gets$ open(jobs/Job/status.json)
                \STATE Creation $\gets$ checkCreationDate(JSON.created, Request)
                \STATE Duration $\gets$ checkDuration(JSON, Request)
                \IF{JSON.type \textbf(in) Type \AND JSON.processID \textbf(in) Process-ID \AND JSON.status \textbf{in} Status
                    \AND Creation == True \AND Duration[0] == True \AND Duration[1] == True}
                    \STATE Job List \textbf{append} JSON
                \ENDIF{}
            \ENDFOR{}

            \IF{Request.f == text/html \OR Request.f == None}
                \STATE Response $\gets$ render\_template(templates/html/jobList.html, Job List[0:Limit])
                \STATE Link-Header $\gets$ http://HOST:PORT/jobs?f=text/html
                \STATE Resource-Header $\gets$ jobs
                \RETURN{Response with Link- and Resource-Header \AND HTTP-Statuscode 200}
            \ELSIF{Request.f == application/json}
                \STATE \textbf{add} Links to self and alternate \textbf{to} Job List
                \STATE Response $\gets$ jsonify(Job Liste[0:Limit])
                \STATE Link-Header $\gets$ http://HOST:PORT/jobs?f=application/json
                \STATE Resource-Header $\gets$ jobs
                \RETURN{Response with Link- and Resource-Header \AND HTTP-Statuscode 200}
            \ELSE{}
                \RETURN{HTTP-Statuscode 406}
            \ENDIF{}
        \end{ALC@g}
        \STATE \hskip0.5em \textbf{exept}
        \begin{ALC@g}
            \RETURN{HTTP-Statuscode 500}
        \end{ALC@g}
    \ENDIF{}
\end{algorithmic}
\end{breakablealgorithm}

\newpage
\subsection{Ablauf eines Requests an den Job Endpoint mit HTTP-Methode Get}
\begin{breakablealgorithm}
\scriptsize
\caption{Ablauf eines Requests an den Job Status Endpoint mit HTTP-Methode Get}\label{PseudocodeJobStatusGET}
\begin{algorithmic}     
    \STATE \textbf{Input:} Request, Job-ID   
    \IF{Request.HTTP-Method == GET}
        \STATE \hskip0.5em \textbf{try}
        \begin{ALC@g}
            \IF{Request.f == text/html \OR Request.f == None}
                \IF{jobs/<Job-ID>/status.json \textbf{exists}}
                \STATE JSON $\gets$ open(jobs/<Job-ID>/status.json)
                \STATE Response $\gets$ render\_template(templates/html/job.html, JSON)
                \STATE Link-Header $\gets$ http://HOST:PORT/jobs/<Job-ID>?f=text/html
                    \STATE Resource-Header $\gets$ job - <Job-ID>
                    \RETURN{Response with Link- and Resource-Header \AND HTTP-Statuscode 200}
                \ELSE{}
                    \STATE Exception $\gets$ No such job exception
                    \STATE Resource-Header $\gets$ no-such-job
                    \RETURN{Exception with Resource-Header \AND HTTP-Statuscode 404}
                \ENDIF{}
            \ELSIF{Request.f == application/json}
                \IF{jobs/<Job-ID>/status.json \textbf{exists}}
                    \STATE Job $\gets$ open(jobs/<Job-ID>/status.json)
                    \STATE \textbf{add} Links to self and alternate \textbf{to} Job
                    \STATE Response $\gets$ jsonify(Job)
                    \STATE Link-Header $\gets$ http://HOST:PORT/jobs/<Job-ID>?f=application/json
                    \STATE Resource-Header $\gets$ job - <Job-ID>
                    \RETURN{Response with Link- and Resource-Header \AND HTTP-Statuscode 200}
                \ELSE{}
                    \STATE Exception $\gets$ No such job exception
                    \STATE Resource-Header $\gets$ no-such-job
                    \RETURN{Exception with Resource-Header \AND HTTP-Statuscode 404}
                \ENDIF
            \ELSE{}
                \RETURN{HTTP-Statuscode 406}
            \ENDIF{}
        \end{ALC@g}
        \STATE \hskip0.5em \textbf{exept}
        \begin{ALC@g}
            \RETURN{HTTP-Statuscode 500}
        \end{ALC@g}
    \ELSIF{Request.HTTP-Method == DELETE}
        \STATE Handling of Request using HTTP DELETE
    \ELSE{}
        \RETURN{HTTP-Statuscode 405}
    \ENDIF{}
\end{algorithmic}
\end{breakablealgorithm}

\newpage
\subsection{Ablauf eines Requests an den Job Endpoint mit HTTP-Methode Delete}
\begin{breakablealgorithm}
\scriptsize
\caption{Ablauf eines Requests an den Job Status Endpoint mit HTTP-Methode Delete}\label{PseudocodeJobStatusDELETE}
\begin{algorithmic}     
    \STATE \textbf{Input:} Request, Job-ID   
    \IF{Request.HTTP-Method == GET}
        \STATE Handling of Request using HTTP GET
    \ELSIF{Request.HTTP-Method == DELETE}
        \STATE \hskip0.5em \textbf{try}
        \begin{ALC@g}
            \IF{jobs/<Job-ID>/status.json \textbf{exists}}
                \STATE Job $\gets$ open(jobs/<Job-ID>/status.json)
                \IF{Job.status != dismissed}
                    \STATE Job $\gets$ open(jobs/<Job-ID>/status.json)
                    \IF{Encoding == text/html \OR Encoding == None}
                        \STATE Response $\gets$ render\_template(templates/html/job.html)
                        \STATE Link-Header $\gets$ http://HOST:PORT/jobs/<Job-ID>?f=text/html
                        \STATE Resource-Header $\gets$ job-dismissed
                        \RETURN{Response with Link- and Resource-Header \AND HTTP-Statuscode 200}
                    \ELSIF{Encoding == application/json}
                        \STATE Response $\gets$ jsonify(Job)
                        \STATE Link-Header $\gets$ http://HOST:PORT/jobs/<Job-ID>?f=application/json
                        \STATE Resource-Header $\gets$ job-dismissed
                        \RETURN{Response with Link- and Resource-Header \AND HTTP-Statuscode 200}
                    \ELSE{
                        \RETURN{HTTP-Statuscode 406}
                    }
                    \ENDIF
                \ELSE{}
                    \IF{Encoding == text/html \OR Encoding == None}
                        \STATE Response $\gets$ render\_template(templates/html/job.html)
                        \STATE Link-Header $\gets$ http://HOST:PORT/jobs/<Job-ID>?f=text/html
                        \STATE Resource-Header $\gets$ job-dismissed
                        \RETURN{Response with Link- and Resource-Header \AND HTTP-Statuscode 410}
                    \ELSIF{Encoding == application/json}
                        \STATE Response $\gets$ jsonify(Job)
                        \STATE Link-Header $\gets$ http://HOST:PORT/jobs/<Job-ID>?f=application/json
                        \STATE Resource-Header $\gets$ job-dismissed
                        \RETURN{Response with Link- and Resource-Header \AND HTTP-Statuscode 410}
                    \ELSE{
                        \RETURN{HTTP-Statuscode 406}
                    }
                    \ENDIF{}
                \ENDIF{}
            \ELSE{}
                \STATE Exception $\gets$ No such job exception
                \STATE Resource-Header $\gets$ no-such-job
                \RETURN{Exception with Resource-Header \AND HTTP-Statuscode 404}
            \ENDIF
        \end{ALC@g}
        \STATE \hskip0.5em \textbf{exept}
        \begin{ALC@g}
            \RETURN{HTTP-Statuscode 500}
        \end{ALC@g}
    \ELSE{}
        \RETURN{HTTP-Statuscode 405}
    \ENDIF{}
\end{algorithmic}
\end{breakablealgorithm}

\newpage
\subsection{Ablauf eines Requests an den Job Results Endpoint}
\begin{breakablealgorithm}
\scriptsize
\caption{Ablauf eines Requests an den Job Results Endpoint}\label{PseudocodeJobResults}
\begin{algorithmic}    
    \STATE \textbf{Input:} Request, Job-ID    
    \IF{Request.HTTP-Method != GET}
        \RETURN{HTTP-Statuscode 405}
    \ELSE{}
    \STATE \hskip0.5em \textbf{try}
    \begin{ALC@g}
        \IF{jobs/<Job-ID>/status.json \textbf{exists}}
            \STATE Job Jobfile $\gets$ open(jobs/<Job-ID/job.json)
            \STATE Job Status $\gets$ open(jobs/<Job-ID/status.json)
            \IF{Job Jobfile.processID == Echo}
                \IF{Job Status.status == successful}
                    \IF{Job Jobfile.responseType == raw}
                        \RETURN{jobs/<jobID>/results/result.json \AND  HTTP-Statuscode 200}
                    \ELSE{}
                    \STATE Result-Value $\gets$ open(jobs/<Job-ID>/results/result.json)
                    \STATE \textbf{embed} Results-Value \textbf{into} Result-Document
                    \RETURN Result-Document \AND HTTP-Statuscode 200
                    \ENDIF{}
                \ELSIF{Job Status.status == failed}
                    \STATE Exception $\gets$ job failed exception
                    \STATE Resource-Header $\gets$ job-failed
                    \RETURN{Exception with Resource-Header \AND HTTP-Statuscode 404}
                \ELSE{}
                    \STATE Exception $\gets$ results not ready exception
                    \STATE Resource-Header $\gets$ results-not-ready
                    \RETURN{Exception with Resource-Header \AND HTTP-Statuscode 404}
                \ENDIF{}
            \ELSIF{Job Jobfile.processID == FloodMonitoring}
                \IF{Job Status.status == successful}
                    \IF{Job Jobfile.responseType == raw}
                        \IF{bin.tif \AND ndsi.tif are requested}
                            \STATE ZIP
                            \STATE \textbf{add} ndsi.tif \AND bin.tif \textbf{to} ZIP 
                            \RETURN{ZIP \AND HTTP-Statuscode 200}
                        \ELSIF{ndsi.tif is requested}
                            \RETURN{ndsi.tif \AND HTTP-Statuscode 200}
                        \ELSIF{bin.tif is requested}
                            \RETURN{bin.tif \AND HTTP-Statuscode 200}
                        \ENDIF{}
                    \ELSE{}
                        \IF{bin.tif \AND ndsi.tif are requested}
                            \STATE bin-Base64 $\gets$ encodeImageBase64(bin.tif)
                            \STATE ndsi-Base64 $\gets$ encodeImageBase64(ndsi.tif)
                            \STATE \textbf{embed} bin-Base64 \AND ndsi-Base64 \textbf{into} Result-Document
                            \STATE \textbf{add} Download-Links \textbf{to} Result-Document
                            \RETURN{Result-Document \AND HTTP-Statuscode 200}
                        \ELSIF{ndsi.tif is requested}
                            \STATE ndsi-Base64 $\gets$ encodeImageBase64(ndsi.tif)
                            \STATE \textbf{embed} ndsi-Base64 \textbf{into} Result-Document
                            \STATE \textbf{add} Download-Link \textbf{to} Result-Document
                            \RETURN{Result-Document \AND HTTP-Statuscode 200}
                        \ELSIF{bin.tif is requested}
                            \STATE bin-Base64 $\gets$ encodeImageBase64(bin.tif)
                            \STATE \textbf{embed} bin-Base64 \textbf{into} Result-Document
                            \STATE \textbf{add} Download-Links \textbf{to} Result-Document
                            \RETURN{Result-Document \AND HTTP-Statuscode 200}
                        \ENDIF{}
                    \ENDIF{}
                \ELSIF{Job Status.status == failed}
                    \STATE Exception $\gets$ job failed exception
                    \STATE Resource-Header $\gets$ job-failed
                    \RETURN{Exception with Resource-Header \AND HTTP-Statuscode 404}
                \ELSE{}
                    \STATE Exception $\gets$ results not ready exception
                    \STATE Resource-Header $\gets$ results-not-ready
                    \RETURN{Exception with Resource-Header \AND HTTP-Statuscode 404}
                \ENDIF{}
            \ENDIF{}
        \ELSE{}
            \STATE Exception $\gets$ No such job exception
            \STATE Resource-Header $\gets$ no-such-job
            \RETURN{Exception with Resource-Header \AND HTTP-Statuscode 404}
        \ENDIF{}
    \end{ALC@g}
    \STATE \hskip0.5em \textbf{exept}
    \begin{ALC@g}
        \RETURN{HTTP-Statuscode 500}
    \end{ALC@g}
    \ENDIF{}
\end{algorithmic}
\end{breakablealgorithm}

\newpage
\subsection{Ablauf eines Requests an den Coverage Endpoint}
\begin{breakablealgorithm}
\scriptsize
\caption{Ablauf eines Requests an den Coverage Endpoint}\label{PseudocodeCoverage}
\begin{algorithmic}  
\STATE \textbf{Input:} Request
\IF{Request.HTTP-Methode != GET}
    \RETURN{HTTP-Statuscode 405}
\ELSE{}
    \STATE \hskip0.5em \textbf{try}
    \begin{ALC@g}
        \STATE KML $\gets$ KML-Files in data/coverage/
        \STATE Coverages $\gets$ []
		\STATE BBoxes $\gets$ []
        \FOR{File \textbf{in} KML}
                \STATE KML $\gets$ open(File)
                \STATE Product $\gets$ {KML.Name, KML.BBox, KML.Date}
                \STATE BBox $\gets$ KML.BBox
                \STATE Coverages \textbf{append} Dataset
                \STATE BBoxes \textbf{append} BBox as .geojson
        \ENDFOR{}
        \IF{Request.f == text/html \OR Request.f == None}
            \STATE Response $\gets$ render\_template(templates/html/coverage.html, Coverages, BBoxes)
            \STATE Link-Header $\gets$ http://HOST:PORT/coverage?f=text/html
            \STATE Resource-Header $\gets$ coverage
            \RETURN{Response with Link- and Resource-Header \AND HTTP-Statuscode 200}
        \ELSIF{Request.f == application/json}
            \STATE Response $\gets$ jsonify(Coverages)
            \STATE Link-Header $\gets$ http://HOST:PORT/coverage?f=application/json
            \STATE Resource-Header $\gets$ coverage
            \RETURN{Response with Link- and Resource-Header \AND HTTP-Statuscode 200}
        \ELSE{}
            \RETURN{HTTP-Statuscode 406}
        \ENDIF{}
    \end{ALC@g}
    \STATE \hskip0.5em \textbf{exept}
    \begin{ALC@g}
        \RETURN{HTTP-Statuscode 500}
    \end{ALC@g}
\ENDIF{}   
\end{algorithmic}
\end{breakablealgorithm}

\subsection{Ablauf eines Requests an den Download Endpoint}
\begin{breakablealgorithm}
\scriptsize
\caption{Ablauf eines Requests an den Download Endpoint}\label{PseudocodeDownload}
\begin{algorithmic}   
    \STATE \textbf{Input:} Request, Job-ID, requested File
        \IF{Request.HTTP-Methode != GET}
        \RETURN{HTTP-Statuscode 405}
    \ELSE{}
        \STATE \hskip0.5em \textbf{try}
        \begin{ALC@g}
            \IF{jobs/<Job-ID>/ \textbf{exists}}
                \STATE Status-JSON $\gets$ open(jobs/<Job-ID>/status.json)
                \STATE Job-JSON $\gets$ open(jobs/<Job-ID>/status.json)
                    \IF{Job-JSON.processID == FloodMonitoring}
                        \IF{<requested File> == bin}
                            \RETURN{bin.tif \AND HTTP-Statuscode 200}
                        \ELSIF{<requested File> == ndsi}
                            \RETURN{ndsi.tif \AND HTTP-Statuscode 200}
                        \ELSE{}
                            \STATE Exception $\gets$ No such file exception
                            \STATE Resource-Header $\gets$ no-such-file
                            \RETURN{Exception with Resource-Header \AND HTTP-Statuscode 404}
                        \ENDIF{}
                    \ELSE{}
                        \RETURN{HTTP-Statuscode 501}
                    \ENDIF{}
                \ELSE{}
                \STATE Exception $\gets$ No such job exception
                \STATE Resource-Header $\gets$ no-such-job
                \RETURN{Exception with Resource-Header \AND HTTP-Statuscode 404}
            \ENDIF{}
        \end{ALC@g}
        \STATE \hskip0.5em \textbf{exept}
        \begin{ALC@g}
            \RETURN{HTTP-Statuscode 500}
        \end{ALC@g}
    \ENDIF{}
\end{algorithmic}
\end{breakablealgorithm}

\subsection{Echo Prozess}
\begin{breakablealgorithm}
\scriptsize
\caption{Ablauf eines Echo Prozesses}\label{PseudocodeEcho}
\begin{algorithmic}   
    \STATE \textbf{Input:} Job
    \STATE \hskip0.5em \textbf{try}
    \begin{ALC@g}
        \STATE \textbf{Break} \textbf{If} Job.Status == dismissed 
        \STATE Job.Started = now()
        \STATE \textbf{Wait} 5 Seconds
        \STATE \textbf{Break} \textbf{If} Job.Status == dismissed 
        \STATE \textbf{Create} result.json
        \STATE Job.Finished = now()
    \end{ALC@g}
    \STATE \hskip0.5em \textbf{exept}
    \begin{ALC@g}
        \STATE Job.Status = failed
    \end{ALC@g}
\end{algorithmic}

\newpage
\end{breakablealgorithm}
\subsection{Überschwemmungmonitoring Prozess}
\begin{breakablealgorithm}
\scriptsize
\caption{Ablauf eines Überschwemmungmonitoring Prozesses}\label{PseudocodeFlood}
\begin{algorithmic}   
    \STATE \textbf{Input:} Job
    \STATE \hskip0.5em \textbf{try}
    \begin{ALC@g}
        \STATE Job.Started = now()
        \STATE \textbf{break} \textbf{if} Job.Status == dismissed 
        \STATE \emph{\#Setup Processing}
        \STATE \textbf{create} footprint.geojson \textbf{from} Job.Input.BBOX
        \STATE Pre-Timeframe $\gets$ \textbf{from} Job.Input.preDate
        \STATE Post-Timframe $\gets$ \textbf{from} Job.Input.postDate
        \STATE \textbf{break} \textbf{if} Job.Status == dismissed 
        \STATE API $\gets$ Login at Copernicus Open Access Hub \textbf{with} Job.Input.Credentials
        \STATE \textbf{break} \textbf{if} Job.Status == dismissed 
        \STATE \emph{\#Retrieve and Calibrate Pre-Product}
        \STATE Pre-Product $\gets$ getProduct(API, Pre-Timeframe, Job.Input.BBOX)
        \STATE \textbf{break} \textbf{if} no matching Product could be found
        \IF{Pre-Dataset \NOT \textbf{in} Datastore}
            \STATE Pre-Dataset $\gets$ retrieveProduct(API, Pre-Product)
            \STATE \textbf{break} \textbf{If} Pre-Product \textbf{in} Long Term Archive
        \ENDIF{}
        \STATE \textbf{apply} Orbit-File to Pre-Dataset
        \STATE \textbf{clip} Pre-Dataset with footprint.geojson
        \STATE \textbf{remove} thermal Noise from Pre-Dataset
        \STATE \textbf{convert} VV Band \textbf{to} $\sigma_0$ Values
        \STATE \textbf{perform} Speckle-Filtering \textbf{with} 5x5 Lee-Sigma-Filter 
        \STATE \textbf{perform} Terrain-Correction with Ellipsoid
        \STATE \textbf{store} calibrated Pre-Dataset \textbf{as} GeoTIFF
        \STATE \textbf{break} \textbf{if} Job.Status == dismissed 
        \STATE \emph{\#Retrieve and Calibrate Post-Product}
        \STATE Post-Product $\gets$ getProduct(API, Post-Timeframe, Job.Input.BBOX)
        \STATE \textbf{break} \textbf{if} no matching Product could be found
        \IF{Post-Product \NOT \textbf{in} Datastore}
            \STATE Post-Dataset $\gets$ retrieveProduct(API, Post-Product)
            \STATE \textbf{break} \textbf{if} Post-Product \textbf{in} Long Term Archive
        \ENDIF{}
        \STATE \textbf{apply} Orbit-File \textbf{to} Post-Dataset
        \STATE \textbf{clip} Post-Dataset \textbf{with} footprint.geojson
        \STATE \textbf{remove} thermal Noise \textbf{from} Post-Dataset
        \STATE \textbf{convert} VV Band \textbf{to} $\sigma_0$ Values
        \STATE \textbf{perform} Speckle-Filtering \textbf{with} 5x5 Lee-Sigma-Filter 
        \STATE \textbf{perform} Terrain-Correction \textbf{with} Ellipsoid
        \STATE \textbf{store} calibrated Post-Dataset \textbf{as} GeoTIFF
        \STATE \textbf{break} \textbf{if} Job.Status == dismissed 
        \STATE \emph{\#Calculate and Threshold NDSI}
        \STATE NDSI $\gets$ calculateNDSI(calibrated Pre- and Post-Dataset)
        \STATE \textbf{store} NDSI \textbf{as} ndsi.tif
        \STATE \textbf{clip} NDSI \textbf{with} footprint.geojson
        \STATE Threshold $\gets$ Threshold NDSI with Ostsus-Method
        \STATE Flood-Mask $\gets$ Binarize NDSI with Threshold  
        \STATE \textbf{store} Flood-Mask \textbf{as} bin.tif
        \STATE Job.Finished = now()
    \end{ALC@g}
    \STATE \hskip0.5em \textbf{except}
    \begin{ALC@g}
        \STATE Job.Status = failed
    \end{ALC@g}
\end{algorithmic}
\end{breakablealgorithm}