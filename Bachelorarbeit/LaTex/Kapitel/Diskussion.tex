\newpage
\restoregeometry
\section{Diskussion}
%Frage
%Im Rahmen dieser Arbeit sollte untersucht werden, ob sich der OGC API - Processes - Part 1: Core Standard dazu eignet 
%unter Verwendung der Programmiersprache Python APIs für Rich Data Interfaces zu entwickeln, welche das Verarbeiten von Daten des Copernicus Programmes erlauben. 
%Ein weiterer Teil der Untersuchung ist die Betrachtung der Nutzerfreundlichkeit einer solchen Anwendungen. \\
%Dazu wurde eine prototypische Anwendung in der Programmiersprache Python entwickelt. Die prototypische Anwendung ermöglicht es Nutzer*innen 
%über eine OGC API - Processes - Part 1: Core standardkonforme API Überschwemmungsmonitoring auf Basis von Daten der Sentinel-1 Mission 
%zu betreiben. \\

%Umsetzung
%Die API ist dabei weitestgehend mit dem \verb|flask|-Framework realisiert worden. Die API setzt dabei einen Großteil der im 
%OGC API - Processes - Part 1: Core Standard definierten Requirements um.
%Die Kopplung zu den Daten der Sentinel-1 Mission erfolgt mit dem \verb|sentinelsat|-Package. Dabei werden zu den Anfragen passende Datensätze aus
%dem Copernicus Open Access Hub gesucht und heruntergeladen. Alternativ erlaubt die Anwendung auch das persistente Hinterlegen von 
%Sentinel-1 Datensätzen. 
%Das geometrische und radiometrische Kalibrieren der Sentinel-1 Datensätze erfolgt mit dem Python-Wrapper \verb|snappy|. Dieser erlaubt die Nutzung der
%Funktionen der Sentinel-1 Toolbox der SNAP. 
%Das auf Basis der kalibrierten Sentinel-1 Datensätze durchgeführte Überschwemmungsmonitoring erzeugt binäre Überschwemmungsmasken aus dem NDSI.
%Die dazu nötigen bandmathematischen und statistischen Berechnungen werden mit den Python-Packages \verb|skimage| und \verb|osgeo| durchgeführt. 
%Die Untersuchung der Nutzerfreundlichkeit der Anwendung erfolgte auf Basis der von Nielsen vorgeschlagenen Heuristiken. \\

%Bewertung
Die prototypisch implementierte OGC API - Processes - Part 1: Core standardkonforme API konnte zeigen, dass die im Standard formulierten 
Ziele erreicht werden. Die API erlaubt Nutzer*innen mit wenigen, einfach zu bedienenden Endpoints Prozesse, welche Geodaten erzeugen oder 
verarbeiten, zu starten, zu überwachen und deren Ergebnisse abzurufen. Dabei werden die Architekturstile REST und HATEOAS berücksichtigt. 
Die angebotenen Ressourcen stehen in menschen- und maschinenlesbaren Formaten zur Verfügung und enthalten Verknüpfungen zu anderen 
Ressourcen.
Die prototypische Anwendung konnte zeigen, dass sich das \verb|flask|-Framework gut dazu eignet, auf wenige Funktionen beschränkte APIs zu entwickeln. 
Endpoints und die mit ihnen verknüpften Funktionen lassen sich mit geringem Aufwand implementieren. Dabei können die bereitgestellten Ressourcen 
auch dynamische Inhalte enthalten.
Das Copernicus Open Access Hub ist als Datenquelle zu Daten des Copernicus-Programmes geeignet. Das \verb|sentinelsat|-Package ermöglicht einfaches 
Auffinden von geeigneten Datensätzen. Sofern diese sich nicht im Langzeitarchiv befinden, können diese direkt heruntergeladen werden. Die 
Limitierungen des Langzeitarchivs können durch das persistente Speichern von Datensätzen umgangen werden. Die räumliche und 
zeitliche Abdeckung wird dann jedoch durch den zur Verfügung stehenden Speicherplatz beschränkt. 
Die SNAP Plattform und der Python-Wrapper \verb|snappy| ermöglichen eine vollständige Kalibrierung von Sentinel-1 Datensätzen. 
Die nötige Installation und Konfiguration der Plattform wirken sich allerdings 
auch nachteilig auf die Eigenschaften der prototypischen Anwendung aus. So wird zum Beispiel die Python Version eingeschränkt. Auch gestaltet sich 
etwaige Containerisierung aufwendig. Zu bemerken ist jedoch, dass für das Kalibrieren von Sentinel-1 Datensätzen kaum alternative Lösungen oder 
Python-Packages zur Verfügung stehen. Allerdings stellen manche DIAS Plattformen bereits kalibrierte Datensätze zur Verfügung. Die Verwendung  
dieser spart zwar die Installation und Konfiguration von SNAP, nimmt Entwicklern allerdings die Möglichkeit, Einfluss auf die Kalibrierungen zu nehmen.
Das implementierte Verfahren zum Überschwemmungsmonitoring auf Basis der Daten der Sentinel-1 Mission erlaubt ein Detektieren von Überschwemmungen. 
Die Berechnung des NDSI und dessen Binärisierung anhand eines Schwellwertes kann leicht mit den Python-Packages \verb|skimage| und \verb|osgeo| 
implementiert werden. Allerdings schwankt die Qualität der Ergebnisse stark. So liefert das gewählte Schwellwertverfahren die verlässlichsten Schwellwerte, 
wenn der NDSI eine stark bimodale Verteilung aufweist. Ist diese nur schwach oder gar nicht vorhanden, kann es zu einer wenig brauchbaren Binärisierung kommen. 
Die Ausprägung der Bimodalität hängt dabei vom gewählten Raumausschnitt sowie dem Verhältnis von überfluteten zu trockenliegenden Flächen in diesem ab. 
Da die Anwendung die Ausgabe von NDSI und binärer Überschwemmungsmaske erlaubt, können Nutzer*innen sowohl analysebereite als auch interpretationsfähige Daten 
beziehen.  \\

Die Evaluation der Nutzbarkeit der prototypischen Anwendung offenbarte, dass diese für Anwendungen erstrebenswerte Eigenschaften aufweist. Zum einen ist die 
Anwendung detailliert dokumentiert. Zum anderen werden Nutzer*innen bereits in vielfältiger Weise über den Stand seiner gestarteten Prozesse informiert. 
Ein wichtiger Aspekt computergestützter Verfahren ist ihre Reproduzierbarkeit. Das Committee on Reproducibility and Replicability in Science definiert 
ein vorgestelltes computergestütztes Verfahren als reproduzierbar, wenn mit identischen Eingaben und unter gleichen Systembedingungen identische Resultate erzielt werden können.  
Dies bedeutet zunächst, dass die verwendete Software quelloffen und kostenlos zur Verfügung steht, um Interessierte in die Lage zu versetzen, das vorgestellte Verfahren
selbst durchzuführen. Um dies so einfach wie möglich zu machen, sollte die verwendete Software detailliert dokumentiert sein. Ein besonderes Augenmerk sollte dabei auf den 
zugrundeliegenden Daten, den verwendeten Methoden und der ursprünglichen Systemumgebung liegen \cite{reproducibility}.
Die API der prototypischen Implementierung ist durch die Dokumente der OGC und die API-Definition gut und ausführlich dokumentiert. Diese gewährleisten zusätzlich das Nutzer*innen 
die Bedienung der API schnell erlernen können. 
Die zum Überschwemmungsmonitoring verwendeten Methoden in dieser Arbeit und den im Rahmen dieser Arbeit zurate gezogenen Arbeiten ausführlich beschrieben. Sie stehen jedoch nicht 
gesammelter und aufbereiteter Form  zur Verfügung. Die verwendete Systemumgebung kann aus der 
der Anwendung beigefügten \verb|environment.yaml| nachvollzogen werden. \\

%Beschränkungen
Da die prototypisch implementierte API den OGC API - Processes - Part 1: Core Standard nicht vollständig umsetzt, konnten nicht alle beschriebenen 
Funktionen getestet werden. Dazu zählen die Callback Funktionalität, die Ausgabe von Ergebnissen im Response-Typ \verb|reference| und ein 
vollständiger Test-Suit.
Im Rahmen dieser Arbeit wurden nur Daten aus dem Copernicus Open Access Hub bezogen. Zur Struktur und Qualität der von den DIAS Plattformen 
bereitgestellten Daten kann keine Aussage gemacht werden. 
Da die Evaluation der Anwendung auf heuristischer Basis erfolgte, sind die vorgestellten Ergebnisse subjektiv und vom Evaluierenden abhängig. Außerdem 
wurden nur Nutzbarkeitsaspekte betrachtet. Eine Evaluierung von technischen Aspekten ist nicht erfolgt. 
Ein Vergleich mit anderen Implementierungen welche zum Beispiel andere Frameworks verwenden oder in anderen Programmiersprachen verfasst sind ist aus 
Gründen des Umfangs ebenfalls nicht Teil dieser Arbeit. 

%Mit Hinblick auf die Beschränkungen der prototypischen Anwendung scheint es sinnvoll andere Programmiersprachen wie R und Java zu erproben da auch für diese 
%eine Vielzahl von Packages und Bibliotheken zur Verfügung stehen um Webanwendungen und Prozesse zu entwickeln. 
%Darüber hinaus scheint die Erprobung der DIAS Plattformen als Datenquelle für Copernicus Daten sinnvoll.
%Um die Verlässlichkeit der Ergebnisse des Überschwemmungsmonitorings zu verbessern, sollten andere, stabilere Verfahren zur Extraktion von 
%Überschwemmungsmasken gefunden und erprobt werden.  
%Um die Evaluation der Anwendung zu verbessern, sollten in Ergänzung zu Heuristiken auch vergleichbare Metriken gefunden und angewandt werden. 
%Neben der Evaluation der Nutzbarkeit sollten auch technische Aspekte wie Skalierbarkeit, Wartbarkeit und Erweiterbarkeit untersucht werden. 


