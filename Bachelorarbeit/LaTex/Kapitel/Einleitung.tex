\newpage
\restoregeometry
\section{Einleitung}
\subsection{Motivation}
?Wofür werden Geodaten über das Web gebraucht?
Geodaten werden von einer breiten Nutzerschicht für die unterschiedlichsten Anwendungsfälle benötigt.
Zu diesen Anwendungsfällen gehören unter anderem der Umwelt- und Katastrophenschutz. Dort werden flächendeckende,
verlässliche und aktuelle Geodaten benötigt. Diese müssen schnell und einfach beschafft und verarbeitet werden können
um zum Beispiel rechtzeitig Maßnahmen ergreifen oder mögliche Schäden quantifizieren. 
Flächendeckende Fernerkundungsdaten von diversen Plattformen werden vom europäischen Copernicus Programm 
erfasst und im Internet bereitgestellt und können über Webseiten und APIs beschafft werden. 
Manche Daten, wie die Radardaten der Sentinel-1 Mission, bedürfen jedoch einer komplexen Vorverarbeitung bevor aus ihnen verlässliche Aussagen über die in den 
abgebildeten Räume stattfindenden Phänomene gemacht werden können. 
Diese Vorverarbeitungen können nicht von allen Nutzern selbst durchgeführt werden. Zum einen Verfügen nicht alle Nutzer über die 
nötigen Fachkenntnisse und zum anderen stehen nicht jedem Nutzer entsprechende technischen Infrastrukturen zur Verfügung. 
Um dieses Problem zu lösen und Nutzer mit einsatzbereiten Geodaten oder abgeleiteten Produkten zu versorgen können sogenannte Rich Data Interfaces zum Einsatz kommen.  
Diese würden Nutzer in die Lage versetzen vorverarbeitete oder abgeleitete Daten direkt über eine API beziehen zu können.  
Um die Interoperabilität sicherzustellen sollte diese nach den Maßgaben eines Standards implementiert werden. OGC API - Processes - Part 1: Core ist ein 
Standard ein solcher Standard. 

\subsection{Ziele}
Ziel dieser Arbeit ist Beantwortung der Frage ob sich der OGC API - Processes - Part 1: Core Standards als Grundlage zur Implementierung von Rich Data Interfaces für die 
Daten des Copernicus Programmes eignet.  
Zur Klärung dieser Frage soll eine prototypische Web-Anwendung implementiert werden. 
Teil dieser Anwendung soll eine API sein welche die Vorgaben des OGC API - Processes - Part 1: Core Standards umsetzt. 
Um eine Bezug zu einem realen Anwendungsszenario zu schaffen sollen Daten welche für das Überschwemmungsmonitoring eignen als Ressourcen über die API abfragbar sein. 
Die zur Verfügung gestellten Ressourcen sollen aus Daten der Sentinel-1 Satellitenmission abgeleitet werden. Ein Test Suit welcher die Konformität 
der API zum zugrundeliegenden Standard teilweise überprüft ist soll ebenfalls Teil der Implementierung sein.
Schlussendlich soll eine Evaluation der prototypischen Implementierung erfolgen. Dazu sollen die Nutzbarkeitsheuristiken nach Jakob verwendet werden welche 
die Nutzerfreundlichkeit und Anwendbarkeit in den Fokus ihrer Betrachtung rücken.

\subsection{Aufbau} 
Im ersten Teil dieser Arbeit sollen die nötigen fachlichen Grundlagen kurz erläutert werden. 
Im Implementierungsteil wird die prototypische Implementierung des Rich Data Interface für Copernicus-Daten vorgestellt. 
Anschließend erfolgt die Evaluation.
Der Diskussionsteil soll Raum für die Erörterung der Forschungsfrage bieten. 
Im Ausblick soll kurz umrissen werden wie eine hier prototypisch umgesetzte Anwendung weiterentwickelt werden könnte. 
Als Abschluss der Arbeit soll ein kurzes Fazit dienen welches die gewonnenen Erkenntnisse zusammenfasst. 


