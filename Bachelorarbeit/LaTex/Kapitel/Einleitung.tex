\newpage
\restoregeometry
\section{Einleitung}
\subsection{Motivation}
Diverse fernerkundungs Programme und Plattformen erfassen täglich eine große Menge 
von Fernerkundungsdaten in unterschiedlichsten Formaten. Die erfassten Daten finden 
in vielen Unterschiedlichen Szenarien anwendung. Eines dieser Szenarien ist das Kartieren 
von Flutkatastrophen und die Unterstützung entsprechnder Risikoanalysen. Da Überschwemmungen 
ca. 75\% der weltweiten Naturkatastrophen ausmachen ist dieses Thema in der EU und Weltweit 
im Fokus des zuständiger Behörden. Um deren Arbeit so effizient und präzise wie möglich zu 
unterstützen werden nicht nur ein Zugang zu den erfassten Fernerkndungsdaten 
benötigt sondern auch die Möglichkeit abgeleitete Geoprodukte beziehen 
zu können. Diese abgeleiteten Geoprodukte sind bereits vorprozessiert und auf spezielle Anwedungsfälle,
zum Beispiel die Überschwemmungskartierung, zugeschnitten. Für die Entscheidungsträger bietet dies einige 
Vorteile Zum einen müssen sie die teilweise aufwendigen und fachlich anspruchsvollen Verarbeitung der 
Geodaten nicht eigenständig durchführen. Zum anderen besteht keine Notwendigkeit auf seiter der 
Konsumenten Infrastrukturen für die Verarbeitung und Speicherung der Rohdaten bereitstellen zu müssen.
Um eine große Nitzergemeinschaft schnell und einfach mit den beschreibenen Daten versoregn zu können 
ist die Standardisierung von Formaten und Schnittstellen unabdingbar. Das Open Geospatial Consortium (OGC)
widmet dieser Aufgabe und definiert seit 1994 dieser Aufgabe. 
\subsection{Ziele}
Das Hauptziel dieser Arbeit ist die Implementierung einer OGC API - Processes - Part 1: Core standardkonformen
API. Die API wird einen Prozess anbieten welches es Nutzer Überschwemmungsmasken für ein gewünschtes Gebiet 
zu erzeugen. Die Überschwemmungsmaken sollen auf der Basis von Fernerkundungsdaten der Sentinel-1 Mission erfolgen.
Der Prozess soll dabei die Beschaffung und Vorprozessierung der Daten sowie die Berechng des Normalized Difference 
Sigma-Naught Index (NDSI) und dessen Grenzwertbestimmung und Binärisierung durchführen.  
\subsection{Aufbau}
Im ersten Teil dieser Arbeit sollen die fachlichen und technischen Grundlagen gechaffen werden. 
Dabei werden zunächst Grundlegende Konzepte der Radarfernerkundung erläutert werden. Ein besonderes
Augenmerk liegt dabei auf der sattelitengestützten Plattformen mit eines synthetischen Apetur zu denen 
die beiden Satteliten der Sentinel-1 Mission zählen. 
Anschließend sollen die Ziele sowie die Struktur das der Sentinel-1 Mission zugrunde liegenden Copernicus 
Programm der Europäischen Union vorgestellt werden. Da im Kontext dieser Arbeit nicht nur die im Rahmen dieses
Programmes erfassten Geodaten von Interesse sind sonder auch der Zugang zu diesen wird dieser Aspekt ebenfalls
näher untersucht.
Da der über die API anzubietende Prozess das Detektieren von Überflutungen auf Radardaten der Sentinel-1 Mission
ermöglichen soll im Grundlageteil auch kurz in dieses Thema eingeführt werden. Dabei werden sowohl die Arbeitscshritte
der Vorprozessierung als auch das Hereausrbeiten der Überflutung beschreiben werden.
Da sich ein großteil der Arbeit mit der Implementierung eines Standards für Application programming Interfaces (API)
beschäftigt soll der Begriff der Schnittstelle im Kontext der Geoinformatik klar definiert und und ihre Eigenschaften 
beschrieben werden. Da der zu implmentierende Stadard ein Entwicklng des OGC ist wird diese Organisation ebenfalls 
kurz vorgestellt. Darauf aufbauend wird der OGC API - Processes - Part 1: Core Standard vorgestellt. Im Fokus dieser 
Beschreibung liegt das Ziel welches das OGC mit dieser Standardisierung verfolgt sowie der Aufbau von OGC Standards Im
Allgemeinen und dem genannten Standard im Besonderen. 
Abschließend werden die Evaluationkriterien mit welchen in dieser ARbeit die zu implementierende API bewertet werden 
soll vorgestellt. 

