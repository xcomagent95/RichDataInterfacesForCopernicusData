\newpage
\restoregeometry
\section{Einleitung}
\subsection{Motivation}
?Wofür werden Geodaten über das Web gebraucht?
Geodaten werden von einer breiten Nutzerschicht für die unterschiedlichsten Anwendungsfälle benötigt.
Zu diesen Anwendungsfällen gehören unter anderem der Umwelt- und Katastropehnschutz. Dort werden flächendeckende,
verlässliche und aktuelle Geodaten benötigt. Diese müssen schnell und einfach beschaft und verarbeitet werden können
um zum Beispiel rechtzeitig Maßnahmen ergreifen oder mögliche Schäden quantifizieren zu können. 

?Wo ist das Problem bei complexen oder rohen Geodaten?
Flächendecknde Fernerkundungsdaten von diversen Plattformen werden vom europäischen Copernicus Programm 
erfasst und im Internet bereitgestellt und können über Webseiten und APIs beschafft werden. 
Manche Daten, wie die Radardaten der Sentinel-1 Mission, bedürfen jedoch einer komplexen Vorverarbeitung befor aus ihnen verlässliche Aussagen über die in den Abgebildeten Räume 
stattfindenden Phänomene gemacht werden können. Die Vorverarbeitungen können nicht von allen Nutzern selbst durchgeführt werden. Zum einen Verfügen nicht alle Nutzer über die 
nötigen Fachkenntnisse und zum anderen stehen nicht jedem Nutzer entsprechende techniche Infrastrukturen zur Verfügung. 

?Daten Beschaffung über API??Rich Data Interfaces als Lösung?
Um dieses Probel zu lösen und Nutzer mit einsatzbereiten Geodaten oder abgeleiteten Produkten zu versorgen können sogenannte Rich Data Interfaces zum Einsatz kommen.  

?In Standardisierter Weise?
Um die Interoprabilität einer solchen Anwendung sicherzsutellen sollte diese nach den Maßgabene eines Standards implementiert werden. OGC API - Processes - Part 1: Core ist ein 
Standard nach dem APIs implementiert werden können welche als Ressource Prozesse bereitstellen. 

\subsection{Ziele}
?Eignet sich der Standard für Rich Data Interfaces?
Ziel dieser Arbeit ist die Erprobung des OGC API - Processes - Part 1: Core Standards als Grundlage zur Implementierung von Rich Data Interfaces für die 
Daten des Copernicus Programmes. 
-> Prototypische Implementierung
-> Mit Geodaten ser S-1 Mission
-> Evaluation des Rich Data Interface
\subsection{Aufbau} 
-> Grundlagen zum schaffen der Grundkenntnisse
-> Implementierung
-> Evaluation durch Heuristiken und Test Suit 
-> Diskussion
-> Ausblick
-> Fazit


