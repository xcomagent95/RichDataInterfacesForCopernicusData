\newpage
\restoregeometry
\section{Einleitung}
\subsection{Motivation}
Geodaten werden für die unterschiedlichsten Anwendungsfälle benötigt.
Zu diesen Anwendungsfällen gehören unter anderem der Umwelt- und Katastrophenschutz. Dort werden flächendeckende,
verlässliche und aktuelle Geodaten benötigt. Diese müssen schnell und einfach beschafft und verarbeitet werden können,
um rechtzeitig Maßnahmen zu ergreifen oder mögliche Schäden zu quantifizieren. 
Flächendeckende Fernerkundungsdaten von diversen Plattformen werden vom europäischen Copernicus-Programm 
erfasst und im Internet bereitgestellt und können über Webseiten und APIs beschafft werden. 
Manche Daten, wie die Radardaten der Sentinel-1 Mission, bedürfen jedoch einer komplexen Vorverarbeitung bevor aus ihnen verlässliche Aussagen über die in den 
abgebildeten Räumen aufgezeichneten Phänomene getätigt oder andere Daten abgeleitet werden können \cite{radiometric_calibration_of_S1_level1_products}. 
Diese Vorverarbeitungen können nicht von allen Nutzer*innen selbst durchgeführt werden. Zum einen verfügt nicht jeder über die 
nötigen Fachkenntnisse und zum anderen stehen nicht jedem die entsprechenden technischen Infrastrukturen zur Verfügung. 
Um dieses Problem zu lösen und Nutzer*innen mit analysebereite oder interpretationsfähige Daten zu versorgen, 
können sogenannte Rich Data Interfaces zum Einsatz kommen.  
Diese versetzen Nutzer*innen in die Lage, analysebereite oder interpretationsfähige Daten direkt über eine API beziehen zu können. 
Um die Interoperabilität sicherzustellen, sollte diese nach den Maßgaben eines Standards wie dem OGC API - Processes - Part 1: Core implementiert werden. 
Beziehen Nutzer*innen analysebereite oder interpretationsfähige Daten in standardisierter Weise bietet dies eine Reihe von Vorteilen.
So ist sichergestellt das die Daten in gleicher Weise und in möglichst hoher Qualität vorprozessiert wurden und so die Interoperabilität sichergestellt ist. 
Darüber hinaus werden Analysen und Ergebnisse leichter vergleichbar da die ihnen zugrunde liegenden Daten den selben Ansprüchen genügen \cite{testbed_16}. 

\subsection{Ziele}
Im Rahmen dieser Arbeit soll untersucht werden, ob sich der OGC API - Processes - Part 1: Core Standards als Grundlage zur Implementierung von Rich Data Interfaces für die 
Daten des Copernicus-Programmes eignet.  
Dazu soll eine prototypische Anwendung in der Programmiersprache Python entwickelt werden, welche eine OGC API - Processes - Part 1: Core standardkonforme API anbietet. 
Um einen Bezug zu einem realistischen Anwendungsszenario zu schaffen, soll die Anwendung Überschwemmungsmonitoring auf Basis von Sentinel-1 Daten ermöglichen. 
Ergebnisse des Überschwemmungsmonitorings sollen der NDSI, also analysebereite Daten sowie aus diesem abgeleitete Überschwemmungsmasken, also interpretationsfähige
Daten sein.
Zusätzlich sollen die Sentinel-1 Datensätze vollständig durch die Anwendung kalibriert werden.
Außerdem soll ein Konzept für den Zugriff auf Daten des Copernicus-Programmes erarbeitet und implementiert werden.
Schlussendlich wird eine heuristische Evaluation der prototypischen Implementierung durchgeführt. 
Dazu werden die Nutzbarkeitsheuristiken nach Jakob Nielsen verwendet, welche die Nutzerfreundlichkeit und Anwendbarkeit in den Fokus ihrer Betrachtung rücken.

\subsection{Aufbau} 
Im ersten Teil dieser Arbeit sollen die nötigen fachlichen Grundlagen kurz erläutert werden. Dazu zählen die Grundlagen der Radarfernerkundung mit 
Synthetic Aperture Radar Systemen (SAR), die Ziele sowie der Aufbau des Copernicus-Programmes, 
die Vorstellung eines möglichen Verfahrens zum radargestütztem Überschwemmungsmonitoring und eine kurze Vorstellung des OGC sowie dem 
OGC API - Processes - Part 1: Core Standard. 
Im Implementierungsteil wird die prototypische Implementierung des Rich Data Interface für Copernicus-Daten vorgestellt. Dabei wird der verwendete Softwarestack, 
die Grundstruktur der Anwendung sowie die Funktionsweisen der Endpoints der API und der angebotenen Prozesse beschrieben. 
Auf die prototypische Implementierung aber auch auf die durch den Standard beschränkten Möglichkeiten bezugnehmend folgt die Evaluation der Anwendung. Diese 
geschieht auf Basis der von Nielsen vorgeschlagenen Heuristiken, welche sich auf einzelne Aspekte der Nutzbarkeit von Anwendungen beziehen. 
Der Diskussionsteil schafft Raum für die Erörterung der Forschungsfrage. Im Ausblick wird kurz umrissen wie die prototypisch umgesetzte Anwendung 
sinnvoll weiterentwickelt werden kann. Als Abschluss der Arbeit soll ein kurzes Fazit dienen, welches die gewonnenen Erkenntnisse zusammenfasst. 



