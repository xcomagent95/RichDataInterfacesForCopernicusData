\newpage
\restoregeometry
\section{Einleitung}
\subsection{Motivation}
Geodaten helfen weltweit bei der Analyse unterschiedlichster Probleme. Um diese Analysen möglichst schnell und präzise durchführen zu 
können müssen Geodaten schnell und in hoher Qualität abrufbar sein. Zwar lassen sich zahlreiche Geodaten nach Bedarf aus dem Internet beziehen, 
manche Geodaten bedürfen allerdings zeit- und und rechenintensiven Vorverarbeitungen welche teilweise fundierte Kenntnisse aus dem Bereich der 
Geoinformatik voraussetzen. Zur Lösung dieses Problems könnten sogenannte Rich Data Interfaces dienen. Diese würden Nutzer in die Lage versetzen 
Geodaten abzurufen welche bereits auf einen speziellen Anwendungsfall zugeschnitten sind. 
\subsection{Ziele}
Ziel dieser Arbeit ist die Implementierung und Evaluation eines leichtgewichtigen, OGC API - Processes - Part: 1 Core Standard konformen Application Programming Interface. 
Die Implementierung soll dabei die Eigenschaften eines Rich Data Interface für Copernicus-Daten aufweisen. Dieses soll Rohdaten der Sentinel-1 Mission 
auf Nutzeranfrage hin beschaffen, vorverarbeiten und als Endprodukt Daten liefern welche sich für das Überschwemmungsmonitoring eignen. 
Es soll untersucht werden inwieweit sich der OGC API - Processes - Part: 1 Core Standard dazu eignet Rich Data Interfaces für Copernicus Daten zu entwerfen. 
\subsection{Aufbau}


