\newpage
\restoregeometry
\section{Einleitung}
\subsection{Motivation}
?Wofür werden Geodaten über das Web gebraucht?
Geodaten werden von einer breiten Nutzerschicht für die unterschiedlichsten Anwendungsfälle benötigt.
Zu diesen Anwendungsfällen gehören unter anderem der Umwelt- und Katastropehnschutz. Dort werden flächendeckende,
verlässliche und aktuelle Geodaten benötigt. Diese müssen schnell und einfach beschaft und verarbeitet werden können
um zum Beispiel rechtzeitig Maßnahmen ergreifen oder mögliche Schäden quantifizieren. 

?Wo ist das Problem bei complexen oder rohen Geodaten?
Flächendecknde Fernerkundungsdaten von diversen Plattformen werden vom europäischen Copernicus Programm 
erfasst und im Internet bereitgestellt und können über Webseiten und APIs beschafft werden. 
Manche Daten, wie die Radardaten der Sentinel-1 Mission, bedürfen jedoch einer komplexen Vorverarbeitung befor aus ihnen verlässliche Aussagen über die in den Abgebildeten Räume 
stattfindenden Phänomene gemacht werden können. Die Vorverarbeitungen können nicht von allen Nutzern selbst durchgeführt werden. Zum einen Verfügen nicht alle Nutzer über die 
nötigen Fachkenntnisse und zum anderen stehen nicht jedem Nutzer entsprechende techniche Infrastrukturen zur Verfügung. 

?Daten Beschaffung über API??Rich Data Interfaces als Lösung?
Um dieses Probel zu lösen und Nutzer mit einsatzbereiten Geodaten oder abgeleiteten Produkten zu versorgen können sogenannte Rich Data Interfaces zum Einsatz kommen.  

?In Standardisierter Weise?
Um die Interoprabilität einer solchen Anwendung sicherzsutellen sollte diese nach den Maßgabene eines Standards implementiert werden. OGC API - Processes - Part 1: Core ist ein 
Standard nach dem APIs implementiert werden können welche als Ressource Prozesse bereitstellen. 

\subsection{Ziele}
Ziel dieser Arbeit ist beantwortung der Frage ob sich des OGC API - Processes - Part 1: Core Standards als Grundlage zur Implementierung von Rich Data Interfaces für die 
Daten des Copernicus Programmes eignet.  
Zur Klärung dieser Frage soll eine prototypische Anwendung implementiert werden. Teil dieser Anwendung soll eine API sein welche die Vorgaben des OGC API - Processes - Part 1: Core Standards 
umsetzt. Um eine Bezug zu einem realen Anwendungsszenario zu schaffen sollen Überschwemmungsmasken und der Sigam-Diffence-Sigma-Naught Index als Ressourcen über die API abfragbar sein. 
Die zur Verfügung gestellten ressourcen sollen aus Daten der Sentinel-1 Sattelitenmission abgeleitet werden.  
Zusätzlich soll die prototypische Anwendung evaluiert werden.
\subsection{Aufbau} 
-> Grundlagen zum schaffen der Grundkenntnisse
Im Implementierungsteil wird die prototypische Implementierung des Rich Data Interface für Copernicus-Daten vorgestellt. Dabei wird zunächst die verwendete Software, die Programmstruktur sowie 
grundlegende Eingenschaften der angebotenen Ressourcen erläutert. Anschließend werden die Implementierungen der umgesetzten Requirements-Classes des OGC API - Processes - Part 1: Core Standard 
sowie zusätzliche Funktionen beschrieben.
beschrieben. 
-> Evaluation durch Heuristiken und Test Suit 
-> Diskussion
Im Ausblick soll kurz umrissen werden wie eine hier prototypisch umgesetzte Anwendung weiterentwickelt werden könnte. 
Als Abschluss der Arbeit soll ein kurzes Fazit dienen welches die gewonnenen Erkenntnisse zusammenfasst. 


