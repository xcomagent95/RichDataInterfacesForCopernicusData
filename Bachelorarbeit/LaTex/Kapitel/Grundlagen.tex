\newpage
\restoregeometry
\section{Grundlagen}
\subsection{Radarfernerkundung}
Bei der Radarfernerkundung werden vom Radarsystem in regelmäßigen Abständen elektromagnetische Signale ausgesandt. Nach dem Senden eines Signals 
(Chrip) folgt ein Zeitfenster indem die Plattform auf Echos des ausgesandten Signals wartet.
Trifft das ausgesandte Signal auf eine Oberfläche, zum Beispiel 
die Erdoberfläche, wird ein Bruchteil reflektiert und als Echo vom Fernerkundungssystem empfangen \cite{tutorial_on_sar}.

Die Radarfernerkundung gehört zu den aktiven Fernerkundungsmethoden da hier im Gegensatz zur optischen Fernerkundung nicht nur 
von Oberflächen reflektierte Strahlung von anderen Strahlungsquellen wie der Sonne aufgenommen wird, sondern das Fernerkundungssystem 
selbst als Strahlungsquelle dient. Messungen können daher tageszeitunabhängig erfolgen. Dabei können die sendenden und erfangenden Komponenten 
unterschiedlich (bi- oder multistatisches Radar) oder identisch sein (monostatitsches Radar). Bildgebende Radarsysteme werden auf mobilen Plattformen 
montiert und blicken seitlich auf die zu beobachtende Oberfläche. Die Flugrichtung wird Azimut und die Blickrichtung als Slant Range 
bezeichnet \cite{tutorial_on_sar} (Abbildung 1). 

Die Eigenschaften des reflektierten Signals hängen sowohl von Parametern des Aufnahmesystems als von Parametern der reflektierenden Oberfläche ab.
So werden in der Radarfernerkundung verschiedenen Frequenzbänder verwendet, welche sich in Frequenz und Wellenlänge unterscheiden. Da sich Wechselwirkung zischen Signalen 
unterschiedlicher Frequenzbänder unterscheidet können so unterschiedliche Aspekte der beobachteten Oberfläche hervorgehoben werden. Dabei kommen in der Regel
Wellenlängen von 0.75m bis 120m zu Einsatz (Tabelle 1).
Mit einer größeren Wellenlänge kann ein Medium auch tiefer durchdrungen werden. Außerdem werden Wolken, Dunst und Rauch durchdrungen was den zusätzlich Vorteil bietet
wetterunabhängig Messungen durchführen zu können \cite{einfuehrung_in_fernerkundung}.

\begin{table}[H]
    \centering
    \caption{Gängige Frequenz-Bänder in der Radarfernerkundung \cite{tutorial_on_sar}}
\begin{center}
    \begin{tabular}{|c|c|c|c|c|c|c|c|} 
        Frequenzband & Ka & Ku & X & C & S & L & P\\ 
        \hline
        Frequenz (GHz) & 40-25 & 17.6-12 & 12-7.5 & 7.5-3.75 & 3.75-2 & 2-1 & 0.5-0.25\\ 
        Wellenlänge (cm) & 0.75–1.2 & 1.7–2.5 & 2.5–4 & 4–8 & 8–15 & 15–30 & 60–120\\ 
    \end{tabular}
    \label{table:1}
\end{center}
\end{table}

Die Durchdringungstiefe hängt auch von der Dielektrizitätskonstante der reflektierenden Oberfläche ab. Ist diese groß, reflektiert die Oberfläche stark und die 
Durchdringungstiefe ist gering. Zusätzlich ist die Polarisation der ausgesandten und empfangenen Signale bei der Messung ausschlaggebend. Sie können horizontal oder 
vertikal polarisiert sein. Dies führt zu vier möglichen Polarisationsmodi für das Senden (transcieve) und das Empfangen (receive) nämlich HH, VV, HV und VH. Auch die 
Polarisation sorgt für eine unterschiedliche Wiedergabe von beobachteten Objekten und kann somit verwendet werden um bestimmte Aspekte hervorzuheben
 \cite{einfuehrung_in_fernerkundung}. Die Auflösung entlang des Azimut unterscheidet sich von der Auflösung in Blickrichtung. Die Auflösung in Azimutrichtung wird von 
 der Antennenlänge bestimmt da diese festlegt wie lange die Reflektion eines Objektes empfangen werden. Die Antennenlänge kann bauartbedingt nicht beliebig gesteigert werden.
Die Antennenlänge bestimmt auch den Abstrahlwinkel und somit die Ausdehnung am Boden eines Impulses in Azimutrichtung. Diese nimmt mit zunehmender Entfernung
zu, während die Auflösung abnimmt.
Die Auflösung in Blickrichtung hängt vom Depressionswinkel der Antenne ab da dieser Einfluss auf die Laufzeit des Signals nimmt. Die Ausdehnung des beobachteten Gebietes 
hängt von der Laufzeit des ausgesandten Signales ab. Die Objekte werden abhängig von ihrer Entfernung zur Antenne verzerrt wiedergegeben da nahegelegende Objekte von der
Wellenfront schneller durchlaufen werden. Die Verzerrung lässt sich jedoch nahezu vollständig korrigieren \cite{einfuehrung_in_fernerkundung}. 
Die bisher beschriebene System wird auch System mit realer Apertur bezeichnet und eignet sich nur für geringe Flughöhen da hier der Abstand zwischen Antenne und Oberfläche 
gering ist.  Bei Radarsystemen mit einer synthetischen Apertur wird durch die Bewegung des Sensors in Azimutrichtung die wirksame Antennenlänge rechnerisch verlängert 
indem die reflektierten Signale eines beobachteten Objektes von verschiedenen Standpunkten und unterschiedlichen Zeitpunkten miteinander korreliert werden. So können 
hohe Azimutauflösungen erzielt werden. Solche Systeme eigenen sich auch für den Einsatz auf Satelliten \cite{einfuehrung_in_fernerkundung}. 

\begin{figure}[H]
    \centering
    \includegraphics[width=12cm]{Bilder/SAR_Prinzip.png}
    \caption{Prinzip des Synthetic Aperture Radar \cite{tutorial_on_sar}}
    \label{fig:sar_prinzip}
\end{figure}

Solche Systeme können in verschiedenen Modi operieren. Der einfachste dieser Modi ist der Strip-Map Modus bei dem nur ein Streifen aufgenommen wird. Breitere Aufnahmestreifen
können mit dem ScanSAR Modus erzielt werden. Dabei werden unter verschiedenen Depressionswinkeln, in Blickrichtung und zeitversetzt mehrere Sub-Aufnahmestreifen erzeugt. 
Im Vergleich zum Strip-Map Modus ist Auflösung jedoch geringer. 
Wird eine höhere Auflösung benötigt kann der Spotlight Modus zum Einsatz kommen bei dem eine fixe Region über einen längeren Zeitraum hinweg beobachtet wird. Dies führt zu 
einer sehr langen wirksamen Antenne und damit zu hohen Azimutauflösungen. Angepasste Modi oder Mischformen können je Beobachtungsszenario zum Einsatz kommen \cite{tutorial_on_sar}. 

\begin{figure}[H]
    \centering
    \includegraphics[width=16cm]{Bilder/SAR_Modi.png}
    \caption{Aufnahmemodi SAR \cite{tutorial_on_sar}}
    \label{fig:sar_prinzip}
\end{figure}

Die Rauigkeit ist eine Eigenschaft der reflektierenden Oberfläche und hat großen Einfluss auf das reflektierte Signal. Ist diese im Verhältnis zu verwandten
Wellenlänge gering so kommt es zur Spiegelung und nur ein geringer bis kein Anteil des kehrt zum Empfänger zurück. Doch auch die Form und Exposition der Oberfläche nimmt 
Einfluss auf das reflektierte Signal. So werden Flächen unterschiedlich stark bestrahlt. Ist eine dem System abgewandte Fläche steiler geneigt als der Depressionswinkel 
liegen Sie sogar im Radarschatten und werden gar nicht bestrahlt \cite{einfuehrung_in_fernerkundung}. Im Gegensatz zu optischen Aufnahmeverfahren liefern die Rohdaten 
einer Befliegung mit Radarsensoren noch keine Bilddaten. Um Bilder zu erzeugen, bedarf es zunächst einer komplexen Verarbeitung der aus Amplitude und Phase bestehenden. 
Signale. Dabei werden die Daten entlang des Azimuts und der Blickrichtung gefiltert. In der Regel repräsentieren die Pixelwerte eines aus Radardaten abgeleiteten Bildes 
der Reflektivität der korrespondierenden Fläche. Mittels Geocodierung kann das so entstandene Bild verortet werden. Zusätzlich können diverse Kalibrierungen vorgenommen 
werden. Dazu gehören Verfahren welche Rauscheffekte minimieren, die geometrischen Eigenschaften verbessern oder die Interpretation der Bilder erleichtern \cite{tutorial_on_sar}. 

\subsection{Copernicus Programm}
\subsubsection{Ziele}
Das Copernicus Programm ging aus dem Global Monitoring for Environmental Security Programm (GMES) Programm hervor welches 1998 mit dem Ziel initiiert wurde um Europa 
zu ermöglichen eine führende Rolle bei der Lösung von weltweiten Problemen im Kontext Umwelt und Klima zu verschaffen. Teil dieser Bestrebungen ist der Aufbau eines 
leistungsfähigen Programms zur Erdbeobachtung. 2012 wurde das GMES-Programm zum Copernicus Programm umbenannt \cite{history_of_copernicus}.
Erklärte Ziele des Copernicus Programmes ist das Überwachen der Erde um den Schutz der Umwelt sowie Bemühungen von Katastrophen- und Zivilschutzbehörden zu 
unterstützen. Gleichzeitig soll die Wirtschaft im Bereich Raumfahrt und der damit verbundenen Services unterstützt und Chancen für neue Unternehmungen geschaffen
werden \cite{copernicus_regulation}.

\subsubsection{Aufbau}
Das Copernicus Programm besteht aus Weltraum, In-Situ- und Service-Komponente. 
Zur Weltraum-Komponente gehören die verschiedenen Satellitenmissionen sowie Bodenstationen welche für den Betrieb sowie die Steuerung und Kalibrierung  der 
Satelliten verantwortlich sind \cite{copernicus_regulation}. Sentinel-1 Satelliten sind mit bildgebenden Radarsystemen ausgerüstet und beobachten wetter- und 
tageszeitunabhängig Land-, Wasser- und Eismassen um unter andrem das Krisenmanagement zu unterstützen. Satelliten der Sentinel-2 Mission führen hochauflösende, 
multispektrale Kameras mit und liefern weltweit optische Fernerkundungsdaten. Altimetrische und radiometrische Daten von Land- und Wasserflächen werden von der 
Sentinel-3 Satellitenmission gesammelt während spektrometrische Daten zur Überwachung der Luftqualität von Sentinel-4 und 5 Satelliten erfasst werden. 
Ozeanografische Daten werden von den Setinel-6 Satelliten geliefert \cite{sentinel_overview}. \\

Die In-Situ-Komponente sammelt Daten von see-, luft- und landbasierten Sensoren sowie geografische und geodätische Referenzdaten. Die harmonisierten Daten 
werden verwendet um die Daten der Weltraum-Komponente zu verifizieren oder zu korrigieren. Gleichzeitig können räumliche oder thematische Lücken in der 
Datenabdeckung gefüllt werden \cite{copernicus_regulation}\cite{what_is_copernicus}. \\

Zur Service-Komponente gehören unterschiedliche Dienste welche jeweils auf Themengebiet abgestimmt sind und Daten in hoher Qualität bereitstellen.
Der Copernicus Atmosphere Monitoring Service (CAMS) soll Informationen zur Luftqualität und der chemischen Zusammensetzung der Atmosphäre liefern. 
Daten bezüglich des Zustands und der Dynamik der Meere und deren Ökosysteme lassen sich über den Copernicus Marine Environment Monitoring Service (CMEMS) beziehen. 
Informationen zur Flächennutzung und und Bodenbedeckung werden vom Copernicus Land Monitoring Service (CLMS) bereitgestellt. 
Um eine nachhaltige Klimapolitik planen und umsetzen zu können stellt der Copernicus Climate Change Service (C3S) aktuelle sowie historische Klimadaten bereit.  
Um den Zivilschutzbehörden schnelle Reaktionen auf Umweltkatastrophen zu ermöglichen stellt der  Emergency Management Service (EMS) entsprechende Fernerkundungsdaten 
breit. Ähnliche Daten können von europäischen Zoll- und Grenzschutzbehörden über den Copernicus Security Service bezogen werden
\cite{copernicus_regulation}\cite{what_is_copernicus}.

\subsubsection{Sentinel 1}
Die Sentinel-1 Satellitenmission liefert wetter- und tageszeitunabhängige Radardaten der Erdoberfläche. Die Satelliten tragen als Hauptinstrument ein 
bildgebendes Radar mit synthetischer Apertur welches im C-Frequenzband arbeitet. Es stehen zwei Polarisationsmodi, Single (HH, VV) oder Dual (HH+HV, VV+VH),
zur Verfügung \cite{sentinel_1_definition}. 
Die Erfassung von Daten kann in vier Aufnahmemodi erfolgen welche sich in Auflösung, Streifenbreite und Anwendungsszenario unterscheiden. 
Der Standardmodus ist der Stripmap Modus (SM) bei dem Aufnahmestreifen mit einer kontinuierlichen Folge von Signalen abgetastet wird \cite{sentinel_1_definition}.
Die Aufnahmemodi Interferometric Wide Swath Mode (IW) und Extra-Wide Swath Mode (EW) arbeiten im TOPSAR Modus mit drei beziehungsweise
fünf Sub-Aufnahmestreifen um ein größeres Gebiet aber in geringerer Auflösung aufnehmen zu können. Der TOPSAR Modus ist eine Abwandlung des ScanSAR Modus bei 
dem die Antenne zusätzlich in Azimutrichtung vor und zurück bewegt wird um die Qualität der resultierenden Bilder zu verbessern. 
Wenn der Wave Modus (WV) zu Einsatz kommt werden kleine, Vignetten genannte, Szenen im Stripmap Modus aufgenommen. Sie werden in regelmäßigen Abständen und
wechselnden Depressionwinkeln aufgenommen.   

Wave Mode (WV)\\


\begin{figure}[H]
    \centering
    \includegraphics[width=16cm]{Bilder/Aquisition_Modes.png}
    \caption{Aufnahmemodi der Sentinel-1 Mission \cite{sentinel_1_overview}}
    \label{fig:aquisition_modes}
\end{figure}

\begin{table}[H]
    \centering
    \caption{Eigenschaften der Aufnahmemodi der Sentinel-1 Mission \cite{sentinel_1_overview}}
\begin{center}
    \begin{tabular}{|c|c|c|c|c|} 
        Parameter & Interferometric Wide-swath mode(IW) & Wave mode (WV) & Strip Map mode (SM) & Extra Wide-swath mode (EW) \\ 
        \hline
        Polarisation & Dual (HH+HV, VV+VH) & Single (HH, VV) & Dual (HH+HV, VV+VH) & Dual (HH+HV, VV+VH) \\ 
        Azimut-Auflösung (m) & 20 & 5 & 5 & 40 \\
        Rage-Auflösung (m) & 5 & 5 & 5 & 20 \\
        Streifenbreite (km) & 250 & Vignette 20x20 & 80 & 410\\
    \end{tabular}
    \label{table:1}
\end{center}
\end{table}

\subsubsection{Datenzugang}
\subsection{Überschwemmungsmonitoring}
\subsection{Schnittstellen}
\subsection{OGC und OGC Standards}
\subsection{OGC API - Processes - Part 1: Core}
\subsubsection{Ziele}
\subsubsection{Aufbau}
\subsection{Evaluationskriterien}

