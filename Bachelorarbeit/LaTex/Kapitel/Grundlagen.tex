\newpage
\restoregeometry
\section{Grundlagen}
\subsection{Radarfernerkundung}
Bei der Radarfernerkundung werden vom Radarsystem in regelmäßigen Abständen elektromagnetische Signale ausgesandt. Nach dem Senden eines Signals 
(Chrip) folgt ein Zeitfenster, indem die Plattform auf Echos des ausgesandten Signals wartet.
Trifft das ausgesandte Signal auf eine Oberfläche, zum Beispiel 
die Erdoberfläche, wird ein Bruchteil in Richtung Empfänger reflektiert und als Echo vom Fernerkundungssystem empfangen \cite{tutorial_on_sar}.

Die Radarfernerkundung gehört zu den aktiven Fernerkundungsmethoden da hier im Gegensatz zur optischen Fernerkundung nicht nur 
von Oberflächen reflektierte Strahlung von anderen Strahlungsquellen wie der Sonne aufgenommen wird, sondern das Fernerkundungssystem 
selbst als Strahlungsquelle dient. Messungen können daher tageszeitunabhängig erfolgen. Bildgebende Radarsysteme werden auf mobilen Plattformen 
montiert und blicken seitlich auf die zu beobachtende Oberfläche. Die Flugrichtung wird Azimut und die Blickrichtung als Slant Range 
bezeichnet \cite{tutorial_on_sar} (Abbildung 1). 

Die Eigenschaften des reflektierten Signals hängen sowohl von Parametern des Aufnahmesystems als von Parametern der reflektierenden Oberfläche ab.
So werden in der Radarfernerkundung verschiedenen Frequenzbänder verwendet, welche sich in Frequenz und Wellenlänge unterscheiden. Da sich die Wechselwirkungen zwischen Signalen 
unterschiedlicher Frequenzbänder und den reflektierenden Oberflächen unterscheidet können so unterschiedliche Aspekte der beobachteten Oberflächen hervorgehoben werden. 
Dabei kommen in der Regel Wellenlängen von 0.75m bis 120m zum Einsatz (siehe Tabelle \ref{frequenzbaender}).
Mit einer größeren Wellenlänge kann ein Medium auch tiefer durchdrungen werden. Außerdem werden Wolken, Dunst und Rauch durchdrungen was den zusätzlich Vorteil bietet
wetterunabhängig Messungen durchführen zu können \cite{einfuehrung_in_fernerkundung}.

\begin{table}[H]
    \caption{Gängige Frequenz-Bänder in der Radarfernerkundung \cite{tutorial_on_sar}}
    \centering
    \begin{tabular}{c|c c c c c c c } 
        Frequenzband & Ka & Ku & X & C & S & L & P\\ 
        \hline
        Frequenz (GHz) & 40-25 & 17.6-12 & 12-7.5 & 7.5-3.75 & 3.75-2 & 2-1 & 0.5-0.25\\ 
        Wellenlänge (cm) & 0.75–1.2 & 1.7–2.5 & 2.5–4 & 4–8 & 8–15 & 15–30 & 60–120\\ 
    \end{tabular}
    \label{frequenzbaender}
\end{table}

Die Durchdringungstiefe hängt auch von der Dielektrizitätskonstante, also der Leitfähigkeit, ab. Ist diese groß, kommt es zu starken Reflektionen und die 
Durchdringungstiefe ist gering. Die Rauigkeit ist eine Eigenschaft der reflektierenden Oberfläche und hat großen Einfluss auf das reflektierte Signal. Ist diese im Verhältnis
zur verwandten Wellenlänge gering so kommt es zu spiegelnden Reflektionen und nur ein geringer Anteil des kehrt zum Empfänger zurück. Je diffuser
die Reflektion mit zunehmender Rauigkeit wird umso größer ist der Anteil des Signals welcher zum Empfänger zurückgeworfenen Signals. Doch auch die Form und Exposition der Oberfläche 
nimmt Einfluss auf das reflektierte Signal. So werden Flächen je nach Neigung unterschiedlich stark bestrahlt. Ist eine dem System abgewandte Fläche steiler geneigt als der Depressionswinkel 
liegen Sie sogar im Radarschatten und werden gar nicht bestrahlt \cite{einfuehrung_in_fernerkundung}. 
Zusätzlich ist die Polarisation der ausgesandten und empfangenen Signale bei der Messung ausschlaggebend. Sie können horizontal oder 
vertikal polarisiert sein. Dies führt zu vier möglichen Polarisationsmodi für das Senden und das Empfangen nämlich HH, VV, HV und VH. Auch die 
Polarisation sorgt für eine unterschiedliche Wiedergabe von beobachteten Objekten und kann somit verwendet werden, um bestimmte Aspekte hervorzuheben
 \cite{einfuehrung_in_fernerkundung}. Die Auflösung entlang des Azimut unterscheidet sich von der Auflösung in Blickrichtung. Die Auflösung in Azimutrichtung wird von 
der Antennenlänge bestimmt da diese festlegt wie lange die Reflektionen eines Objektes empfangen werden. Die Antennenlänge kann bauartbedingt nicht beliebig gesteigert werden.
Die Bauart der Antenne bestimmt auch den Abstrahlwinkel $\Theta_a$ und somit die Ausdehnung am Boden eines Impulses in Azimutrichtung. Diese nimmt mit zunehmender Entfernung
zu, während die Auflösung abnimmt.
Die Auflösung in Blickrichtung hängt von der Bandbreite ab welche sich aus der Sendefrequenz und der Signaldauer. Die Ausdehnung des beobachteten Gebietes 
in Blickrichtung hängt von der Laufzeit des ausgesandten Signales ab. Die Objekte werden abhängig von ihrer Entfernung zur Antenne verzerrt wiedergegeben da nahegelegene 
Objekte von der Wellenfront schneller durchlaufen werden. Dieser Unterschied zwischen Schrägdistanz und Bodendistanz lässt sich jedoch nahezu vollständig korrigieren 
\cite{einfuehrung_in_fernerkundung}. Die bisher beschriebenen Systeme werden auch als Systeme mit realer Apertur bezeichnet und eignen sich nur für geringe Flughöhen da hier 
der Abstand zwischen Antenne und Oberfläche gering ist. Bei Radarsystemen mit einer synthetischen Apertur wird durch die Bewegung des Sensors in Azimutrichtung die 
wirksame Antennenlänge rechnerisch verlängert indem die reflektierten Signale eines beobachteten Objektes von verschiedenen Standpunkten und unterschiedlichen Zeitpunkten 
miteinander korreliert werden. So können hohe Azimutauflösungen erzielt werden. Solche Systeme eigenen sich auch für den Einsatz auf Satelliten \cite{einfuehrung_in_fernerkundung}. 

\begin{figure}[H]
    \centering
    \includegraphics[width=0.6\textwidth]{Bilder/SAR_Prinzip.png}
    \caption{Prinzip eines SAR Fernerkundungssystems \cite{tutorial_on_sar}}
    \label{sar_prinzip}
\end{figure}

Solche Systeme können in unterschiedlichen Aufnahmeverfahren arbeiten. Das einfachste dieser Verfahren ist das Stripmap Verfahren bei dem nur ein Aufnahmestreifen
kontinuierlich aufgenommen wird. Breitere Aufnahmestreifen können mit dem ScanSAR Verfahren erzielt werden. Dabei werden unter verschiedenen Depressionswinkeln, 
in Blickrichtung und zeitversetzt mehrere Subaufnahmestreifen erzeugt. Im Vergleich zum Stripmap Verfahren ist Auflösung jedoch geringer. 
Wird eine höhere Auflösung benötigt kann das Spotlight Verfahren zum Einsatz kommen, bei dem eine fixe Region über einen längeren Zeitraum hinweg beobachtet wird. Dies führt zu 
einer sehr langen wirksamen Antenne. Angepasste Verfahren oder Mischformen können je Beobachtungsszenario zum Einsatz kommen (siehe Abbildung \ref{sar_scan_modi})\cite{tutorial_on_sar}. 

\begin{figure}[H]
    \centering
    \includegraphics[width=\textwidth]{Bilder/SAR_Modi.png}
    \caption{Aufnahmeverfahren SAR Systemen \cite{tutorial_on_sar}}
    \label{sar_scan_modi}
\end{figure}

Im Gegensatz zu optischen Aufnahmeverfahren liefern die Rohdaten 
einer Befliegung mit Radarsensoren noch keine Bilddaten. Um Bilder zu erzeugen, bedarf es zunächst einer komplexen Verarbeitung der aus Amplitude und Phase bestehenden 
reflektierten Signale. Dabei werden die Daten entlang des Azimuts und der Blickrichtung gefiltert. In der Regel repräsentieren die Pixelwerte eines aus Radardaten 
abgeleiteten Bildes die Reflektivität des korrespondierenden Bodenelements. Mittels Geocodierung kann das so entstandene Bild verortet werden. Zusätzlich können diverse 
Kalibrierungen vorgenommen werden. Dazu gehören Verfahren welche Rauscheffekte minimieren, die geometrischen Eigenschaften verbessern oder die Interpretation der Bilder 
erleichtern \cite{tutorial_on_sar}. 

\subsection{Copernicus Programm}
\subsubsection{Ziele}
Das Copernicus-Programm ging aus dem Global Monitoring for Environmental Security Programm (GMES) Programm hervor welches 1998 mit dem Ziel initiiert wurde um Europa 
zu ermöglichen eine führende Rolle bei der Lösung von weltweiten Problemen im Kontext Umwelt und Klima zu verschaffen. Teil dieser Bestrebungen ist der Aufbau eines 
leistungsfähigen Programms zur Erdbeobachtung. 2012 wurde das GMES-Programm zum Copernicus-Programm umbenannt \cite{history_of_copernicus}.
Erklärte Ziele des Copernicus-Programmes ist das Überwachen der Erde um den Schutz der Umwelt sowie Bemühungen von Katastrophen- und Zivilschutzbehörden zu 
unterstützen. Gleichzeitig soll die Wirtschaft im Bereich Raumfahrt und der damit verbundenen Dienstleistungen unterstützt und Chancen für neue Unternehmungen geschaffen
werden \cite{copernicus_regulation}.

\subsubsection{Aufbau}
Das Copernicus-Programm besteht aus Weltraum, In-Situ- und Service-Komponente. 
Zur Weltraum-Komponente gehören die verschiedenen Satellitenmissionen sowie Bodenstationen welche für den Betrieb sowie die Steuerung und Kalibrierung der 
Satelliten sowie der Verarbeitung und Validierung der Daten verantwortlich sind \cite{copernicus_regulation}. \\ 
Sentinel-1 Satelliten sind mit bildgebenden Radarsystemen ausgerüstet und beobachten wetter- und tageszeitunabhängig Land-, Wasser- und Eismassen, um unter andrem das 
Krisenmanagement zu unterstützen.
Satelliten der Sentinel-2 Mission führen hochauflösende, multispektrale Kameras mit und liefern weltweit optische Fernerkundungsdaten. \\
Altimetrische und radiometrische Daten von Land- und Wasserflächen werden von der Sentinel-3 Satellitenmission gesammelt während spektrometrische Daten zur 
Überwachung der Luftqualität von Sentinel-4 und 5 Satelliten erfasst werden.
Ozeanografische Daten sollen von den Sentinel-6 Satelliten geliefert werden \cite{sentinel_overview}.

Die In-Situ-Komponente sammelt Daten von See-, luft- und landbasierten Sensoren sowie geografische und geodätische Referenzdaten. Die harmonisierten Daten 
werden verwendet, um die Daten der Weltraum-Komponente zu verifizieren oder zu korrigieren. Gleichzeitig können räumliche oder thematische Lücken in der 
Datenabdeckung gefüllt werden \cite{copernicus_regulation}\cite{what_is_copernicus}. \\

Zur Service-Komponente gehören unterschiedliche Dienste, welche jeweils auf Themengebiet abgestimmt sind und Daten in hoher Qualität bereitstellen.
Der Copernicus Atmosphere Monitoring Service (CAMS) soll Informationen zur Luftqualität und der chemischen Zusammensetzung der Atmosphäre liefern. 
Daten bezüglich des Zustands und der Dynamik der Meere und deren Ökosysteme lassen sich über den Copernicus Marine Environment Monitoring Service (CMEMS) beziehen. 
Informationen zur Flächennutzung und Bodenbedeckung werden vom Copernicus Land Monitoring Service (CLMS) bereitgestellt. 
Um eine nachhaltige Klimapolitik planen und umsetzen zu können stellt der Copernicus Climate Change Service (C3S) aktuelle sowie historische Klimadaten bereit.  
Um den Zivilschutzbehörden schnelle Reaktionen auf Umweltkatastrophen zu ermöglichen, stellt der Emergency Management Service (EMS) entsprechende Fernerkundungsdaten 
breit. Ähnliche Daten können von europäischen Zoll- und Grenzschutzbehörden über den Copernicus Security Service bezogen werden
\cite{copernicus_regulation}\cite{what_is_copernicus}.

\subsubsection{Sentinel 1}
Die Sentinel-1 Satellitenmission liefert wetter- und tageszeitunabhängige Radardaten der Erdoberfläche. Die Mission besteht aus zwei Satelliten, Sentinel-1 A und B,
sowie einer Bodenkomponente welche für Steuerung und Kalibrierung und Datenverarbeitung verantwortlich ist. Die Satelliten tragen als Hauptinstrument ein 
bildgebendes Radar mit synthetischer Apertur welches im C-Frequenzband arbeitet. Es stehen zwei Polarisationsmodi, Single (HH, VV) oder Dual (HH+HV, VV+VH),
zur Verfügung \cite{sentinel_1_definition}. 
Die Erfassung von Daten kann in vier Aufnahmemodi erfolgen welche sich in Auflösung, Streifenbreite und Anwendungsszenario unterscheiden (siehe Tabelle \ref{aufnahmemodi_sentinel_1}). 
Der Standardmodus ist der Stripmap Modus (SM) bei dem Aufnahmestreifen mit einer kontinuierlichen Folge von Signalen abgetastet wird \cite{sentinel_1_definition}.
Die Aufnahmemodi Interferometric Wide Swath Mode (IW) und Extra-Wide Swath Mode (EW) arbeiten im TOPSAR Verfahren mit drei beziehungsweise
fünf Sub-Aufnahmestreifen um ein größeres Gebiet aber in geringerer Auflösung aufnehmen zu können. TOPSAR ist eine Abwandlung des ScanSAR Verfahrens bei 
dem die Antenne zusätzlich in Azimut-Richtung vor und zurück bewegt wird, um die radiometrische Qualität der resultierenden Bilder zu verbessern. 
Wenn der Wave Modus (WV) zu Einsatz kommt werden kleine, Vignetten genannte, Szenen im Stripmap Verfahren aufgenommen. Sie werden in regelmäßigen Abständen und
wechselnden Depressionwinkeln aufgenommen (siehe Abbildung \ref{sar_modi_sentinel_1})\cite{tutorial_on_sar}\cite{sentinel_1_definition}.   

\begin{figure}[H]
    \centering
    \includegraphics[width=\textwidth]{Bilder/Aquisition_Modes.png}
    \caption{Aufnahmemodi der Sentinel-1 Mission \cite{sentinel_1_overview}}
    \label{sar_modi_sentinel_1}
\end{figure}

\begin{center}
\begin{table}[H]
    \caption{Eigenschaften der Aufnahmemodi der Sentinel-1 Mission \cite{sentinel_1_overview}}
    \centering
    \begin{tabular}{c|c c c c } 
        Modus & IW & WV & SM & EW \\ 
        \hline
        Polarisation & Dual & Single & Dual & Dual \\ 
        Azimutauflösung (m) & 20 & 5 & 5 & 40 \\
        Rage-Auflösung (m) & 5 & 5 & 5 & 20 \\
        Streifenbreite (km) & 250 & 20x20 & 80 & 410\\
    \end{tabular}
    \label{aufnahmemodi_sentinel_1}
\end{table}
\end{center}

Beide Satelliten befinden sich auf einem polnahen, sonnensynchronen Orbit. Ein Zyklus dauert 12 Tage, in denen die Erde 175 umrundet wird. Da er sich um ein Satellitenpaar
handelt welches als Tandem die Erde umrundet wird ein Punkte alle sechs Tage von einem der Satelliten überflogen. Das System kann eine zuverlässige globale und systematische
Abdeckung liefern. Dabei können im IW Modus alle relevanten Land-, Wasser- und Eismassen alle zwölf Tage vollständig von einem Satelliten erfasst werden. 
In Krisensituationen können nach Bedarf innerhalb von zweieinhalb und fünf Tagen Daten erfasst werden \cite{sentinel_1_overview}. 

Nach dem Erfassen der Daten und Übersenden an eine Bodenstation werden diverse Vorverarbeitungsschritte vorgenommen in die sowohl interne also auch externe
Parameter einfließen. Daraus ergeben sich diverse Produkte welche sich durch Aufnahmemodus (IW, SM, EW und WV), Produkt-Typ sowie durch ihre 
Auflösung (Full-, High-, und Medium-Resolution) unterscheiden. Single Look Complex (SLC) Produkte sind im wesentlichen kalibrierte Rohdaten in denen Amplitude und Phase nicht
zur Reflektivität kombiniert wurden und die geometrische Auflösung sich in Azimut- und Blickrichtung unterscheidet. Ground Rage Detected (GRD) Produkte bilden hingegen die 
Reflektivität ab und haben eine annähern quadratische geometrische Auflösung. Die Reflektivität wird in der logarithmischen Maßeinheit Dezibel (dB) angegeben. Die Korrektur 
der Schrägdistanz in Blickrichtung erfolgt durch Projektion auf einen Ellipsoiden. \cite{sentinel_1_definition}. Aus den Level-1 Produkten, SLC und GRD, können die 
Level-2 Produkte OSW, OWI und RVL abgeleitet werden.

\subsubsection{Datenzugang}
Die Daten des Copernicus-Programmes sollen einer möglichst breiten Nutzergruppe möglichst einfach zugänglich gemacht werden. Sie sollen frei zugänglich und kostenlos angeboten 
werden \cite{copernicus_regulation}. Daten der Sentinel-1, 2, 3 und 5 können über das von der ESA betriebene Copernicus Open Access Hub bezogen werden. Datensätze können sowohl
auf der Webseite als auch mithilfe einer API gesucht und heruntergeladen werden. Der Zugang zu Daten der Sentinel-3, 6 und 4 sowie weiterer Satelliten können über das 
dem Copernicus Open Access Hub ähnlichen EUMETCast bezogen werden.
In Ergänzung zu diesen Quellen werden Daten von fünf privaten, in Kooperation mit dem Copernicus-Programm stehenden Unternehmen in unterschiedlichen Formen bereitgestellt. 
Diese als Data and Information Access Services (DIAS) bezeichneten Zugänge stellen unverarbeitete und abgeleitete Daten sowie Werkzeuge zur Analyse zur Verfügung \cite{dias_factsheet}.
Da die DIAS kommerziell betrieben werden müssen einige Dienste und Werkzeuge bezahlt werden während Nutzer sich lediglich am Copernicus Open Access Hub oder EUMETCast 
registrieren müssen. Zu erwähnen ist das die DIAS Zugriff auf die gesamten Daten gestatten. Aus dem Copernicus Open Access Hub lassen sich nur Teile der Daten synchron beziehen.
In der Regel müssen Daten welche älter als einen Monat sind aus dem Archiv wiederhergestellt werden. Dieser Vorgang kann einige Zeit in Anspruch nehmen. 


\subsection{Überschwemmungsmonitoring}
Um Wasserflächen und damit auch überflutete Areale auf Radarbildern zu erkennen können die Reflektionseigenschaften von Wasserflächen genutzt werden. Das Wasser eine 
sehr niedrige Rauigkeit besitzt kommt beim Aufprall eines Radarsignals zu einer spiegelnden Reflektion und nur ein sehr geringer Teil des Signals wird zum Empfänger 
zurückgeworfen. In den resultierenden Bildern äußert sich dieser Umstand in niedrigen Refelektivitätswerten. 
Um die Areale mit niedrigen Reflektionswerten zu detektieren können Verfahren genutzt werden, welche aus den Histogrammen der Bilder einen Schwellwert ermitteln.
Um die Ergebnisse einer solchen Schwellwertbestimmung zu verbessern, sollten die Radardaten, zum Beispiel Sentinel-1 IW GRD, zusätzlich Kalibriert werden. 
So können die genaue Kenntnis über die tatsächliche Flugbahn des Satelliten dazu betragen die geografische Genauigkeit zu verbessern. Diese kann zusätzlich durch Verfahren 
wie die Diffentialentzerrung gesteigert werden die die durch das Relief enstandenen Lagefehler ausgleicht \cite{einfuehrung_in_fernerkundung}.
Die radiometrische Genauigkeit kann gesteigert werden indem zum Beispiel thermisches Rauschen aus den Daten entfernt wird und die Reflektivitätswerte zum 
sogenannten $\sigma_0$-Wert umgerechnet werden. Dieser repräsentiert den Querschnitt der Reflektivität für eine normierte Fläche am Boden \cite{radiometric_calibration_of_S1_level1_products}.
Dieses Maß erlaubt zudem das Vergleichen unterschiedlicher Radaraufnahmen.  
Auch sollte ein Speckle-Filter zum Einsatz kommen um. Dieser reduziert körnige Bildstrukturen welche auf homogenen Flächen in Radarbildern auftreten und die 
rechnerische Bildauswertung erschweren können. \cite{einfuehrung_in_fernerkundung}\cite{sentinel_1_flood_mapping_tutorial}.  
Auf Basis des Schwellwertes kann ein Binärisierung des Bilder durchgeführt werden. Die entstehenden Werte würden überflutete beziehungsweise trocken liegende 
Areale repräsentieren \cite{sentinel_1_flood_mapping_tutorial}.
Die Binärisierung kann direkt auf Basis der Radaraufnahme der Überflutung, oder auf abgeleiteten Daten erfolgen. So können zum Beispiel das Radaraufnahme der überflutung mit 
einer überflutungsfreien Referenzaufnahme kombiniert zum Normalized Difference Sigma-Naught Index (NDSI) \cite{flood_proxy_mapping_ndsi}. Dabei die Reflektivitätswerte von 
zwei unterschiedlichen Zeitpunkten zu einem Maß verrechnet welches als stärke der Veränderung interpretiert werden kann. 

\begin{equation}
    NDSI = \frac{\sigma_0^f-\sigma_0^r}{\sigma_0^f+\sigma_0^r}
\end{equation} 

Dieses Maß bewegt sich zwischen $-1$ und $1$ wobei 
Werte um $0$ für identische Reflektionswerte an beiden Zeitpunkten und daher für geringe Veränderung stehen. 
Aufgrund der Reflektionseigenschaften von Wasserflächen deuten Werte nahe $-1$ auf überflutete Areale hin \cite{flood_proxy_mapping_ndsi}. 

\subsection{Schnittstellen}
\subsection{OGC und OGC Standards}
Das Open Geospatial Consotium (OGC) widmet sich der Aufgabe die Entwicklung von internationalen Standards und unterstützender Dienste welche die Interoperabilität im 
Bereich der Geoinformatik verbessern voranzutreiben. Das OGC soll dabei offene Systeme und Techniken verbreiten welche es erlauben Dienste und Prozesse mit geobezug
in Kreisen der Informatik verbreiten und die Nutzung von interoperabler und kommerzieller Software fördern.


\subsection{OGC API - Processes - Part 1: Core}
Der OGC API - Processes - Part 1: Core Standrad soll das Bereitstellen von aufwendingen Prozessierungsaufgaben und ausführbaren Prozessen welche über eine webbasierte 
Programmierschittstelle von anderen Programmen aufgerufen und gestartet werden können unterstützen \cite{ogc_api_processes_core}. Der Standard ist dabei von Konzepten des 
OGC Web Processing Service 2.0 Interface Standards beeinflusst und bedient sich des RESTful Paradigmas sowie der Java Script Object Notation (JSON)
\subsection{Evaluationskriterien}


