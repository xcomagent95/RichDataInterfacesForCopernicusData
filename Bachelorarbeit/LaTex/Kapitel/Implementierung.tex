\newpage
\restoregeometry
\section{Implementierung}
\subsection{Requirements Class Core}
\subsection{Requirements Class OGC Process Description}
\subsection{Requirements Class JSON}
\subsection{Requirements Class HTML}
\subsection{Requirements Class OpenAPI 3.0}
\subsection{Requirements Class Job List}
\subsection{Requirements Class Dismiss}
\subsection{Zusätzliche Funktionalitäten}
\subsection{Überschwemmungsmonitoring}
Eines dieses Schwellwertverfahren wurde von Nobuyuki Otsu entwickelt und ist nach ihm benannt. Bei diesem Verfahren werden alle Werte eines Histogramms durchlaufen.
Jeder dieser Werte teilt das Histogramm in zwei Gruppen und bildet so einen Schwellwert. Jener Wert welcher die gewichtete Varianz innerhalb der Klassen minimiert.
Gegeben sei ein Bild $C$ mit $N$ Pixeln und $L$ Grauwertstufen. Die Anzahl der Pixel einer Grauwertstufe $i$ sein dann gegeben durch $n_i$ mit:

\begin{equation}
    N = \sum_{i=1}^{L} n_i
\end{equation}
Der momentan betrachtete Grenzwert $t$ teil das Bild in die Gruppen $C_0$ und $C_1$ wobei $C_0$ alle Pixel der Graustufen 1 bis $t$ und 
$C_1$ alle Pixel der Graustufen $t+1$ bis $L$ enthält. Die Gewichte für die Gruppen $C_0$ und $C_1$ sind nun gegeben durch:
\begin{equation}
    w_0(t) = w(t) = \sum_{i=1}^{t} p_i
    \text{   und   }
    w_1(t) = \sum_{i=t+1}^{L} p_i
\end{equation}
Mit pi:
\begin{equation}
    p_i = \frac{n_i}{N}
\end{equation}
und:
\begin{equation}
    \mu_0(t) = \sum_{i=1}^{t} ip_i/w_0
    \text{   und   }
    \mu_1(t) = \sum_{i=t+1}^{L} ip_i/w_1
    \text{   und   }
    \mu_T = \sum_{i=1}^{L} ip_i
\end{equation}
Die Klassenvarianzen sind gegeben durch:
\begin{equation}
    \sigma_0^2(t) =  \sum_{i=1}^{t} (i-\mu_0)^2p_i/w_0
    \text{   und   }
    \sigma_1^2(t) = \sum_{i=t+1}^{L} (i-\mu_1)^2p_i/w_1
\end{equation}
Zu Maximieren ist nun die Inter-Klassenvarianz was equivalent zur Minimierung der Intra-Klassenvarianz ist:
\begin{equation}
    K = \frac{\sigma_t^2}{\sigma_W^2}
\end{equation}
wobei: 
\begin{equation}
    \sigma_W^2 = w_0\sigma_0^2 + w_1\sigma_1^2
    \text{   und   }
    \sigma_T^2 = \sum_{i=1}^{L} (i-\mu_T)^2p_i
\end{equation}
