\newpage
\restoregeometry
\section{Implementierung}
\subsection{Requirements Class Core}
In der Requirements-Class Core werden alle wesentlichen Funktionalitäten beschreiben die eine minimale Implementierung des Standards zur Verfügung stellen muss. 
Dazu gehören Endpunkte zum Beschaffen von Ressourcen, zum Ausführen der angebotenen Prozesse sowie zum Beschaffen der Prozessierungsergebnisse. Auch definiert diese 
Requirements-Class grundlegende Eigenschaften der Implementierung wie die HTTP-Version. 
\subsubsection{API Landig Page}
Die Landing Page der API kann als Zugangspunkt zu allen weiteren Funktionen und Ressourcen der API gesehen werden. Sie ist über den URL \textit{/} mittels der HTTP-Get
Über den Parameter $f$ kann definiert werden ob die Ressource als HTML- oder JSON-Dokument geliefert werden soll. Bei erfolgreicher Ausführung wird mit dem HTTP-Statuscode
200 geantwortet. Wird der Endpunkt mit einer falschen HTTP-Methode aufgerufen wird mit dem HTTP-Statuscode 405 geantwortet. 
Ein erfolgreiches Abrufen der Landing Page mittels einer

\subsection{Requirements Class OGC Process Description}
\subsection{Requirements Class JSON}
\subsection{Requirements Class HTML}
\subsection{Requirements Class OpenAPI 3.0}
\subsection{Requirements Class Job List}
\subsection{Requirements Class Dismiss}
Die Requirements-Class Dismiss beschreibt eine weitere Interaktionsmöglichkeit mit von Nutzern angelegten Jobs. Sie soll dazu dienen angelegte Jobs abzubrechen oder 
Dateien eines abgeschlossenen Jobs zu löschen \cite{ogc_api_processes_core}. 
\subsection{Prozesse}
\subsubsection{Echo}
\subsubsection{Überflutungsmonitoring}
\subsection{Zusätzliche Funktionalitäten}